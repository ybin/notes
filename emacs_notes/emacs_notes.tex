\documentclass[a4paper,11pt]{article}


%%% fontenc
%\usepackage{fontspec,xunicode,xltxtra}
%\setmainfont{Times New Roman}
%\setsansfont{Source Sans Pro}
%\setmonofont{Source Sans Pro}

%%% xeCJK
\usepackage{xeCJK}
\setCJKmainfont[BoldFont=Adobe Heiti Std]{Adobe Song Std}
\setCJKsansfont[BoldFont=Adobe Heiti Std]{Adobe Song Std}
\setCJKmonofont[BoldFont=Adobe Heiti Std]{Adobe Song Std}
\XeTeXlinebreaklocale "zh"
\XeTeXlinebreakskip=0pt plus 1pt minus 0.1pt

\usepackage{xcolor}
\usepackage{graphicx}

%%% get total page number
\usepackage{lastpage}

%%% customized definition
\makeatletter
\def\sybtitle#1{\def\@sybtitle{#1}}
\def\sybauthor#1{\def\@sybauthor{#1}}
\def\sybdate#1{\def\@sybdate{#1}}
\sybtitle{}
\sybauthor{}
\sybdate{}
\def\sybmaketitle{
  \begin{center}
  \vspace*{.8in}
  {\huge\bfseries\@sybtitle}
  \par
  \vspace{.8in}
  {\Large\@sybauthor}
  \par
  \vspace{.2in}
  \@sybdate
  \vspace{.5in}
  \end{center}
}
\makeatother
\setlength{\parindent}{0pt}
\renewcommand{\today}{\number\month 月 \number\day 日, ~\number\year 年}
\def\lt{\textless}
\def\gt{\textgreater}
\renewcommand\contentsname{\bfseries 目~~录}
\newcommand\bs{\texttt{\symbol{'134}}} % input backslash sign
%\newcommand\bs{\string\} % same as above definition
\long\def\cmd#1{\par\vspace{.5em}\hspace*{2em}#1\vspace{.5em}\par}
\def\cstr#1{\texttt{\string#1}} % e.g. \cstr{\latex}
\long\def\runcode#1{\par\bigskip#1\bigskip\par}
% 我不想看到那么多的underful hbox,尤其是minted环境加上背景色之后
\hbadness=10000
% 适当放宽overful hbox的限制,运行2pt的溢出
\hfuzz=2pt
\parskip=3\lineskip


%%% change background color & add frame for enumerate enviroment
\usepackage{mdframed}
\newmdenv[backgroundcolor=blue!10,linewidth=0pt]{coloredframe}
\newenvironment{coloredenumerate}{
  \begin{coloredframe}
  \begin{enumerate}
}{
  \end{enumerate}
  \end{coloredframe}
}

%%% geometry
\usepackage[includehead,includefoot,hmargin=21mm,vmargin=10.5mm,
            headsep=12pt,headheight=25pt]{geometry}
%\usepackage[includehead,includefoot,hmargin=1.2in,vmargin=1in]{geometry}

%%% fancyhdr
\usepackage{fancyhdr}
\makeatletter
\fancypagestyle{main} {
  \fancyhf{} % clear header & footer
  \fancyhead[L]{\bfseries\@sybtitle}
  \fancyhead[R]{\thepage/\pageref*{LastPage}}
  \renewcommand{\headrulewidth}{0.4pt} % header line
  \renewcommand{\footrulewidth}{0pt} % footer line
}
\fancypagestyle{header} {
  \fancyhf{} % clear header & footer
  \fancyfoot[C]{\roman{page}}
  \renewcommand{\headrulewidth}{0pt} % header line
  \renewcommand{\footrulewidth}{0pt} % footer line
}
\makeatother

\usepackage{titlesec}
\titleformat{\part}{\centering\Large\bfseries}{第\,\thepart\,部分}{1em}{}
\titleformat{\section}{\large\bfseries}{\thesection}{1em}{}
\titleformat{\subsection}{\normalsize\bfseries}{\thesubsection}{1em}{}
%\titlespacing*{章节命令}{左边距}{上文距}{下文距}[右边距]
\titlespacing*{\section}{0pt}{2\baselineskip}{\parsep}


\usepackage{hyperref}

%%% perfect source code display
\usepackage{minted}
%\usemintedstyle{colorful}
\definecolor{srcbg}{rgb}{0.95,0.95,0.95}
\newminted{java}{linenos,tabsize=4,bgcolor=srcbg}
\newminted{xml}{linenos,tabsize=4,bgcolor=srcbg}
\newminted{cpp}{linenos,tabsize=4,bgcolor=srcbg}
\newminted{bash}{linenos,tabsize=4,bgcolor=srcbg}
\newminted{latex}{linenos,tabsize=4,bgcolor=srcbg}
\newminted{scheme}{linenos,tabsize=4,bgcolor=srcbg}

\usepackage{amsmath}


\input{../styles/tikz_preamble}

\sybtitle{Emacs Notes}
\sybauthor{孙延宾}
\sybdate{\today}

\begin{document}
  \tt % I love Typewriter font.
%%%%%%%% the title page and toc %%%%%%%%%%
  \pagestyle{header}
  \sybmaketitle
  \tableofcontents
  \newpage

%%%%%%% the main content %%%%%%%%%
  \pagestyle{main}
  \setcounter{page}{1}
  
  \part[Emacs函数列表]{Emacs函数列表}
  \section[eval-after-load]{eval-after-load}
  eval-after-load函数只

  \section[shortcuts list]{shortcuts list}
  prefix command + C-h : 列出所有隶属于该prefix的快捷键,
  如C-x C-h将会列出所有C-x开头的快捷键

  mode, mode-map, mode-hook
  The bindings between key sequences and command functions are recorded in data structures called keymaps.
  Each major or minor mode can have ites own keymap which overrides the global definitions of some keys.

  Textinfo shortcuts:\\
  Gnu Info分dir mode和manual mode,顶部的(dir)和(manual name)可以标识。
  space: 向下滚屏,达到末尾时自动进入下一个文档节点\\
  backspace: 与space相反
  l(L): 回溯浏览历史
  n: 进入下一个同级别的节点
  p: 与n相反
  t: 进入顶级节点(manual的顶部)
  d: top dir, dir mode的顶部(注意跟t区分)
  u: 进入上一级节点
  b: 移动到本节点的顶端
  m: 列出下一级节点的菜单(menu),如mEmacs可以进入Emacs manual, mInfo进入Info manual
  f: 进入交叉链接(l可以返回原节点)
  i: 列出文档所有索引标题
  s: 搜索文档

  setq-default always set variables global value instead of local value.
  default-value get default value, e.g. global value.

  Emacs内部有自己的编码方案,该方案是Unicode的超集,它可以编码几乎所有
  语言。在读取、写入文件时可以实现编码方式的自动转换。

  C-h h会打开一个hello文件,记录了所有语种的''hello''翻译。

  C-q -> quote-insert

  C-x = -> what-cursor-position,显示当前光标处字符的信息,对于多字节字符,
  Emacs在charactor code的最后显示为'file...'
  更详细的信息可以通过''C-u C-x =''来显示。

  不要单独设置一个''语言'',而是设置一种''语言环境'',一个语言环境设置了很多
  默认参数。Choose a language environment instead of language.
  语言环境用于读取文件内容时对文件的编码方案进行识别。
  
  变量current-language-environment用来标识当前的语言环境(识别当前语言环境时使用该变量);
  函数set-language-environment用来设置语言环境(修改语言环境时应该使用该函数)

  language-info-alist变量保当前所有的语言环境信息,每一种语言环境都包含多个变量,
  如coding system, input method等,可以定义自己的语言环境,然后添加到该alist中即可。
  
  describe-language-environment函数用于显示某种语言环境的信息,默认参数使用
  current choice,即当前正在使用的语言环境。
  注意:语言环境的信息在mode line的第一部分显示,如:
  "U"表示UTF-8,''c''表示chinese-GBK等。
  顺便说一下,mode line第一部分的组成(鼠标放上去之后会有提示):\\
  <coding system>(end-of-line style)<is writable><is modified><file path>\\
  <file path>一般用''-''代替。

  定制语言环境:\\
  set-language-environment切换语言环境时会调用两个hook里面的函数:\\
  exit-language-environment-hook\\
  set-language-environment\\
  他们按照顺序执行,所以定制某一种语言环境时可以在set hook中设置,
  在exit hook中还原,另外使用current-language-environment变量
  识别当前所在的语言环境。

  Emacs原来自带拼音输入法啊! C-\ -> toggle-input-method

  修改input method:\\
  set-input-method修改输入法;list-input-methods列出所有可用输入法。
  
  输入法的原因是某些字符键盘上是没有的(中文字符一个都没有),需要通过某种方法
  进行转换,转换方法一般有三种:\\
  简单转换:将可见ascii字符映射为其他字符,一一映射。该方法适用于映射的目标集合小于等于
  ascii可见字符集合的情况。\\
  组合转换:多个ascii可见字符组合起来代表一个特殊字符,如法语,''o\^{}''表示$\hat{o}$
  简单+组合:将两种方法结合起来,如中文拼音。

  通过Unicode吗直接输入字符:\\
  C-x 8 <RET> -> insert-char,然后输入十六进制的Unicode码即可,
  如5BBE将会输入''宾''字符。

  \section[Emacs Coding System]{Emacs Coding System}
  Emacs编码系统的常用命令:\\
  list-coding-systems: 列出当前列表编码系统列表
  describe-coding-system: 打印当前编码系统的详细信息

  Possible completions are:
  describe-bindings 	describe-buffer-case-table
  describe-categories 	describe-char
  describe-char-after 	describe-character-set
  describe-class 	describe-coding-system
  describe-copying 	describe-current-coding-system
  describe-current-coding-system-briefly 	describe-current-display-table
  describe-distribution 	describe-face
  describe-font 	describe-fontset
  describe-function 	describe-generic
  describe-gnu-project 	describe-input-method
  describe-key 	describe-key-briefly
  describe-language-environment 	describe-minor-mode
  describe-minor-mode-from-indicator 	describe-minor-mode-from-symbol
  describe-mode 	describe-no-warranty
  describe-package 	describe-prefix-bindings
  describe-project 	describe-specified-language-support
  describe-syntax 	describe-text-category
  describe-text-properties 	describe-theme
  describe-variable

  每一种编码方案都有三个变种(variat):--dos, --unix, --mac

  在读取文件时,Emacs按照一个编码方案列表逐个测试,可以使用 prefer-coding-system 函数
  将某个编码方案放置到列表的前头,从而使得该编码方案被优先使用(如果不匹配,Emacs仍然会
  继续测试剩余的编码方案,而不会一定就是用该preferred编码方案)

  file-coding-system-alist变量保存了文件与编码方案的关联,可以单独设置编码、解码方案,也可
  编解码使用同一个方案,函数 modify-coding-system-alist可以修改该变量。

  使用 revert-buffer-with-coding-system函数可以使用指定的编码方案重新渲染(revisit)当前buffer

  一旦Emacs识别出文件的编码方案之后,就将该编码方案名称保变量 buffer-file-coding-system 中。

  设置buffer的编码方案: set-buffer-file-coding-system函数。

  universal-coding-system-argument函数 用指定的参数来影响“紧接下来的”命令,如C-x i(insert-file),
  时使用该函数并指定编码方案作为参数,可以以指定的编码方案编码insert进来的文件内容。

  当Emacs创建新文件时,会使用 buffer-file-coding-system 变量指定的编码方案处理新文件,注意,
  这个变量是buffer-local的,所以要使用 setq-default 修改全局变量才会生效,也正因为它是buffer-local的,
  所以这样修改并不影响打开其他编码方案的文件,这样的文件的buffer-local buffer-file-coding-system值
  依然正确。


  \section[查找替换]{查找替换}
  一次性替换所有匹配: replace-string 函数。

  查找替换: query-replace, M-%,进入查找替换模式之后,使用下列命令进行操作:\\
  y: 替换当前并前进到下一个匹配处\\
  n: 忽略当前并前进到下一个匹配处\\
  !: 替换所有匹配\\
  \^{}: 回到前一个匹配\\
  q: 退出\\
  e: 修改新字符串\\
  ?: 显示帮助信息\\
  任意其他键退出查找替换模式。

  C-x ESC ESC 重新执行之前的命令。

  \section[自定义当前行高亮模式]{自定义当前行高亮模式}
  自定义方式:\\
  M-x customize-face RET hightlight RET

  能否定义当前行透明呢,这样就可以保持代码高亮的颜色了。


  \section[buffer相关的函数]{buffer相关的函数}
  buffer-name: 返回buffer的name,默认返回当前buffer的name;\\
  buffer-file-name: 返回buffer关联的file name,默认为当前buffer;\\
  current-buffer: 返回当前buffer对象;\\
  other-buffer: 返回其他buffer对象,默认为最近离开的buffer;\\
  switch-to-buffer: 切换到其他buffer,参数为buffer对象;\\
  buffer-size: buffer的size;\\
  point: 当前光标在buffer中的位置;\\
  point-min: 光标在buffer中可以到达的最小位置;\\
  point-max: 光标在buffer中可以到达的最大位置;\\
  

  \section[define interactive function]{define interactive function}
  interactive function指的是可以通过M-x或者绑定快捷键调用的函数,定义该类函数
  只需在普通函数的函数体开始处调用interactive函数即可。

  interactive函数的原型是这样的:\\
  (interactive \&optional ARGS)\\
  ARGS一般为一个字符串,该字符串由code letter(一些表示特殊含义的字母)和
  提示字符串组成,如"sPlease input a string: ",每个code letter可以跟
  一个提示字符串,多组code letter(一个code letter和其提示字符串算作一组)
  之间用换行符'\bs n'隔开。例如:\\
  "nPlease input a number: \bs nsPlease input a string: "\\
  函数的完整定义如下:\\
  \begin{schemecode}
(defun interactive-demo (number string)
  "Receive a number and a string, then show them."
  (interactive "nPlease input a number: \bs nsPlease input a string: ")
  (message "You have input a number '%d' and a string '%s'" number string))
  \end{schemecode}

  \section[special forms]{special forms}
  defun

  let

  if

  save-excursion

  \section[C-x C-x]{C-x C-x}
  切换point与mark,同时选中期间的区域。

  选中期间的区域很关键,如M-<先在当前设置mark然后移动point到buffer的开始处,
  此时C-x C-x就会选中光标直到文件开头的区域。


  \part[Emacs Mode]{Emacs Mode}

  \section[paredit mode key binding]{paredit mode key binding}
  paredit对于lisper们来说真是太重要了,它操纵圆括号出神入化。

  单词解释:\par
  slurp: 吞(非常夸张的狂吃的那种)\\
  barf : 吐(疯狂呕吐的那种)\\[10pt]
  split: 分割、分裂\\
  splice: 拼接,合并
  
  \subsection[输入篇]{输入篇}
  \subsubsection[输入"("]{输入"("}
  输入“(”,自动补全“)”,注释及字符串中除外
  \subsubsection[输入")"]{输入")"}
  输入“)”,自动跳转到下一个“)”之后(跳出括号),注释及字符串中除外
  \subsubsection[输入"M-("]{输入"M-("}
  “M-)”,自动跳转到下一个“)”之后并换行、缩进,注释及字符串中除外

  \subsection[删除篇]{删除篇}
  \subsubsection[删除两个词之间的空格]{删除两个词之间的空格}
  删除两个word之间的空格(emacs, not paredit),即拼接两个word: M-\bs \\
  例如:aaa  |   bbb => aaabbb
  \subsubsection[删除整个list]{删除整个list}
  没有快捷键可以直接删除整个list,但是可以通过两个组合键完成:\\
  C-k -> C-d\\
  这要求光标处于list的第一个元素上。

  \subsection[移动篇]{移动篇}
  将list作为一个word进行移动,直接跨过真个list,如果没有list则依次移动一个word。
  C-M-f, C-M-b\\
  forward-sexp, backward-sexp\\
  
  \subsection[深度篇]{深度篇}
  \subsubsection[按照深度拼接列表]{按照深度拼接列表}
  将子列表的内容拼接到上级列表中:M-s\\
  例如:(foo bar (baz |baby) chop bee) => (foo bar baz baby chop bee)

  \subsubsection[提升子列表]{提升子列表}
  提升子列表,同时将上级列表的其他内容删除:M-r(raise)\\
  例如:(foo (bar |baz) bee) => (bar baz)

  \subsubsection[插入子列表]{插入子列表}
  插入子列表,并将光标后的word/list作为其内容:M-(\\
  例如:(foo bar |baz) => (foo bar (baz))

  \subsection[barfage \& slurpage]{barfage \& slurpage}
  列表吞吐分为两类:向前吞吐(操作"(")、向后吞吐(操作")")\\
  \subsubsection[向前吞吐]{向前吞吐}
  向前吞:C-M-<left>\\
  例如:(foo bar (baz |bee)) => (foo (bar baz |bee))
  
  向前吐:C-M-<right>\\
  例如:(foo bar (baz |bee)) => (foo bar baz (|bee))
  \subsubsection[向后吞吐]{向后吞吐}
  向后吞:C-<right>\\
  例如:(foo (bar |baz) bee) => (foo (bar |baz bee))

  向后吐:C-<left>\\
  例如:(foo (bar |baz) bee) => (foo (bar|) baz bee)
  
  \subsection[其他]{其他}


  \section[按照某个字符对齐]{按照某个字符对齐}
  如果要按照某个字符进行对齐,如等号,可这样操作:

  1. 选中需要对齐的文本

  2. M-x align-regexp RET <your sign> RET


  
  
\end{document}