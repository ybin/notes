\documentclass[a4paper,11pt]{article}


%%% fontenc
%\usepackage{fontspec,xunicode,xltxtra}
%\setmainfont{Times New Roman}
%\setsansfont{Source Sans Pro}
%\setmonofont{Source Sans Pro}

%%% xeCJK
\usepackage{xeCJK}
\setCJKmainfont[BoldFont=Adobe Heiti Std]{Adobe Song Std}
\setCJKsansfont[BoldFont=Adobe Heiti Std]{Adobe Song Std}
\setCJKmonofont[BoldFont=Adobe Heiti Std]{Adobe Song Std}
\XeTeXlinebreaklocale "zh"
\XeTeXlinebreakskip=0pt plus 1pt minus 0.1pt

\usepackage{xcolor}
\usepackage{graphicx}

%%% get total page number
\usepackage{lastpage}

%%% customized definition
\makeatletter
\def\sybtitle#1{\def\@sybtitle{#1}}
\def\sybauthor#1{\def\@sybauthor{#1}}
\def\sybdate#1{\def\@sybdate{#1}}
\sybtitle{}
\sybauthor{}
\sybdate{}
\def\sybmaketitle{
  \begin{center}
  \vspace*{.8in}
  {\huge\bfseries\@sybtitle}
  \par
  \vspace{.8in}
  {\Large\@sybauthor}
  \par
  \vspace{.2in}
  \@sybdate
  \vspace{.5in}
  \end{center}
}
\makeatother
\setlength{\parindent}{0pt}
\renewcommand{\today}{\number\month 月 \number\day 日, ~\number\year 年}
\def\lt{\textless}
\def\gt{\textgreater}
\renewcommand\contentsname{\bfseries 目~~录}
\newcommand\bs{\texttt{\symbol{'134}}} % input backslash sign
%\newcommand\bs{\string\} % same as above definition
\long\def\cmd#1{\par\vspace{.5em}\hspace*{2em}#1\vspace{.5em}\par}
\def\cstr#1{\texttt{\string#1}} % e.g. \cstr{\latex}
\long\def\runcode#1{\par\bigskip#1\bigskip\par}
% 我不想看到那么多的underful hbox,尤其是minted环境加上背景色之后
\hbadness=10000
% 适当放宽overful hbox的限制,运行2pt的溢出
\hfuzz=2pt
\parskip=3\lineskip


%%% change background color & add frame for enumerate enviroment
\usepackage{mdframed}
\newmdenv[backgroundcolor=blue!10,linewidth=0pt]{coloredframe}
\newenvironment{coloredenumerate}{
  \begin{coloredframe}
  \begin{enumerate}
}{
  \end{enumerate}
  \end{coloredframe}
}

%%% geometry
\usepackage[includehead,includefoot,hmargin=21mm,vmargin=10.5mm,
            headsep=12pt,headheight=25pt]{geometry}
%\usepackage[includehead,includefoot,hmargin=1.2in,vmargin=1in]{geometry}

%%% fancyhdr
\usepackage{fancyhdr}
\makeatletter
\fancypagestyle{main} {
  \fancyhf{} % clear header & footer
  \fancyhead[L]{\bfseries\@sybtitle}
  \fancyhead[R]{\thepage/\pageref*{LastPage}}
  \renewcommand{\headrulewidth}{0.4pt} % header line
  \renewcommand{\footrulewidth}{0pt} % footer line
}
\fancypagestyle{header} {
  \fancyhf{} % clear header & footer
  \fancyfoot[C]{\roman{page}}
  \renewcommand{\headrulewidth}{0pt} % header line
  \renewcommand{\footrulewidth}{0pt} % footer line
}
\makeatother

\usepackage{titlesec}
\titleformat{\part}{\centering\Large\bfseries}{第\,\thepart\,部分}{1em}{}
\titleformat{\section}{\large\bfseries}{\thesection}{1em}{}
\titleformat{\subsection}{\normalsize\bfseries}{\thesubsection}{1em}{}
%\titlespacing*{章节命令}{左边距}{上文距}{下文距}[右边距]
\titlespacing*{\section}{0pt}{2\baselineskip}{\parsep}


\usepackage{hyperref}

%%% perfect source code display
\usepackage{minted}
%\usemintedstyle{colorful}
\definecolor{srcbg}{rgb}{0.95,0.95,0.95}
\newminted{java}{linenos,tabsize=4,bgcolor=srcbg}
\newminted{xml}{linenos,tabsize=4,bgcolor=srcbg}
\newminted{cpp}{linenos,tabsize=4,bgcolor=srcbg}
\newminted{bash}{linenos,tabsize=4,bgcolor=srcbg}
\newminted{latex}{linenos,tabsize=4,bgcolor=srcbg}



\sybtitle{Linux Notes}
\sybauthor{孙延宾}
\sybdate{\today}

\begin{document}
  \tt % I love Typewriter font.
%%%%%%%% the title page and toc %%%%%%%%%%
  \pagestyle{header}
  \sybmaketitle
  \tableofcontents
  \newpage

%%%%%%% the main content %%%%%%%%%
  \pagestyle{main}
  \setcounter{page}{1}

  \part[Useful Commands]{Useful Commands}
  \section[find命令使用举例]{find命令使用举例}
  find是Linux上的重量级的命令,使用方法多样且比较复杂,以下分节举例说明其使用方法。
  内容主要来源于thegeekstuff.com,find命令只是通过各种条件查找文件,而不会深入到文件
  中去搜索,要进行文件内容的搜索请使用grep命令。\par
  \bigskip
  FAQ:文件名何时该用双引号括起来,何时又可以省略呢?\\
  答:使用双引号是为了防止文件名中包含空格,尤其是包含通配符时,如"*.zip",其他情况下
  可以省略双引号。

  \subsection[通过文件名查找文件]{通过文件名查找文件}
  默认搜索路径为当前路径:
  \cmd{find -name "myfile.c"}

  忽略文件名的大小写:
  \cmd{find -iname "myfile.c"}

  \subsection[按照文件类型查找]{按照文件类型查找}
  查找当前目录下的socket文件:
  \cmd{find . -type s}
  查找当前目录下的所有目录:
  \cmd{find . -type d}
  查找当前目录下的一般文件:
  \cmd{find . -type f}
  查找当前目录下的隐藏文件(Linux上以“.”开头的文件为隐藏文件):
  \cmd{find . -type f -name ".*"}
  查找当前目录下的隐藏目录:
  \cmd{find . -type d -name ".*"}

  \subsection[相反匹配]{相反匹配}
  -not可以将“紧随”其后的(一个)条件置反。
  \cmd{find -not -iname "myfile.c"}

  \subsection[查找空文件]{查找空文件}
  查找当前目录下的所有空文件:
  \cmd{find . -empty}

  \subsection[按照文件大小查找]{按照文件大小查找}
  文件大小的排序需要借助其他命令,不过Linux shell就是干这个的,
  管道可不是徒有虚名哦!

  查找最大的5个文件:
  \cmd{find . -type f -exec ls -s \{\} \bs; | sort -n -r | head -5}
  ls的-s表示打印“文件”的大小;sort的-n表示按照number计算大小,-r表示reverse。

  查找最小的5个非空文件:
  \cmd{find . -not -empty -type f -exec ls -s \{\} \bs; | sort -n | head -5}

  其实find本身提供-size选项:
  查找大于100M的文件:
  \cmd{find . -size +100M}
  查找小于100M的文件:
  \cmd{find . -size -100M}
  查找等于100M的文件:
  \cmd{find . -size 100M}

  \subsection[通过与其他文件比较修改时间查找文件]{通过与其他文件比较修改时间查找文件}
  这个功能确实比较少见。

  查找比myfile.c更新的文件:
  \cmd{find -newer "myfile.c"}


  \subsection[限定搜索目录的深度]{限定搜索目录的深度}
  -mindepth、-maxdepth可以限定find命令的搜索最小、最大深度,
  1表示当前搜索目录,数字表示“层级数”而非“第几级”:
  \cmd{find -mindepth 3 -maxdepth 5 -name "myfile.c"}

  \subsection[在查找到的文件上执行命令]{在查找到的文件上执行命令}
  可以这样理解:"\{\}"代表查找到的文件名,"\{\}"与"\bs;"之间的内容为要执行命令的参数。
  \cmd{find -iname "myfile.c" -exec mv \{\} myfile.c.new \bs;}
  充分利用"\{\}":
  \cmd{find -iname "myfile.c" -exec mv \{\} \{\}.new \bs;}

  \subsection[find与alias联合]{find与alias联合}
  删除大于100M的zip文件可以这样做:
  \cmd{find . -type f -name *.zip -size +100M -exec rm -i \{\} \bs;}
  然后将这个命令取个名字,方便以后使用:
  \cmd{alias rm100m="find . -type f -name \bs"*.zip\bs" -size +100M -exec rm -i \{\} \bs;"}
  类似的:
  \cmd{alias rm1g="find . -type f -name \bs"*.zip\bs" -size +1G -exec rm -i \{\} \bs;"}

  \subsection[执行多条命令]{执行多条命令}
  类似于filter(过滤器),满足条件N就执行命令N。
  \cmd{find . \bs( -name "*.java" -exec "java file: \{\} \bs; \bs), \bs\\
        \hspace*{15ex}\bs( -name "*.xml" -exec "xml file: \{\} \bs; \bs)}
  注意:\par
  \bs(、\bs)、\bs;以及中间的逗号,他们两边的空格不能丢,"*.java"、"*.xml"两边的引号也不能丢,这点儿比较蛋疼啊!


  \section[grep命令使用举例]{grep命令使用举例}
  grep命令用于从文件中搜索字符串,基本命令格式:
  \cmd{grep [options] <PATTERN> <file-list>}
  其中<PATTERN>为正则表达式,<file-list>可以使用通配符,如"*.c"。
  
  \subsection[在文件中搜索指定字符串]{在文件中搜索指定字符串}
  在file.c、file.h两个文件中搜索字符串"return":
  \cmd{grep return file.c file.h}
  或者:
  \cmd{grep "return" file.c file.h}
  什么时候pattern需要使用引号呢?当pattern中包含空格的时候!
  
  \subsection[使用正则表达式]{使用正则表达式}
  pattern使用正则表达式:
  \cmd{grep "return .*;" file.c file.h}
  
  \subsection[使用通配符]{使用通配符}
  file list可以使用通配符,如在当前目录(不包括子目录)中搜索字符串:
  \cmd{grep "return" *}
  
  \subsection[使用options]{使用options}
  grep的options可以做很多事情:
  \begin{coloredenumerate}
    \item -r, --recursive: 递归搜索
    \item -w, --word-regexp: 强制pattern匹配单个word
    \item -i, --ignore-case: 忽略pattern的大小写
    \item -v, --invert-match: 选中未选中的行
    \item -n, --line-number: 在输出中显示行所在的行号
    \item -l, --files-with-matches: 只列出包含pattern的文件名
    \item -c, --count: 显示pattern在文件中的匹配数量
  \end{coloredenumerate}
   
  \subsection[综合起来]{综合起来}
  在当前目录下,递归搜索所有文件,查找字符串:
  \cmd{grep -r "is.*file" *}
  注意正则表达式和通配符的区别!


  \section[chmod命令修改文件权限]{chmod命令修改文件权限}
  文件权限的表示:\par
  \begin{center}
  \begin{tabular}{|c|c|c|}
    \hline
    权限 & 字母表示 & 数字表示 \\\hline
    read & r & 4 \\\hline
    write & w & 2 \\\hline
    execute & x & 1 \\\hline
  \end{tabular}
  \end{center}

  文件权限的类别:\par
  \begin{center}
  \begin{tabular}{|c|c|}
    \hline
    权限类别 & 字母表示 \\\hline
    user & u \\\hline
    group & g \\\hline
    other & o \\\hline
  \end{tabular}
  \end{center}

  使用chmod修改文件权限:
  \cmd{chmod \lt 类别\gt \lt +|-\gt\lt 权限\gt\ file1 file2 ...}
  “类别”的可选项有u、g、o和a(all),权限的可选项有r、w、x,
  “+”表示增加权限,“-”表示减掉权限。

  例如:\par
  为所有类别赋予可执行权限:
  \cmd{chmod a+x filename}
  增加group的写入权限、去掉other的执行权限:
  \cmd{chmod g+w,o-x file1 file2 ...}
  注意:“g+w,o-x”之间不要有空格!

  当然,也可以使用数字计算出各个类别的权限,然后一次性
  处理这些权限,如user可读写、group只读、other只读:
  \cmd{chmod 644 file1 file2 ...}


\end{document}
