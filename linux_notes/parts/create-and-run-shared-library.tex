\section[Shared library(.so)]{创建并使用动态链接库(.so)}
动态链接库的创建及使用大致有以下步骤:
\begin{enumerate}
  \item 创建动态链接库代码
  \item 创建测试程序代码
  \item 创建Makefile文件
\end{enumerate}

\subsection[code for .so file]{创建动态链接库代码}
我们创建一个名为"shared"的动态链接库,代码如下,

\begin{cppcode}
/* shared.h */
void say_hi();

/* shared.c */
#include <stdio.h>
#include "shared.h"
void say_hi() {
  printf("Hi, dynamic linked library!\n");
}
\end{cppcode}

\subsection[code for test program]{创建测试程序代码}
\begin{cppcode}
/* test.c */
#include "shared.h"
int main()
{
  say_hi();
}
\end{cppcode}

\subsection[create Makefile]{创建Makefile文件}
代码并没有什么稀奇的,稀松平常,关键是Makefile文件,

\begin{makecode}
# Step 1. 注意-L和-l两个参数,-l<xxx>,对应于libxxx.so
test : libshared.so
  gcc -L/path/to/libshared.so/ -lshared -Wall test.c -o test

# Step 0. 注意-shared和-fPIC两个参数
libshared.so : shared.o
  gcc -shared -o libshared.so shared.o
shared.o : shared.c shared.h
  gcc -c -Wall -fPIC shared.c
\end{makecode}

动态链接库的关键特性就是runtime时载入,所以要运行test程序,
就要让系统在运行时找到libshared.so文件才行,这通过设置环境变量
来实现,

\begin{bashcode}
export LD_LIBRARY_PATH=/path/to/test/program/:$LD_LIBRARY_PATH
# 运行测试程序
./test
\end{bashcode}