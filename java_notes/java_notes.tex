
\documentclass[a4paper,11pt]{article}


%%% fontenc
%\usepackage{fontspec,xunicode,xltxtra}
%\setmainfont{Times New Roman}
%\setsansfont{Source Sans Pro}
%\setmonofont{Source Sans Pro}

%%% xeCJK
\usepackage{xeCJK}
\setCJKmainfont[BoldFont=Adobe Heiti Std]{Adobe Song Std}
\setCJKsansfont[BoldFont=Adobe Heiti Std]{Adobe Song Std}
\setCJKmonofont[BoldFont=Adobe Heiti Std]{Adobe Song Std}
\XeTeXlinebreaklocale "zh"
\XeTeXlinebreakskip=0pt plus 1pt minus 0.1pt

\usepackage{xcolor}
\usepackage{graphicx}

%%% get total page number
\usepackage{lastpage}

%%% customized definition
\makeatletter
\def\sybtitle#1{\def\@sybtitle{#1}}
\def\sybauthor#1{\def\@sybauthor{#1}}
\def\sybdate#1{\def\@sybdate{#1}}
\sybtitle{}
\sybauthor{}
\sybdate{}
\def\sybmaketitle{
  \begin{center}
  \vspace*{.8in}
  {\huge\bfseries\@sybtitle}
  \par
  \vspace{.8in}
  {\Large\@sybauthor}
  \par
  \vspace{.2in}
  \@sybdate
  \vspace{.5in}
  \end{center}
}
\makeatother
\setlength{\parindent}{0pt}
\renewcommand{\today}{\number\month 月 \number\day 日, ~\number\year 年}
\def\lt{\textless}
\def\gt{\textgreater}
\renewcommand\contentsname{\bfseries 目~~录}
\newcommand\bs{\texttt{\symbol{'134}}} % input backslash sign
%\newcommand\bs{\string\} % same as above definition
\long\def\cmd#1{\par\vspace{.5em}\hspace*{2em}#1\vspace{.5em}\par}
\def\cstr#1{\texttt{\string#1}} % e.g. \cstr{\latex}
\long\def\runcode#1{\par\bigskip#1\bigskip\par}
% 我不想看到那么多的underful hbox,尤其是minted环境加上背景色之后
\hbadness=10000
% 适当放宽overful hbox的限制,运行2pt的溢出
\hfuzz=2pt
\parskip=3\lineskip


%%% change background color & add frame for enumerate enviroment
\usepackage{mdframed}
\newmdenv[backgroundcolor=blue!10,linewidth=0pt]{coloredframe}
\newenvironment{coloredenumerate}{
  \begin{coloredframe}
  \begin{enumerate}
}{
  \end{enumerate}
  \end{coloredframe}
}

%%% geometry
\usepackage[includehead,includefoot,hmargin=21mm,vmargin=10.5mm,
            headsep=12pt,headheight=25pt]{geometry}
%\usepackage[includehead,includefoot,hmargin=1.2in,vmargin=1in]{geometry}

%%% fancyhdr
\usepackage{fancyhdr}
\makeatletter
\fancypagestyle{main} {
  \fancyhf{} % clear header & footer
  \fancyhead[L]{\bfseries\@sybtitle}
  \fancyhead[R]{\thepage/\pageref*{LastPage}}
  \renewcommand{\headrulewidth}{0.4pt} % header line
  \renewcommand{\footrulewidth}{0pt} % footer line
}
\fancypagestyle{header} {
  \fancyhf{} % clear header & footer
  \fancyfoot[C]{\roman{page}}
  \renewcommand{\headrulewidth}{0pt} % header line
  \renewcommand{\footrulewidth}{0pt} % footer line
}
\makeatother

\usepackage{titlesec}
\titleformat{\part}{\centering\Large\bfseries}{第\,\thepart\,部分}{1em}{}
\titleformat{\section}{\large\bfseries}{\thesection}{1em}{}
\titleformat{\subsection}{\normalsize\bfseries}{\thesubsection}{1em}{}
%\titlespacing*{章节命令}{左边距}{上文距}{下文距}[右边距]
\titlespacing*{\section}{0pt}{2\baselineskip}{\parsep}


\usepackage{hyperref}

%%% perfect source code display
\usepackage{minted}
%\usemintedstyle{colorful}
\definecolor{srcbg}{rgb}{0.95,0.95,0.95}
\newminted{java}{linenos,tabsize=4,bgcolor=srcbg}
\newminted{xml}{linenos,tabsize=4,bgcolor=srcbg}
\newminted{cpp}{linenos,tabsize=4,bgcolor=srcbg}
\newminted{bash}{linenos,tabsize=4,bgcolor=srcbg}
\newminted{latex}{linenos,tabsize=4,bgcolor=srcbg}
\newminted{scheme}{linenos,tabsize=4,bgcolor=srcbg}

\usepackage{amsmath}


\input{../styles/tikz_preamble}

\sybtitle{Java Notes}
\sybauthor{孙延宾}
\sybdate{\today}

\begin{document}
\tt % I love Typewriter font.
%%%%%%%% the title page and toc %%%%%%%%%%
\pagestyle{header}
\sybmaketitle
\tableofcontents
\newpage

%%%%%%% the main content %%%%%%%%%
\pagestyle{main}
\setcounter{page}{1}

\part[Basic Usage]{Basic Usage}
\section[Enumeration vs Iterator]{Enumeration vs Iterator}
Enumeration和Iterator都是用于遍历集合,Enumeration主要用于遍历
Vector、HashTable,而Iterator用于遍历其他集合。两者的主要区别是:
Enumeration不可以在遍历过程中删除元素,而Iterator则可以,
比较Enumeration和Iterator的定义就可以明确这一点。

下面是Enumeration的定义,

\begin{javacode}
public interface Enumeration<E> {
  boolean hasMoreElements();
  E nextElement();
}
\end{javacode}

下面是Iterator的定义,

\begin{javacode}
public interface Iterator<E> {
  boolean hasNext();
  E next();
  void remove();
}
\end{javacode}

另附使用实例(Vector既可以Enumeration遍历也可用Iterator遍历),

\begin{javacode}
Vector<String> v = new Vector<String>();
v.add("A");
v.add("B");
v.add("C");

// 使用Enumeration遍历
// 'for' version
String elem;
for(Enumeration<String> e = v.elements(); e.hasMoreElements(); ) {
  elem = e.nextElement();
  System.out.println(elem);
}
// 'while' version
Enumeration<String> e = v.elements();
while(e.hasMoreElements()) {
  elem = e.nextElement();
  System.out.println(elem);
}

// 使用Iterator遍历
// 'for' version
String cur;
for(Iterator<String> iter = v.iterator(); iter.hasNext(); ) {
  cur = iter.next();
  System.out.println(cur);
  if("B".equals(cur)) {
    iter.remove();
  }
}
// 'while' version
Iterator<String> iter = v.iterator();
while(iter.hasNext()) {
  cur = iter.next();
  System.out.println(cur);
  if("B".equals(cur)) {
    iter.remove();
  }
}
\end{javacode}

\section[遍历Map]{遍历Map}
遍历Map时需要从collection的视角去看Map。

\begin{javacode}
public void operateMap() {
  HashMap<String, String> info = new HashMap<String, String>() {
    private static final long serialVersionUID = 1L;
    {
      put("John", "42");
      put("Eric", "43");
      put("Waston", "44");
    }
  };

  /*
   * Collection<E> interface extends Iterable<T> interface,
   * so all Collection including Set<E>, List<E>, Queue<E>
   * can get an Iterator via iterator() method.
   */

  // 遍历Map - 1
  Iterator<Entry<String, String>> entryIter = info.entrySet().iterator();
  while(entryIter.hasNext()) {
    Entry<String, String> entry = entryIter.next();
    System.out.println("key: " + entry.getKey() + ", value: " + entry.getValue());
  }

  // 遍历Map - 2
  Iterator<String> keyIter = info.keySet().iterator();
  while(keyIter.hasNext()) {
    String key = keyIter.next();
    String value = info.get(key);
    System.out.println("key: " + key + ", value: " + value);
  }

  // 遍历Map - 3,遍历Value
  Iterator<String> valueIter = info.values().iterator();
  while(valueIter.hasNext()) {
    String value = valueIter.next();
    System.out.println("value: " + value);
  }
}
\end{javacode}

\section[Print system properties]{Print system properties}
\begin{javacode}
public void printSystemProperties() {
  Properties p = System.getProperties();
  Enumeration<Object> e = p.keys();
  String key;

  for( ; e.hasMoreElements(); ) {
    key = (String) e.nextElement();
    System.out.println(key + " = " + p.getProperty(key));
  }
}
\end{javacode}

\section[Inner Class]{Inner Class}
Java中,Inner class的field/method相当于Outer class的field/method,
只不过要通过Inner class或Inner class的对象对其进行访问而已。

\subsection[内部类的可见性]{内部类的可见性}
内部类的field/method跟外部类的field/method本质上是一样的,只是访问
方式不同而已,所以内部类和外部类相互之间是“完全透明”的。

\subsection[Create object of Inner class ]{Create object of Inner class}
由于non-static inner class依附于outer class而存在,所以创建这种类型的
对象时必须有outer class对象存在才行;而对于static inner class则无此限制。

\begin{javacode}
class Outer {
  private int out = 42;

  // non-static inner class
  public class InstanceInner {
    private int instanceInner = out;
  }

  // static inner class
  public static class StaticInner {
    private static int staticInner;
  }

  private void out() {
    // access 'private' field of inner class
    int i = new InstanceInner().instanceInner;
    int k = StaticInner.staticInner;
  }
}

class Test {
  void test() {
    Outer outer = new Outer();
    // non-static内部类依附于外部类而存在,所以实例化时必须存在外部类对象
    InstanceInner iInner = outer.new InstanceInner();
    // static内部类则可直接实例化,不需外部类对象存在
    StaticInner sInner = new Outer.StaticInner();
  }
}
\end{javacode}

\subsection[Inner class for JVM]{Inner class for JVM}
在JVM看来,inner class是如何表示的呢?

\emph{在JVM中,根本没有inner class这个概念!一切都是顶级类!}

只是内部类最终编译成class文件时,其名称是...\$...class这样的形式。

啊?那岂不是说:类的private成员可以被外部类(not outer class)访问?


\section[Double braces initialization]{Double braces initialization}
Java支持匿名内部类(Inner anonymous class),既然是匿名的,就无法在其他地方引用,
也是也就只能在创建类的同时创建对象(毕竟,类不用来创建对象还能干啥呢),可是
其他类都可以有构造函数,而且构造函数的名字就是类名,匿名内部类也可以有
构造函数,eh....,好吧,它的构造函数没有名字,匿名嘛!但是这是不对的,只是
可以这样理解,内层的大括号叫做instance initializer,它在匿名内部类构造时执行。

于是乎,有了下面的代码,

\begin{javacode}
// 初始化HashMap
HashMap<String, String> hm = new HashMap<String, String>() {
  // instance initializer(对比static initializer)
  {
    // 匿名内部类的this指向父类(即HashMap)对象
    this.put("k1", "v1");
    this.put("k2", "v2");
    this.put("k3", "v3");
  }
}

// 初始化ArrayList
ArrayList<String> al = new ArrayList<String>() {
  {
    add("A");
    add("A");
    add("A");
  }
}

// 任意类都可以,Base是任意自定义类
Base b = new Base() {
  {
    setName(Inner anonymous class, instance initializer.);
  }
}
\end{javacode}


\part[Java IO]{Java IO}
Java的IO库位于java.io.*包中。

\section[面向字节的IO接口]{面向字节的IO接口}
InputStream和OutputStream是面向字节的IO接口,它们处理的是字节流

\subsection[InputStream]{InputStream}
\begin{figure}
  \centering
  \includegraphics[width=.9\textwidth]{picturedir/inputstream.png}
  \caption{面向字节的InputStream接口}
  \label{fig:inputstream}
\end{figure}

图\ref{fig:inputstream}展示了InputStream的基本接口。
InputStream是一个抽象类,真正使用时都是使用其子类进行实例化。

\subsubsection[FileInputStream]{FileInputStream}
\subsubsection[ByteArrayInputStream]{ByteArrayInputStream}
\subsubsection[ObjectInputStream]{ObjectInputStream}
\subsubsection[PipedInputStream]{PipedInputStream}
\subsubsection[FilterInputStream]{FilterInputStream}

\subsection[OutputStream]{OuttputStream}
\begin{figure}
  \centering
  \includegraphics[width=.9\textwidth]{picturedir/outputstream.png}
  \caption{面向字节的OutputStream接口}
  \label{fig:outputstream}
\end{figure}

图\ref{fig:outputstream}展示了OutputStream的基本接口。
OutputStream是一个抽象类,真正使用时都是使用其子类进行实例化。
\subsubsection[FileOutputStream]{FileOutputStream}
\subsubsection[ByteArrayOutputStream]{ByteArrayOutputStream}
\subsubsection[ObjectOutputStream]{ObjectOutputStream}
\subsubsection[PipedOutputStream]{PipedOutputStream}
\subsubsection[FilterOutputStream]{FilterOutputStream}


\section[面向字符的IO接口]{面向字符的IO接口}
\subsection[面向字符的Input接口]{面向字符的Input接口}
\begin{figure}
  \centering
  \includegraphics[width=.9\textwidth]{picturedir/reader.png}
  \caption{面向字符的Reader接口}
  \label{fig:reader}
\end{figure}

图\ref{fig:reader}展示了Reader的基本接口。

\subsection[面向字符的Output接口]{面向字符的Output接口}
\begin{figure}
  \centering
  \includegraphics[width=.9\textwidth]{picturedir/writer.png}
  \caption{面向字符的Writer接口}
  \label{fig:writer}
\end{figure}

图\ref{fig:writer}展示了Writer的基本接口。



\part[Java NIO]{Java NIO}



\part[Multi-thread and Concurrency]{Multi-thread Concurrency}
\section[How to create Java thread]{How to create Java thread}

\section[package: concurrency]{package concurrency}

\section[keyword: join]{keyword: join}
join用于等待其他线程结束,如otherThread.join(),当前线程将会等待
otherThread结束后才会退出。

实例:

\begin{javacode}
public class JoinTest {
  public static void main(String[] args) {
    Thread t1 = new Thread(new MyRunnable(), "Thread-1");
    Thread t2 = new Thread(new MyRunnable(), "Thread-2");

    t1.start();
    t2.start();
    System.out.println("Main thread end.");
  }
}

class MyRunnable implements Runnable {
  String name = Thread.currentThread().getName();
  @Override
  public void run() {
    System.out.println("wtf: " + name);
    try {
      Thread.sleep(3000);
    } catch (InterruptedException e) {
      e.printStackTrace();
    }
    System.out.println(">>> " + Thread.currentThread().getName());
  }
}
\end{javacode}

注意:wtf那一行输出的是“main”,即name变量保有的是主线程的名字,WTF!

可能的输出,

\begin{bashcode}
Main thread end.
wtf: main
wtf: main
>>> Thread-1
>>> Thread-2
\end{bashcode}

\section[keyword: synchronized]{keyword: synchronized}
'synchronized'关键字用于锁住资源,从而保证多线程环境下,同一时刻只有一个
线程能拥有锁住的资源,资源用完之后释放锁。

有两种synchronized方式:
\begin{enumerate}
\item synchronized block: 只锁住该block里的资源
\item synchronized method: 锁住整个对象;如果是static方法,就锁住该Class
\end{enumerate}
另外还需注意:
\begin{itemize}
\item synchronized关键字“不能”用于构造函数和变量
\item synchronized影响效率,在真正必须的时候才使用
\item synchronized只在同一个JVM实例上有效
\item synchronized不要用于non-private的对象,以及有getter函数的private对象
\item synchronized不要用于常量池中的对象,如String对象
\end{itemize}
推荐使用方法:\\
\begin{javacode}
// dummy object variable for synchronization
private Object mutex = new Object();
int count = 0;
synchronized(mutex) {
    count++;
}
\end{javacode}

注意:synchronized需要程序员自觉才行,如果在某个线程程序员没有
使用synchronized而是直接就count++,此时synchronized就被人为的绕过
去了,尤其是用继承Thread类的方法创建线程时,这种情况更容易发生。

不要这样用:\\
\begin{javacode}
public class MyObject {
    // will lock on the object's monitor
    public synchronized void doSomething() { /* do something...*/ }

    MyObject myObject = new Myobject();
    synchronized(myObject) {
        while(true) {
            Thread.sleep(Integer.MAX_VALUE);
        }
    }
}
\end{javacode}

两个synchronized锁住的是同一个monitor,所以一旦第二个执行,
doSomething()将不可能再执行,导致死锁和DoS(Denial of Service)。

不要这样用:\\
\begin{javacode}
public class MyObject {
    public Object lock = new Object();
    public void doSomething() {
        synchronized(lock) {
            // do something...
        }
    }
}

MyObject myObject = new MyObject();
// 修改lock的引用,doSomething()函数有可能并发执行,
// 同理,有getter函数的private lock也是一样的
myObject.lock = new Object();
\end{javacode}



\section[keyword: wait, notify and notifyAll]{keyword: wait, notify and notifyAll}
wait, notify, notifyAll三个函数用于多线程竞争某个资源的情况,这三个函数在调用
之前都必须先取得“竞态资源”的monitor才行,所以这三个函数必须放到sychronized中
才行。

下面是一个生产者、消费者的实例。

测试用例:

\begin{javacode}
public class WaitNotifyTest {
  public static void main(String[] args) {
    Message msg = new Message("Init Message");
    new Thread(new Waiter(msg), "waiter1").start();
    new Thread(new Waiter(msg), "waiter2").start();

    new Thread(new Notifier(msg), "notifier").start();
    System.out.println("Main thread end.");
  }
}
\end{javacode}

竟态资源:

\begin{javacode}
class Message {
  private String msg;

  public Message(String str) {
    this.msg = str;
  }
  public String getMsg() {
    return msg;
  }
  public void setMsg(String str) {
    this.msg = str;
  }
}
\end{javacode}

消费者:

\begin{javacode}
class Waiter implements Runnable {
  private Message msg;

  public Waiter(Message m) {
    this.msg = m;
  }
  @Override
  public void run() {
    String name = Thread.currentThread().getName();
    synchronized (msg) {
      try {
        System.out.println("*" + name + "*" + " waiting msg at: "
                           + System.currentTimeMillis());
        msg.wait();
      } catch (InterruptedException e) {
        e.printStackTrace();
      }
      System.out.println("*" + name + "*" + " got msg at: "
                         + System.currentTimeMillis());
      // process the message now
      System.out.println("*" + name + "*" + " process msg: " + msg.getMsg());
    }
  }
}
\end{javacode}

生产者:

\begin{javacode}
class Notifier implements Runnable {
  private Message msg;

  public Notifier(Message msg) {
    this.msg = msg;
  }
  @Override
  public void run() {
    String name = Thread.currentThread().getName();
    System.out.println("*" + name + "*" + " started.");
    try {
      Thread.sleep(1000);
      synchronized (msg) {
        msg.setMsg("\"This is " + "*" + name + "*'s" + " msg.\"");
        //msg.notify();
        msg.notifyAll();
      }
    } catch (InterruptedException e) {
      e.printStackTrace();
    }
  }
}
\end{javacode}

注意,notifyAll会把所有等待的线程都唤醒,但是无法确定这些线程
的执行顺序,每次只能执行一个,而notify函数只能唤醒一个,随机的
一个。

执行顺序是这样的:

线程A获得msg.lock之后进入synchronized block,调用msg.wait()
阻塞A线程,然后释放msg.lock(并未退出wait()方法,在synchronized中,
没有lock就寸步难行)。线程B获得msg.lock之后进入synchronized block,
调用msg.notify()/msg.notifyAll(),通知线程A或者其他等待msg.lock
的线程,等待线程被唤醒,但是没有msg.lock仍然无法运行,之后线程B
继续执行synchronized block剩余的内容,执行完之后释放msg.lock,
正在等待的线程(如线程A)会取得msg.lock然后完成wait()方法,进而
完成synchronized block的剩余内容,之后释放msg.lock。如果线程B
调用的是msg.notifyAll()并且还有其他等待线程的话,其他等待线程中的
一个会获得msg.lock,然后执行完msg.wait()方法继而完成
synchronized block之后释放msg.lock,依次重复,直到所有等待线程
都完成这个过程,整个同步过程也就结束了。


\part[JVM]{Java Virtual Machine}
Java Virtural Machine(JVM)可能有三种解释:
\begin{enumerate}
  \item The abstract specification,即JVM规范
  \item A concrete implementation,即一个具体的JVM实现
  \item A runtime instance,即一个运行时JVM实例
\end{enumerate}
简单一点儿理解,Java Virtual Machine可能指的是"specification"、"implementation"、
"instance"。

\section[Lifetime of JVM instance]{Lifetime of JVM instance}
JVM instance的使命就是“运行Java application”,Java application启动时创建,结束时
销毁。Java application和JVM instance是一一对应的,即每个Java application都运行在
一个“单独”的JVM instance中,如果一个电脑上同时运行了3个Java application,那么将会有
3个JVM instance同时存在。


\end{document}
