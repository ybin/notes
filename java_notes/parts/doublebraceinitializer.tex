\section[Double braces initialization]{Double braces initialization}
要讲明白双大括号初始化就要先了解什么是instance initializer和static initializer,
该部分Java语言规范在\href{http://docs.oracle.com/javase/specs/jls/se7/html/jls-8.html#jls-8.6}{JLS 8.6}。

\begin{javacode}
class Initializer {

  // executed when an instance of the class is created
  static {
    System.out.println("static initializer");
  }

  // executed when the class is initialized
  {
    System.out.println("instance initializer");
  }

  public Initializer() {
    System.out.println("constructor");
  }

  void initialize() {
    System.out.println("instance method");
  }
}
\end{javacode}

执行顺序:static initializer, instance initializer, constructor。

Java支持匿名内部类(Inner anonymous class),既然是匿名的,就无法在其他地方引用,
因此也就只能在创建类的同时创建对象(毕竟,类不用来创建对象还能干啥呢),内层的
大括号即为instance initializer。

于是乎,有了下面的代码,

\begin{javacode}
package com.example.javademo;
Class DoubleBraceTest {
  void test() {
    // 初始化HashMap
    HashMap<String, String> hm = new HashMap<String, String>() {
      // instance initializer(对比static initializer)
      {
        // 匿名内部类的this指向父类(即HashMap)对象
        this.put("k1", "v1");
        this.put("k2", "v2");
        this.put("k3", "v3");
      }
    }

    // 任意类都可以,Base是任意自定义类
    Base b = new Base() {
      {
        setName(Inner anonymous class, instance initializer.);
      }
    }

    // result: super class: java.util.HashMap
    System.out.println("super class: " + hm.getClass().getSuperclass().getName());
    // result: class name: com.example.javademo.DoubleBraceTest$2
    System.out.println("class name: " + hm.getClass().getName());
  }
}
\end{javacode}

注意:double braces initialization是不鼓励使用的,因为它本质上是定义了一个
匿名内部类,因此实例化的instance会一直保存对外部类(DoubleBraceTest类)实例
的引用,并且这也会让类加载器管理更多的类。