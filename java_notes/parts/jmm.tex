\section[Java Memory Model]{Java Memory Model}
Java Memory Model(JMM),即Java内存模型,是存在于Java语言层面的内容,
并不涉及到JVM,JMM是用于Java线程间通讯的一套规则,目的是防止线程间
出现数据竞争。所谓的数据竞争指的是:如果A操作、B操作同时需要某些数据,
并且A、B之间存在影响(单向或者相互影响),但是并没有任何手段可以
保证它们之间的执行顺序,那么就说A、B之间存在数据竞争。如A读取某个值
而B写入某个值,但是没有任何手段可以保证它们之间的顺序,那么它们存在
数据竞争。

为此JSR-133制定了一套规则叫做happens-before规则,以此避免数据竞争。
但是需要注意的是happens-before并不仅仅是指的操作顺序,它更主要的是
明确“可见性”,如其中有一条规则如下:

\emph{对同一个monitor的unlock必须发生在其他线程lock之前}

这个很好理解,先释放锁再获取锁,但是这里不仅仅指的是操作顺序,它还要
保证,unlock之前的monitor监视的操作结果(如synchronize block里的操作)
必须能够被lock之后的线程看到,也就是说
如果Action A happens-before Action B,那么A的操作结果必须对B是
可见的,也就是说A的操作结果要写入主内存,而B在操作之前必须先从主内存
中同步数据到它的工作内存(e.g. CPU Cache)。