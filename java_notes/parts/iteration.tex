\section[Enumeration vs Iterator]{Enumeration vs Iterator}
Enumeration和Iterator都是用于遍历集合,Enumeration主要用于遍历
Vector、HashTable,而Iterator用于遍历其他集合。两者的主要区别是:
Enumeration不可以在遍历过程中删除元素,而Iterator则可以,
比较Enumeration和Iterator的定义就可以明确这一点。

下面是Enumeration的定义,

\begin{javacode}
public interface Enumeration<E> {
  boolean hasMoreElements();
  E nextElement();
}
\end{javacode}

下面是Iterator的定义,

\begin{javacode}
public interface Iterator<E> {
  boolean hasNext();
  E next();
  void remove();
}
\end{javacode}

另附使用实例(Vector既可以Enumeration遍历也可用Iterator遍历),

\begin{javacode}
Vector<String> v = new Vector<String>();
v.add("A");
v.add("B");
v.add("C");

// 使用Enumeration遍历
// 'for' version
String elem;
for(Enumeration<String> e = v.elements(); e.hasMoreElements(); ) {
  elem = e.nextElement();
  System.out.println(elem);
}
// 'while' version
Enumeration<String> e = v.elements();
while(e.hasMoreElements()) {
  elem = e.nextElement();
  System.out.println(elem);
}

// 使用Iterator遍历
// 'for' version
String cur;
for(Iterator<String> iter = v.iterator(); iter.hasNext(); ) {
  cur = iter.next();
  System.out.println(cur);
  if("B".equals(cur)) {
    iter.remove();
  }
}
// 'while' version
Iterator<String> iter = v.iterator();
while(iter.hasNext()) {
  cur = iter.next();
  System.out.println(cur);
  if("B".equals(cur)) {
    iter.remove();
  }
}
\end{javacode}

\section[遍历Map]{遍历Map}
遍历Map时需要从collection的视角去看Map。

\begin{javacode}
public void operateMap() {
  HashMap<String, String> info = new HashMap<String, String>() {
    private static final long serialVersionUID = 1L;
    {
      put("John", "42");
      put("Eric", "43");
      put("Waston", "44");
    }
  };

  /*
   * Collection<E> interface extends Iterable<T> interface,
   * so all Collection including Set<E>, List<E>, Queue<E>
   * can get an Iterator via iterator() method.
   */

  // 遍历Map - 1
  Iterator<Entry<String, String>> entryIter = info.entrySet().iterator();
  while(entryIter.hasNext()) {
    Entry<String, String> entry = entryIter.next();
    System.out.println("key: " + entry.getKey() + ", value: " + entry.getValue());
  }

  // 遍历Map - 2
  Iterator<String> keyIter = info.keySet().iterator();
  while(keyIter.hasNext()) {
    String key = keyIter.next();
    String value = info.get(key);
    System.out.println("key: " + key + ", value: " + value);
  }

  // 遍历Map - 3,遍历Value
  Iterator<String> valueIter = info.values().iterator();
  while(valueIter.hasNext()) {
    String value = valueIter.next();
    System.out.println("value: " + value);
  }
}
\end{javacode}

\section[Print system properties]{Print system properties}
\begin{javacode}
public void printSystemProperties() {
  Properties p = System.getProperties();
  Enumeration<Object> e = p.keys();
  String key;

  for( ; e.hasMoreElements(); ) {
    key = (String) e.nextElement();
    System.out.println(key + " = " + p.getProperty(key));
  }
}
\end{javacode}
