\section[Polymorphism - 多态]{Polymorphism - 多态}
在Java中(其他OOP中也是如此),Polymorphism(多态)是一个概念,它用于method的定义,
有两种表现形式:override和overload。两种多态形式各有不同的限定条件。

\subsection[override]{override}
父类方法和子类方法构成override必须满足如下条件:

\begin{enumerate}
  \item 子类方法和父类方法的name、parameters、return type必须完全一样
  \item 子类方法不能缩小(reduce)父类方法的访问权限
  \item 子类方法不能扩大或者改变父类方法的异常范围,详见第\ref{sec:exceptions-in-override}节
\end{enumerate}

注意:final修饰的方法不能被override。

\subsection[overload]{overload}
位于同一个类中或者分别位于父类和子类中的两个方法,构成overload必须满足如下条件:

\begin{enumerate}
  \item parameters(包括参数类型、个数、顺序)不能完全相同
\end{enumerate}

注意,如果两个方法参数完全相同而只是返回值不同,则它们不构成overload关系。

\section[静态方法能否被override]{静态方法能否被override}
静态方法(类方法)能否被覆盖(override)?

这个问题本身就是错误的!

override被设计用于instance methods,而不是用于static methods。
类加载之后,static method放置在Method Area,因此每个类,不论子类父类,都有自己
单独的static field和static method,相互之间没有任何关系。

\begin{javacode}
class Base {
  public static void staticPrint() {
    System.out.println("Base static method");
  }
  public void instancePrint() {
    System.out.println("Base instance method");
  }
}

class Derived extends Base {
  public static void staticPrint() {
    System.out.println("Derived static method");
  }
  @Override
  public void instancePrint() {
    System.out.println("Derived instance method");
  }
}

Base base = new Derived();
// result: Base static method
base.staticPrint();
// result: Derived nstance method
base.instancePrint();
// result: Derived static method
((Derived) base).staticPrint();
\end{javacode}

需要注意的是:static method是静态绑定的,即在编译期就绑定完毕,
因此base.staticPrint()会被绑定到Base类的static method;
而instance method是动态绑定的,即在运行期才会寻找instance数据
(位于Heap中,base的instance数据根据Derived类实例化),
查找method的bytecode然后绑定。