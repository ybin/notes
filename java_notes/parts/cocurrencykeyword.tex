\section[keyword: join]{keyword: join}
join用于等待其他线程结束,如otherThread.join(),当前线程将会等待
otherThread结束后才会退出。

实例:

\begin{javacode}
public class JoinTest {
  public static void main(String[] args) {
    Thread t1 = new Thread(new MyRunnable(), "Thread-1");
    Thread t2 = new Thread(new MyRunnable(), "Thread-2");

    t1.start();
    t2.start();
    System.out.println("Main thread end.");
  }
}

class MyRunnable implements Runnable {
  String name = Thread.currentThread().getName();
  @Override
  public void run() {
    System.out.println("wtf: " + name);
    try {
      Thread.sleep(3000);
    } catch (InterruptedException e) {
      e.printStackTrace();
    }
    System.out.println(">>> " + Thread.currentThread().getName());
  }
}
\end{javacode}

注意:wtf那一行输出的是“main”,即name变量保有的是主线程的名字,WTF!

可能的输出,

\begin{bashcode}
Main thread end.
wtf: main
wtf: main
>>> Thread-1
>>> Thread-2
\end{bashcode}

\section[keyword: synchronized]{keyword: synchronized}
'synchronized'关键字用于锁住资源,从而保证多线程环境下,同一时刻只有一个
线程能拥有锁住的资源,资源用完之后释放锁。

有两种synchronized方式:
\begin{enumerate}
\item synchronized block: 只锁住该block里的资源
\item synchronized method: 锁住整个对象;如果是static方法,就锁住该Class
\end{enumerate}
另外还需注意:
\begin{itemize}
\item synchronized关键字“不能”用于构造函数和变量
\item synchronized影响效率,在真正必须的时候才使用
\item synchronized只在同一个JVM实例上有效
\item synchronized不要用于non-private的对象,以及有getter函数的private对象
\item synchronized不要用于常量池中的对象,如String对象
\end{itemize}
推荐使用方法:\\
\begin{javacode}
// dummy object variable for synchronization
private Object mutex = new Object();
int count = 0;
synchronized(mutex) {
    count++;
}
\end{javacode}

注意:synchronized需要程序员自觉才行,如果在某个线程程序员没有
使用synchronized而是直接就count++,此时synchronized就被人为的绕过
去了,尤其是用继承Thread类的方法创建线程时,这种情况更容易发生。

不要这样用:\\
\begin{javacode}
public class MyObject {
    // will lock on the object's monitor
    public synchronized void doSomething() { /* do something...*/ }

    MyObject myObject = new Myobject();
    synchronized(myObject) {
        while(true) {
            Thread.sleep(Integer.MAX_VALUE);
        }
    }
}
\end{javacode}

两个synchronized锁住的是同一个monitor,所以一旦第二个执行,
doSomething()将不可能再执行,导致死锁和DoS(Denial of Service)。

不要这样用:\\
\begin{javacode}
public class MyObject {
    public Object lock = new Object();
    public void doSomething() {
        synchronized(lock) {
            // do something...
        }
    }
}

MyObject myObject = new MyObject();
// 修改lock的引用,doSomething()函数有可能并发执行,
// 同理,有getter函数的private lock也是一样的
myObject.lock = new Object();
\end{javacode}



\section[keyword: wait, notify and notifyAll]{keyword: wait, notify and notifyAll}
wait, notify, notifyAll三个函数用于多线程竞争某个资源的情况,这三个函数在调用
之前都必须先取得“竞态资源”的monitor才行,所以这三个函数必须放到sychronized中
才行。

下面是一个生产者、消费者的实例。

测试用例:

\begin{javacode}
public class WaitNotifyTest {
  public static void main(String[] args) {
    Message msg = new Message("Init Message");
    new Thread(new Waiter(msg), "waiter1").start();
    new Thread(new Waiter(msg), "waiter2").start();

    new Thread(new Notifier(msg), "notifier").start();
    System.out.println("Main thread end.");
  }
}
\end{javacode}

竟态资源:

\begin{javacode}
class Message {
  private String msg;

  public Message(String str) {
    this.msg = str;
  }
  public String getMsg() {
    return msg;
  }
  public void setMsg(String str) {
    this.msg = str;
  }
}
\end{javacode}

消费者:

\begin{javacode}
class Waiter implements Runnable {
  private Message msg;

  public Waiter(Message m) {
    this.msg = m;
  }
  @Override
  public void run() {
    String name = Thread.currentThread().getName();
    synchronized (msg) {
      try {
        System.out.println("*" + name + "*" + " waiting msg at: "
                           + System.currentTimeMillis());
        msg.wait();
      } catch (InterruptedException e) {
        e.printStackTrace();
      }
      System.out.println("*" + name + "*" + " got msg at: "
                         + System.currentTimeMillis());
      // process the message now
      System.out.println("*" + name + "*" + " process msg: " + msg.getMsg());
    }
  }
}
\end{javacode}

生产者:

\begin{javacode}
class Notifier implements Runnable {
  private Message msg;

  public Notifier(Message msg) {
    this.msg = msg;
  }
  @Override
  public void run() {
    String name = Thread.currentThread().getName();
    System.out.println("*" + name + "*" + " started.");
    try {
      Thread.sleep(1000);
      synchronized (msg) {
        msg.setMsg("\"This is " + "*" + name + "*'s" + " msg.\"");
        //msg.notify();
        msg.notifyAll();
      }
    } catch (InterruptedException e) {
      e.printStackTrace();
    }
  }
}
\end{javacode}

注意,notifyAll会把所有等待的线程都唤醒,但是无法确定这些线程
的执行顺序,每次只能执行一个,而notify函数只能唤醒一个,随机的
一个。

执行顺序是这样的:

线程A获得msg.lock之后进入synchronized block,调用msg.wait()
阻塞A线程,然后释放msg.lock(并未退出wait()方法,在synchronized中,
没有lock就寸步难行)。线程B获得msg.lock之后进入synchronized block,
调用msg.notify()/msg.notifyAll(),通知线程A或者其他等待msg.lock
的线程,等待线程被唤醒,但是没有msg.lock仍然无法运行,之后线程B
继续执行synchronized block剩余的内容,执行完之后释放msg.lock,
正在等待的线程(如线程A)会取得msg.lock然后完成wait()方法,进而
完成synchronized block的剩余内容,之后释放msg.lock。如果线程B
调用的是msg.notifyAll()并且还有其他等待线程的话,其他等待线程中的
一个会获得msg.lock,然后执行完msg.wait()方法继而完成
synchronized block之后释放msg.lock,依次重复,直到所有等待线程
都完成这个过程,整个同步过程也就结束了。