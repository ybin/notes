\section[Inner Class]{Inner Class}
Java中,Inner class的field/method相当于Outer class的field/method,
只不过要通过Inner class或Inner class的对象对其进行访问而已。

\subsection[内部类的可见性]{内部类的可见性}
内部类的field/method跟外部类的field/method本质上是一样的,只是访问
方式不同而已,所以内部类和外部类相互之间是“完全透明”的。

\subsection[Create object of Inner class ]{Create object of Inner class}
由于non-static inner class依附于outer class而存在,所以创建这种类型的
对象时必须有outer class对象存在才行;而对于static inner class则无此限制。

\begin{javacode}
class Outer {
  private int out = 42;

  // non-static inner class
  public class InstanceInner {
    private int instanceInner = out;
  }

  // static inner class
  public static class StaticInner {
    private static int staticInner;
  }

  private void out() {
    // access 'private' field of inner class
    int i = new InstanceInner().instanceInner;
    int k = StaticInner.staticInner;
  }
}

class Test {
  void test() {
    Outer outer = new Outer();
    // non-static内部类依附于外部类而存在,所以实例化时必须存在外部类对象
    InstanceInner iInner = outer.new InstanceInner();
    // static内部类则可直接实例化,不需外部类对象存在
    StaticInner sInner = new Outer.StaticInner();
  }
}
\end{javacode}

\subsection[Inner class for JVM]{Inner class for JVM}
在JVM看来,inner class是如何表示的呢?

\emph{在JVM中,根本没有inner class这个概念!一切都是顶级类!}

只是内部类最终编译成class文件时,其名称是...\$...class这样的形式。

啊?那岂不是说:类的private成员可以被外部类(not outer class)访问?

是的,只不过这里的情形比较特殊,看下面的代码,

\begin{javacode}
class OuterClass {
  private int out = new InnerClass().in;
  class InnerClass {
    private int in = 0;
  }
}
\end{javacode}

分别反编译OuterClass.class和OuterClass\$InnerClass.class,
在OuterClass.class中可以看到这样的代码:

\begin{javacode}
13: invokestatic #17; //Method com/example/javademo/OuterClass$InnerClass
                  .access$0:(Lcom/example/javademo/OuterClass$InnerClass;)I
16: putfield #21; //Field out:I
\end{javacode}

这便是new InnerClass().in编译之后的代码。再看OuterClass\$InnerClass.class
的反编译结果:

\begin{javacode}
static int access$0(com.example.javademo.OuterClass$InnerClass);
  Code:
    Stack=1, Locals=1, Args_size=1
    0: aload_0
    1: getfield #17; //Field in:I
    4: ireturn
\end{javacode}

这个静态函数其实只是把属性"in"的值返回而已,这个静态函数叫做sythetic函数,
也叫做bridge函数。进一步查看.class的二进制数据(用十六进制编辑器),可以看到
该函数的access\_flag设置为10 08,即ACC\_SYNTHETIC | ACC\_STATIC。