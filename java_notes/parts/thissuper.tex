\section[关于this和super]{关于this和super}
this和super两个关键字值得研究,他们有本质的区别。

\emph{this、super是对象的两个属性},这种说法是及其错误的!\\
\emph{父类对象}这一说法也是及其错误的!没有这样的对象!

\subsection[this]{this}
this并非instance的field,this其实压根儿就不在heap中,this(的值)存在于thread stack中的
frame中的Local Variable的index 0处,它是JVM自动生成的。

\subsection[super]{super}
super就更让人大跌眼镜了,super不仅不是instance的field,它其实压根儿“不存在”,super
在compile时被翻译为this(它们的值相等),当然如果子类中field、method有隐藏父类field、method的
情况时,compiler也在此时一并处理了。这不仅让我想起来Clojure里的Reader Macro...

\subsection[没有父类对象]{没有父类对象}
加载子类时,父类当然是要一并加载的,但是创建子类对象时并不会创建父类对象,而是将
父类对象的fields一并存储到子类对象中(父类的method存在于Method Area中),并且这些
属性排列还有一定规则:先排放父类属性,再排放子类属性!