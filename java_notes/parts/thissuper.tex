\section[关于this和super]{关于this和super}
this和super两个关键字值得研究,他们有本质的区别。

\emph{this、super是对象的两个属性},这种说法是及其错误的!\\
\emph{父类对象}这一说法也是及其错误的!没有这样的对象!

\subsection[this]{this}
this并非instance的field,this其实压根儿就不在heap中,this(的值)存在于thread stack中的
frame中的Local Variable的index 0处,它是JVM自动生成的。

\subsection[super]{super}
super就更让人大跌眼镜了,super不仅不是instance的field,它其实压根儿“不存在”,super
在compile时被翻译为this(它们的值相等),当然如果子类中field、method有隐藏父类field、method的
情况时,compiler也在此时一并处理了。这不仅让我想起来Clojure里的Reader Macro...

\subsection[没有父类对象]{没有父类对象}
加载子类时,父类当然是要一并加载的,但是创建子类对象时并不会创建父类对象,而是将
父类对象的fields一并存储到子类对象中(父类的method存在于Method Area中),并且这些
属性排列还有一定规则:先排放父类属性,再排放子类属性!

\begin{javacode}
class Base {
  public void foo() throws IOException { }
  public void bar() throws IOException { }
  public void base1() throws IOException { }
  public void base2() throws ArithmeticException { }
  public void base3() throws IOException { }
  public void base4() throws ArithmeticException { }
  public void base5() { }
}

class Derived extends Base {
  @Override
  public void foo() throws FileNotFoundException { } // allowed
  @Override
  public void bar() throws SQLException { } // NOT allowed

  @Override
  public void base1() { } // allowed, do not throw any exception
  @Override
  public void base2() throws IndexOutOfBoundsException { } // allowed
  @Override
  public void base3() throws IndexOutOfBoundsException { } // allowed
  @Override
  public void base4() throws IOException { } // NOT allowed
  @Override
  public void base5() throws IOException { } // NOT allowed
}
\end{javacode}

IOException、FileNotFoundException、SQLException都是checked异常,并且
FileNotFound\-Exception是IOException的子类,ArithmeticException和
IndexOutOfBoundsException都是runtime异常。注意,base4(), base5()两种
情况是不允许的!

\emph{为何要作此限制呢?}

考虑如下情形,

\begin{javacode}
Base base = new Derived();
try {
  base.bar();
} catch(??? e) {
  // 这里应该捕获何种异常类型呢?
}
try {
  base.foo();
} catch(IOException e) {
  // 这里应该捕获的异常类型就很明确了!
}
\end{javacode}