\section[Java命令行使用方法]{Java命令行使用方法}
大家几乎都被IDE绑架了,谁还关注命令行java呢?\par
注意:java的参数都是使用“-”的,不论长格式还是短格式,这跟GNU格式不同。

\subsection[编译命令:javac]{编译命令:javac}
\cmd{javac -classpath </path/to/folder;/path/to/.jar;/path/to/.zip>\\
      \hspace*{5em}-sourcepath </path/to/folder;/path/to/.jar;/path/to/.zip>}
-classpath的值可以是folder、.jar、.zip,其中包含的都是.class文件;\\
-sourcepath的值可以是folder、.jar、.zip,其中包含的都是.java文件。

\subsection[运行命令:java]{运行命令:java}
\cmd{java -classpath </path/to/folder;/path/to/.jar;/path/to/.zip> runnable}
-classpath的值可以是folder、.jar、.zip,其中包含的都是.class文件。\\
runnable是一个.class文件,扩展名.class不要加上。

注意:-classpath或者-cp时,必须加上“当前目录”,否则找不到runnable,这个太蛋疼了吧!

运行Eclipse项目的某一个class文件时,在bin(output dir)目录下执行,
\cmd{java com.example.DemoClass}

常用命令参数,

\begin{itemize}
  \item -client 启用client VM
  \item -server 启用server VM
  \item -verbose:[class|gc|jni] 启用详细输出,使用class可以查看加载的所有class文件
  \item -Xint,-Xcomp,-Xmixed 启用interpreted mode、compiled mode、mixed mode,默认为mixed mode
  可以使用java -X<mode> -version进行查看
    \begin{itemize}
      \item interpreted mode: 强制JVM执行所有的字节码
      \item mixed mode: 使用JIT,部分解释执行部分编译成本地代码
      \item compiled mode: 把所有的字节码转换为本地代码
    \end{itemize}
  \item -enableassertions[:<packagename>...|:<classname>]
  \item -disenableassertions[:<packagename>...|:<classname>]\\
  启用、关闭指定粒度的断言
  \item  -esa | -enablesystemassertions
  \item -dsa | -disablesystemassertions\\
  启用、关闭系统断言
\end{itemize}

\subsection[反汇编命令:javap]{反汇编命令:javap}
javap是java的反汇编命令,但是很少有人用它来进行反汇编,
原因是有很多其他工具反汇编工作做的比javap好的多,但是
用javap来输出.class文件的字节码信息,对于学习JVM来说是
很有帮助的。
\cmd{javap -verbose runnable}
注意:runnable为.class文件,后缀.class不要加上。