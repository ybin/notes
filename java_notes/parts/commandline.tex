\section[Java命令行使用方法]{Java命令行使用方法}
大家几乎都被IDE绑架了,谁还关注命令行java呢?\par
注意:java的参数都是使用“-”的,不论长格式还是短格式,这跟GNU格式不同。

下面的参数说明,
\begin{itemize}
  \item runnable: 指的是“全限定名”格式的class,不是.class文件。
  \item -classpath: 其值可以是包含.class文件的folder、.jar文件、.zip文件,默认包含当前路径。
\end{itemize}


\subsection[编译命令:javac]{编译命令:javac}
\cmd{javac -classpath </path/to/folder;/path/to/.jar;/path/to/.zip>\\
      \hspace*{5em}-sourcepath </path/to/folder;/path/to/.jar;/path/to/.zip>}

\subsection[运行命令:java]{运行命令:java}
\cmd{java -classpath </path/to/folder;/path/to/.jar;/path/to/.zip> runnable}

实例:\\
运行Eclipse项目的某一个class文件时,在bin(output dir)目录下执行,
\cmd{java com.example.DemoClass}

全限定类名称本身就携带了其对应.class文件的“相对路径”\\(如com/example/DemoClass.class),
所以java.exe搜索classpath时“不会”递归搜索,而是把classpath里的路径跟.class
文件的相对路径拼凑成完整的路径,然后再看这个完整路径对应的文件是否存在。

常用命令参数,

\begin{itemize}
  \item -client 启用client VM
  \item -server 启用server VM
  \item -verbose:[class|gc|jni] 启用详细输出,使用class可以查看加载的所有class文件
  \item -Xint,-Xcomp,-Xmixed 启用interpreted mode、compiled mode、mixed mode,默认为mixed mode
  可以使用java -X<mode> -version进行查看
    \begin{itemize}
      \item interpreted mode: 强制JVM执行所有的字节码
      \item mixed mode: 使用JIT,部分解释执行部分编译成本地代码
      \item compiled mode: 把所有的字节码转换为本地代码
    \end{itemize}
  \item -enableassertions[:<packagename>...|:<classname>]
  \item -disenableassertions[:<packagename>...|:<classname>]\\
  启用、关闭指定粒度的断言
  \item  -esa | -enablesystemassertions
  \item -dsa | -disablesystemassertions\\
  启用、关闭系统断言
\end{itemize}

\subsection[反汇编命令:javap]{反汇编命令:javap}
javap是java的反汇编命令,但是很少有人用它来进行反汇编,
原因是有很多其他工具反汇编工作做的比javap好的多,但是
用javap来输出.class文件的字节码信息,对于学习JVM来说是
很有帮助的。
\cmd{javap -verbose runnable}