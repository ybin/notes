\section[class loader]{class loader}
Java的class loader分为两种:
\begin{enumerate}
  \item built-in class loader: 内置的类加载器只有一个,叫做bootstrap class loader
    \begin{itemize}
      \item Bootstrap: 加载[JRE]\bs lib\bs rt.jar和-Xbootclasspath指定的目录
    \end{itemize}
  \item user-defined class loader: JDK中预先提供两个用户自定义类加载器:
    \begin{itemize}
      \item ExtClassLoader: 加载[JRE]\bs lib\bs ext\*.jar和-Djava.ext.dirs指定的目录
      \item AppClassLoader: CLASSPATH环境变量和-Djava.class.path指定的目录
    \end{itemize}
\end{enumerate}
Bootstrap内置于JVM实现,由C++部分实现,所以没有Java Class对象与之对应(null与之对应),
用户自定义的类加载器都有Java层Class对象与之对应。Java提供ClassLoader类,所有用户自定义的
加载器都继承自该类,如图\ref{fig:classloadertree}所示。
% class inheret tree
\begin{figure}
  \centering
  \begin{tikzpicture}[<-,>=stealth',level/.style={sibling distance = 4cm,level distance=1.5cm}]
  \node {ClassLoader}
  child {
    node {SecureClassLoader}
    child {
      node {URLClassLoader}
      child {
        node {ExtClassLoader}
      }
      child {
        node {AppClassLoader}
      }
    }
  };
  \end{tikzpicture}
  \caption{JDK中自定义类的(OOP)继承体系}\label{fig:classloadertree}
\end{figure}

\subsection[DIY class loader]{DIY class loader}
自定义class loader只需继承ClassLoader类即可,该类中设计到自定义类加载器的有
如下这些方法:\\
\begin{javacode}
public Class<?> loadClass(String name)
  throws ClassNotFoundException { }
protected Class<?> findClass(String name)
  throws ClassNotFoundException { }
protected final Class<?> defineClass(String name, byte[] b, int off, int len,
  ProtectionDomain protectionDomain) throws ClassFormatError { }

protected synchronized Class<?> loadClass(String name, boolen resolve)
  throws ClassNotFoundException { }
public final ClassLoader getParent(){ }
protected final Class<?> findLoadedClass(String name) { }
protected final void resolveClass(Class<?> c) { }
\end{javacode}

其中,loadClass()会实现代理模式,详见第\ref{sec:delegationmodel}节,
所以不要轻易override这个方法,自定义class loader时只需override findClass(),
标准的做法是这样的:\\
\begin{javacode}
class MyClassLoader extends ClassLoader {

  public Class findClass(String name) {
    byte[] b = loadClassData(name);
    return defineClass(name, b, 0, b.length);
  }

  private byte[] loadClassData(String name) {
    // load the class data from file system, network etc.
  }
}
\end{javacode}

下面是一个完整的例子,

\begin{javacode}
class MyClassLoader extends ClassLoader {
  private static String prefix = "path/to/project/";

  public MyClassLoader() { // 使用AppClassLoader作为其parent
    super();
  }
  public MyClassLoader(ClassLoader parent) {
    super(parent); // if parrent == null then use bootstrap class loader as parent
  }

  @Override
  public Class findClass(String name) throws ClassNotFoundException {
    System.out.println("can not load class, not try finding: " + name);
    byte[] data = null;
    try {
      data = getClassData(name);
    } catch (IOException e) {
      e.printStackTrace();
    }
    return defineClass(name, data, 0, data.length);
  }

  private byte[] getClassData(String name) throws IOException {
    File f = new File(prefix + convertPath(name) + ".class");
    FileInputStream input = new FileInputStream(f);
    byte[] data = new byte[(int) f.length()];
    input.read(data);
    input.close();
    return data;
  }

  private String convertPath(String dotPath) {
    StringBuilder builder = new StringBuilder();
    for(String s : dotPath.split("\\.")) {
      builder.append(s + "/");
    }
    return builder.subSequence(0, builder.length()-1).toString();
  }
}
// test
void testClassLoader() {
  MyClassLoader loader1 = new MyClassLoader(); // use AppClassLoader as parent
  Class c1 = loader1.loadClass("com.example.javademo.ThisSuper");
  ThisSuper o1 = (ThisSuper) c1.newInstance();

  MyClassLoader loader2 = new MyClassLoader(null); // use Bootstrap as parent
  Class c2 = loader2.loadClass("com.example.javademo.ThisSuper");

  System.out.println(c1 == c2); // false
}
\end{javacode}

\subsection[parent delegation model]{parent delegation model}
\label{sec:delegationmodel}
Java管理这些类加载器时采用delegation model(代理模式),每个loader都有一个parent属性
(定义在ClassLoader类中),load class时首先使用parent进行加载,如果无法加载再自行加载,

\begin{javacode}
// implementation of loadClass() in ClassLoader
...
// First, check if the class has already been loaded
Class c = findLoadedClass(name);
if (c == null) {
  try {
    // Second, try to delegate class-loading to 'parent'
    if (parent != null) {
      c = parent.loadClass(name, false);
    } else {
      c = findBootstrapClass0(name);
    }
  } catch (ClassNotFoundException e) {
    // finally, find class using findClass()
    c = findClass(name);
  }
}
...
\end{javacode}

注意,这是个递归调用,最终的结果就是从child到parent进行扫描,如果没有找到class,
就再从parent到child依次调用findClass(),这是一个从child到parent然后再返回到child
的过程,中间任何一步找到class就结束整个流程。

为何要搞一个代理模式处理呢?为了安全!

代理模式可以保证优先从Java Core API中加载类,如java.lang.String类,走到Bootstrap
的时候已经找到了,所以即使用户自己重新定义String类,也不会被用到,这就保证了
核心库不被恶意替换掉。

但是,如果重写loadClass()函数的话就可能破坏代理模式,如Tomcat中的class loader
加载类的顺序正好跟这里的代理模式相反,不过在defineClass()的时候会有package检查,
如java.*这样的package是无法通过defineClass()重新定义的,而defineClass()是final的,
这就保证了核心库的安全性,除非用户不使用defineClass()而自己搞一套类定义机制出来
(可能吗?)。

\subsection[namespace]{namespace}
任何一个class都是由某一个class loader加载到JVM中的,在JVM中,一个类的唯一性并不
仅仅是靠package来保证,而是package和class loader一同保证,即不同class loader
完全可能加载同一个class到JVM,形成两个不同的class对象,这种机制就叫做namespace。

那么namespace是如何构成的呢?一个class loader构成一个namespace吗?No,这会造成
重复加载,想象一下一个class loader中就有一个String类,太浪费了!

答案是利用delegation model构成namespace,在HotSpot中,没加载一个类JVM内部都会
把它缓存起来,JVM内部维护一个巨大的哈希表:

$$hash\;table: class\;name+loader\;name\;\overset{hash}\Longrightarrow\;class\;object$$

加上代理模式,每个child loader都可以看到parent加载的class object,于是一个
class loader所能加载的class连同其parent所能加载的class就构成了一个namespace,
不同namespace之间无法相互访问,就像无法访问object(in heap)的private field一样。
