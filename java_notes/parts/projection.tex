\section[Reflection]{Reflection}

\subsection[明确定义的 vs 继承而来的]{明确定义的 vs 继承而来的}
类中明确定义的field, method, class, interface,用getDeclaredXXX()
来获取,这个方法不会涉及到继承而来的东西,如要获取继承到的field,
method等,可以使用getXXX(),不过这些东西只能取到public的,如父类
中定义了,

\begin{javacode}
public void print0() { }
private void print1() { }
\end{javacode}

则通过getXXX()只能获取到print0。当然这些函数都是Class对象的方法。
下面是一个示例:“如何让2+2=5”,

\begin{javacode}
// 详细内容请参阅Integer类的实现,java把-128到128范围内的整数都缓存了。

public static void testIntegerCache() throws Exception {
  // 取到Integer的内部类,IntegerCache
  Class integerCache = Integer.class.getDeclaredClasses()[0];
  // 该内部类有个field,cache,它的值是一个Integer[]
  Field cache = integerCache.getDeclaredField("cache");
  cache.setAccessible(true);
  // 取到该数组,并修改"4"所在位置上的值为"5"
  Integer[] arr = (Integer[])cache.get(integerCache);
  arr[132] = arr[133];
  System.out.printf("2 + 2 = %d\n", 2 + 2);
}
\end{javacode}