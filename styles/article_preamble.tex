
%%% fontenc
%\usepackage{fontspec,xunicode,xltxtra}
%\setmainfont{Times New Roman}
%\setsansfont{Source Sans Pro}
%\setmonofont{Source Sans Pro}

%%% xeCJK
%\usepackage{xeCJK}
%\defaultfontfeatures{Path = ../fonts/, Mapping=tex-text}
%\setCJKmainfont[BoldFont={Adobe Heiti Std},ItalicFont={Adobe Kaiti Std}]{Adobe Song Std}
%\setCJKsansfont[BoldFont={Adobe Heiti Std},ItalicFont={Adobe Kaiti Std}]{Adobe Song Std}
%\setCJKmonofont+[BoldFont={Adobe Heiti Std},ItalicFont={Adobe Kaiti Std}]{Adobe Song Std}
%\XeTeXlinebreaklocale "zh"
%\XeTeXlinebreakskip=0pt plus 1pt minus 0.1pt

\usepackage{fontspec}
\usepackage{xeCJK}

\defaultfontfeatures{Path = ../fonts/, Mapping=tex-text}
\setCJKmainfont[
    BoldFont=AdobeHeitiStd-Regular.otf,
    ItalicFont=AdobeKaitiStd-Regular.otf,
    SmallCapsFont=AdobeHeitiStd-Regular.otf
]{AdobeSongStd-Light.otf}
\setCJKsansfont{AdobeHeitiStd-Regular.otf}
\setCJKmonofont{AdobeHeitiStd-Regular.otf}

\usepackage{xcolor}
\usepackage{graphicx}

%%% get total page number
\usepackage{lastpage}

%%% customized definition
\makeatletter
\def\sybtitle#1{\def\@sybtitle{#1}}
\def\sybauthor#1{\def\@sybauthor{#1}}
\def\sybdate#1{\def\@sybdate{#1}}
\sybtitle{}
\sybauthor{}
\sybdate{}
\def\sybmaketitle{
  \begin{center}
  \vspace*{.8in}
  {\huge\bfseries\@sybtitle}
  \par
  \vspace{.8in}
  {\Large\@sybauthor}
  \par
  \vspace{.2in}
  \@sybdate
  \vspace{.5in}
  \end{center}
}
\makeatother
\setlength{\parindent}{0pt}
\renewcommand{\today}{\number\month 月 \number\day 日, ~\number\year 年}
\def\lt{\textless}
\def\gt{\textgreater}
\renewcommand\contentsname{\bfseries 目~~录}
\newcommand\bs{\texttt{\symbol{'134}}} % input backslash sign
%\newcommand\bs{\string\} % same as above definition
\long\def\cmd#1{\par\vspace{.5em}\hspace*{2em}#1\vspace{.5em}\par}
\def\cstr#1{\texttt{\string#1}} % e.g. \cstr{\latex}
\long\def\runcode#1{\par\bigskip#1\bigskip\par}
% 我不想看到那么多的underful hbox,尤其是minted环境加上背景色之后
\hbadness=10000
% 适当放宽overful hbox的限制,运行2pt的溢出
\hfuzz=2pt
\parskip=3\lineskip


%%% change background color & add frame for enumerate enviroment
\usepackage{mdframed}
\newmdenv[backgroundcolor=blue!10,linewidth=0pt]{coloredframe}
\newenvironment{coloredenumerate}{
  \begin{coloredframe}
  \begin{enumerate}
}{
  \end{enumerate}
  \end{coloredframe}
}

%%% geometry
\usepackage[includehead,includefoot,hmargin=21mm,vmargin=10.5mm,
            headsep=12pt,headheight=25pt]{geometry}
%\usepackage[includehead,includefoot,hmargin=1.2in,vmargin=1in]{geometry}

%%% fancyhdr
\usepackage{fancyhdr}
\makeatletter
\fancypagestyle{main} {
  \fancyhf{} % clear header & footer
  \fancyhead[L]{\bfseries\@sybtitle}
  \fancyhead[R]{\thepage/\pageref*{LastPage}}
  \renewcommand{\headrulewidth}{0.4pt} % header line
  \renewcommand{\footrulewidth}{0pt} % footer line
}
\fancypagestyle{header} {
  \fancyhf{} % clear header & footer
  \fancyfoot[C]{\roman{page}}
  \renewcommand{\headrulewidth}{0pt} % header line
  \renewcommand{\footrulewidth}{0pt} % footer line
}
\makeatother

\usepackage{titlesec}
\titleformat{\part}{\centering\Large\bfseries}{第\,\thepart\,部分}{1em}{}
\titleformat{\section}{\large\bfseries}{\thesection}{1em}{}
\titleformat{\subsection}{\normalsize\bfseries}{\thesubsection}{1em}{}
%\titlespacing*{章节命令}{左边距}{上文距}{下文距}[右边距]
\titlespacing*{\section}{0pt}{2\baselineskip}{\parsep}
\usepackage{titletoc}
%\titlecontents{标题层次}[左间距]{整体格式}{标题序号}{标题内容}{指引线和页码}[下间距]
%\titlecontents{subsection}[4em]{}{\contentslabel{3em}}{}{\titlerule*[1em]{$\cdot$}\contentspage}


\usepackage[colorlinks=true,linkcolor=blue,linktoc=section]{hyperref}

%%% perfect source code display
\usepackage{minted}
%\usemintedstyle{colorful}
\definecolor{srcbg}{rgb}{0.95,0.95,0.95}
\newminted{java}{linenos,tabsize=4,bgcolor=srcbg}
\newminted{xml}{linenos,tabsize=4,bgcolor=srcbg}
\newminted{cpp}{linenos,tabsize=4,bgcolor=srcbg}
\newminted{bash}{linenos,tabsize=4,bgcolor=srcbg}
\newminted{latex}{linenos,tabsize=4,bgcolor=srcbg}
\newminted{scheme}{linenos,tabsize=4,bgcolor=srcbg}
\newminted{javascript}{linenos,tabsize=4,bgcolor=srcbg}
\newminted{sql}{linenos,tabsize=4,bgcolor=srcbg}
\newminted{make}{linenos,tabsize=4,bgcolor=srcbg}

\usepackage{amsmath}

