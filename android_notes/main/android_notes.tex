\documentclass[a4paper,11pt]{article}


%%% fontenc
%\usepackage{fontspec,xunicode,xltxtra}
%\setmainfont{Times New Roman}
%\setsansfont{Source Sans Pro}
%\setmonofont{Source Sans Pro}

%%% xeCJK
\usepackage{xeCJK}
\setCJKmainfont[BoldFont=Adobe Heiti Std]{Adobe Song Std}
\setCJKsansfont[BoldFont=Adobe Heiti Std]{Adobe Song Std}
\setCJKmonofont[BoldFont=Adobe Heiti Std]{Adobe Song Std}
\XeTeXlinebreaklocale "zh"
\XeTeXlinebreakskip=0pt plus 1pt minus 0.1pt

\usepackage{xcolor}
\usepackage{graphicx}

%%% get total page number
\usepackage{lastpage}

%%% customized definition
\makeatletter
\def\sybtitle#1{\def\@sybtitle{#1}}
\def\sybauthor#1{\def\@sybauthor{#1}}
\def\sybdate#1{\def\@sybdate{#1}}
\sybtitle{}
\sybauthor{}
\sybdate{}
\def\sybmaketitle{
  \begin{center}
  \vspace*{.8in}
  {\huge\bfseries\@sybtitle}
  \par
  \vspace{.8in}
  {\Large\@sybauthor}
  \par
  \vspace{.2in}
  \@sybdate
  \vspace{.5in}
  \end{center}
}
\makeatother
\setlength{\parindent}{0pt}
\renewcommand{\today}{\number\month 月 \number\day 日, ~\number\year 年}
\def\lt{\textless}
\def\gt{\textgreater}
\renewcommand\contentsname{\bfseries 目~~录}
\newcommand\bs{\texttt{\symbol{'134}}} % input backslash sign
%\newcommand\bs{\string\} % same as above definition
\long\def\cmd#1{\par\vspace{.5em}\hspace*{2em}#1\vspace{.5em}\par}
\def\cstr#1{\texttt{\string#1}} % e.g. \cstr{\latex}
\long\def\runcode#1{\par\bigskip#1\bigskip\par}
% 我不想看到那么多的underful hbox,尤其是minted环境加上背景色之后
\hbadness=10000
% 适当放宽overful hbox的限制,运行2pt的溢出
\hfuzz=2pt
\parskip=3\lineskip


%%% change background color & add frame for enumerate enviroment
\usepackage{mdframed}
\newmdenv[backgroundcolor=blue!10,linewidth=0pt]{coloredframe}
\newenvironment{coloredenumerate}{
  \begin{coloredframe}
  \begin{enumerate}
}{
  \end{enumerate}
  \end{coloredframe}
}

%%% geometry
\usepackage[includehead,includefoot,hmargin=21mm,vmargin=10.5mm,
            headsep=12pt,headheight=25pt]{geometry}
%\usepackage[includehead,includefoot,hmargin=1.2in,vmargin=1in]{geometry}

%%% fancyhdr
\usepackage{fancyhdr}
\makeatletter
\fancypagestyle{main} {
  \fancyhf{} % clear header & footer
  \fancyhead[L]{\bfseries\@sybtitle}
  \fancyhead[R]{\thepage/\pageref*{LastPage}}
  \renewcommand{\headrulewidth}{0.4pt} % header line
  \renewcommand{\footrulewidth}{0pt} % footer line
}
\fancypagestyle{header} {
  \fancyhf{} % clear header & footer
  \fancyfoot[C]{\roman{page}}
  \renewcommand{\headrulewidth}{0pt} % header line
  \renewcommand{\footrulewidth}{0pt} % footer line
}
\makeatother

\usepackage{titlesec}
\titleformat{\part}{\centering\Large\bfseries}{第\,\thepart\,部分}{1em}{}
\titleformat{\section}{\large\bfseries}{\thesection}{1em}{}
\titleformat{\subsection}{\normalsize\bfseries}{\thesubsection}{1em}{}
%\titlespacing*{章节命令}{左边距}{上文距}{下文距}[右边距]
\titlespacing*{\section}{0pt}{2\baselineskip}{\parsep}


\usepackage{hyperref}

%%% perfect source code display
\usepackage{minted}
%\usemintedstyle{colorful}
\definecolor{srcbg}{rgb}{0.95,0.95,0.95}
\newminted{java}{linenos,tabsize=4,bgcolor=srcbg}
\newminted{xml}{linenos,tabsize=4,bgcolor=srcbg}
\newminted{cpp}{linenos,tabsize=4,bgcolor=srcbg}
\newminted{bash}{linenos,tabsize=4,bgcolor=srcbg}
\newminted{latex}{linenos,tabsize=4,bgcolor=srcbg}
\newminted{scheme}{linenos,tabsize=4,bgcolor=srcbg}

\usepackage{amsmath}



\sybtitle{Android Notes}
\sybauthor{孙延宾}
\sybdate{\today}

\begin{document}
  \tt % I love Typewriter font.

%%%%%%%% the title page and toc %%%%%%%%%%
  \pagestyle{header}
  \sybmaketitle
  \tableofcontents
  \newpage

%%%%%%% the main content %%%%%%%%%
  \pagestyle{main}
  \setcounter{page}{1}

  \part[Commands for Android]{Android命令行工具}
  \section[logcat - Log dumper for Android] {logcat命令}
  \subsection[filter tag name]{过滤tag}
  使用方法:adb logcat -s \lt TAG-NAME1\gt \lt TAG-NAME2\gt

  \subsection[filter by priority]{按照等级过滤}
  使用方法:adb logcat "TAG:PRIORITY"\\
  或者\\
  adb logcat "*:PRIORITY"

  \subsection[filter by tag and priority]{结合tag和priority}
  使用方法:\\
  adb logcat -s TAG-NAME1:PRIORITY TAG-NAME2:PRIORITY

  例如:\\
  adb logcat -s camera:E ybsolar:I

  priority包括:\\
  V: verbose\\
  D: debug\\
  I: info\\
  W: warning\\
  E: error\\
  F: fatal\\
  S: silent

  \subsection[use grep]{搭配grep}
  使用方法:\\
  adb logcat | grep "key1\bs|key2"

  例如:\\
  adb logcat | grep "Exception\bs|Error"\\
  同时搜索Exception或者Error的log。

  \subsection[clear buffer]{清空缓存}
  使用方法:\\
  adb logcat -c\\
  清空缓存,即清空旧的日志数据。

  \section[screencap]{screencap命令}
  使用方法:adb shell screencap -p | perl -pe 's/\bs x0D\bs x0A/\bs x0A/g' > screen.png
  
  \section[aapt - Android Asset Packaging Tool]{aapt命令}
  使用方法: aapt \textless subcommand\textgreater\ \lt options\gt
  \subsection[print apk file badging information]{打印apk文件概要信息}
  aapt dump badging filename.apk

  apk文件的概括性的信息,主要用来查看versionCode、versionName信息。

  \subsection[print apk file verbose information]{打印apk文件详细信息}
  aapt list -a filename.apk

  输出的信息中包括resource和manifest两方面的内容。manifest文件中的所有信息都被打印出来,
  包括声明的Activity、Receiver、Service,各种permission、intent-filter以及
  versionCode、versionName等等,一应俱全。

  \section[pm - Package Manager]{pm命令}
  使用方法:pm \lt subcommand\gt\ \lt options\gt

  \subsection[list packages]{查看设备上的package信息}
  pm list packages [-f] [-d] [-e] [-s] [-3] FILTER

  \begin{description}
    \item[-f:] 显示package name对应的apk file路径
    \item[-d:] 只显示disabled packages
    \item[-e:] 只显示enabled packages
    \item[-s:] 只显示system packages,apk文件位于/system/app/目录
    \item[-3:] 只显示third party packages,apk文件位于/data/app/目录
    \item[FILTER:] 过滤器,如字符串"camera"将过滤出包名中包含camera的包
  \end{description}
  应用:如何删除不想用的预安装软件呢?
  \begin{coloredenumerate}
    \item 在任务管理器中禁用不想用的应用
    \item adb shell pm list packages -f -d
    \item 删除2中的apk文件
  \end{coloredenumerate}
  应用:使用awk批量删除不想用的预安装应用\\
  adb shell pm list packages -d | gawk -F : '\{system("adb shell pm uninstall " \$2)\}'


  \subsection[package file path]{打印apk文件路径}
  pm path \lt package-name\gt

  \subsection[enable or disable packages]{启用、禁用app}
  pm clear \lt package-name\gt

  清空package的数据信息

  pm enable \lt package-name\gt

  启用某个应用

  pm disable \lt package-name\gt

  禁用某个应用

  \subsection[安装应用]{安装应用}
  pm install [-s] [-f] [-d] [-r] \lt filename.apk\gt
  \begin{description}
    \item[-s:] 将应用安装到sdcard上
    \item[-f:] 将应用安装到内部flash上
    \item[-d:] 允许降级安装,即安装当前应用的低版本apk
    \item[-r:] reinstall app,保持应用数据不被删除
  \end{description}

  \subsection[卸载应用]{卸载应用}
  pm uninstall [-k] \lt package-name\gt

  \begin{description}
    \item[-k:] 保持应用的data和cache不被删除
  \end{description}

  \subsection[查看权限信息]{查看权限信息}
  pm list permissions [-f] [-s]

  \begin{description}
    \item[-f:] 列出所有系统权限信息
    \item[-s:] 只列出权限的摘要信息
  \end{description}


  \section[am - Application Manager]{am命令}
  使用方法:am \lt subcommand\gt\ \lt options\gt

  \subsection[启动Activity]{启动Activity}
  am start [-D] [-W] [-R \lt count\gt] [-S] \lt intent\gt

  \begin{description}
    \item[-D:] 开启debug
    \item[-W:] wait for launch to complete
    \item[-R \lt count\gt:] 重复启动count次,每次启动前先结束之前的Activity
    \item[-S] 强制关闭先前的Activity再启动新的Activity
  \end{description}

  \lt intent\gt 有多种形式:

  \begin{coloredenumerate}
    \item -a \lt ACTION\gt] [-d \lt DATA\_URI\gt] [-t \lt MIME\_TYPE\gt
    \item -c \lt CATEGORY\gt\ [-c \lt CATEGORY\gt] ...
    \item -e|--es \lt EXTRA\_KEY\gt\ \lt EXTRA\_STRING\_VALUE\gt\ ...
    \item --esn \lt EXTRA\_KEY\gt\ ...
    \item --ez \lt EXTRA\_KEY\gt\ \lt EXTRA\_BOOLEAN\_VALUE\gt\ ...
    \item --ei \lt EXTRA\_KEY\gt\ \lt EXTRA\_INT\_VALUE\gt\ ...
    \item --el \lt EXTRA\_KEY\gt\ \lt EXTRA\_LONG\_VALUE\gt\ ...
    \item --ef \lt EXTRA\_KEY\gt\ \lt EXTRA\_FLOAT\_VALUE\gt\ ...
    \item --eu \lt EXTRA\_KEY\gt\ \lt EXTRA\_URI\_VALUE\gt\ ...
    \item --ecn \lt EXTRA\_KEY\gt\ \lt EXTRA\_COMPONENT\_NAME\_VALUE\gt
    \item --eia \lt EXTRA\_KEY\gt\ \lt EXTRA\_INT\_VALUE\gt [,\lt EXTRA\_INT\_VALUE...]
    \item --ela \lt EXTRA\_KEY\gt\ \lt EXTRA\_LONG\_VALUE\gt[,\lt EXTRA\_LONG\_VALUE...]
    \item --efa \lt EXTRA\_KEY\gt\ \lt EXTRA\_FLOAT\_VALUE\gt[,\lt EXTRA\_FLOAT\_VALUE...]
    \item -n \lt COMPONENT\gt] [-f \lt FLAGS\gt
    \item --grant-read-uri-permission] [--grant-write-uri-permission]
    \item --debug-log-resolution] [--exclude-stopped-packages]
    \item --include-stopped-packages]
    \item --activity-brought-to-front] [--activity-clear-top]
    \item --activity-clear-when-task-reset] [--activity-exclude-from-recents]
    \item --activity-launched-from-history] [--activity-multiple-task]
    \item --activity-no-animation] [--activity-no-history]
    \item --activity-no-user-action] [--activity-previous-is-top]
    \item --activity-reorder-to-front] [--activity-reset-task-if-needed]
    \item --activity-single-top] [--activity-clear-task]
    \item --activity-task-on-home
    \item --receiver-registered-only] [--receiver-replace-pending]
    \item --selector
    \item \lt URI\gt\ | \lt PACKAGE\gt\ | \lt COMPONENT\gt
  \end{coloredenumerate}

  \subsubsection[通过package name启动应用]{通过package name启动应用}
  am start -n \lt package-name\gt/\lt package-name.activity-name\gt

  如:am start -n zte.com.cn.camera/zte.com.cn.camera.Camera\\
  可以简写为:\\
  am start -n zte.com.cn.camera/.Camera

  \subsubsection[通过Intent启动应用]{通过Intent启动应用}
  am start -a android.intent.action.CALL -d tel:10086

  \subsection[启动Service]{启动Service}
  am startservice \lt intent\gt

  \subsection[关闭应用]{\underline{关闭应用}}
  am force-stop \lt package-name\gt

  \subsection[发送广播]{发送广播}
  am broadcast \lt intent\gt

  \subsection[dump heap]{dump heap}
  am dumpheap \lt process\gt\ \lt file\gt

  \begin{description}
    \item[process:] 可以是pid,也可以是进程名称(一般为package name)
    \item[file:] 设备上的文件路径
  \end{description}
  如:am dumpheap zte.com.cn.camera /sdcard/camera.hprof

  \subsection[监视系统运行]{监视系统运行}
  am monitor

  启动、恢复app时都会有相应的package name打印。此方法可以很快得知设备上某个应用的package name.

  \subsection[设置设备的显示尺寸、显示密度]{设置设备的显示尺寸、显示密度}
  am display-size [reset|WxH]

  am display-density [reset|density]

  其中,WxH、density均为数值。

  \section[获取系统属性列表]{获取系统属性列表}
  \hspace*{2ex}命令:getprop\par\vspace{2ex}
  该命令打印出当前系统属性(SystemProperties)列表,其中包括dalvik heapsize等。\par
  如:[dalvik.vm.heapsize]: [256m]

  \section[Android上的top命令]{Android上的top命令}
  相比于Linux,top命令在android上被大大简化了。
  \cmd{top [-m max\_procs] [-n iterations] [-d delay] [-s sort\_column] [-t] [-h]}
  \begin{coloredenumerate}
    \item -m max\_procs: 最多显示max\_procs个进程信息(默认显示所有进程信息)
    \item -n iterations: 刷新iterations次就退出(默认一直刷新)
    \item -d delay: 两次刷新之间的时间间隔(单位是秒)
    \item -s sort\_column: 按照哪一行进行排序,可选项为:cpu,vss,rss,thr
    \item -t: 显示线程号而非进程号
    \item -h: 打印帮助信息
  \end{coloredenumerate}
  输出信息解释:

  \begin{center}
  \begin{tabular}{llllllllll}
    PID & PR & CPU\% & S & \#THR & VSS & RSS & PCY & UID & Name\\ \hline
    4145 & 3 & 0\% & S & 14 & 549400K & 89172K & bg & u0\_a11 & zte.com.cn.camera
  \end{tabular}
  \end{center}
  \begin{coloredenumerate}
    \item PID: 进程号
    \item PR: Priority,进程优先级(数字越小,优先级越高)
    \item CPU\%: 进程占用CPU的百分比
    \item S: Status,进程状态:R(run)、S(sleep)、Z(zombie)
    \item \#THR: 进程当前使用的线程数量
    \item VSS: Virtual Set Size 虚拟耗用内存(包含共享库占用的内存)
    \item RSS: Resident Set Size 实际使用物理内存(包含共享库占用的内存)
    \item PCY: 进程前台/后台运行(bg/fg),对应的单词不清楚
    \item UID: user id
    \item Name: 进程名称,在Linux上为启动进程的命令
  \end{coloredenumerate}
  注意:VSS、RSS中的"K"表示kilo bits而非kilo bytes。\par
  示例:
  \cmd{adb shell top -d 5 -s rss -m 10 -n 3}
  每隔5s显示一次系统内存占用量最多的10个进程,总共显示3次。
  
  \section[使用screencap命令进行屏幕截图]{使用screencap命令进行屏幕截图}
  利用screencap在电脑上通过adb shell进行手机屏幕截图,
  截取的图片保存到当前目录下,文件名以当前系统时间命名,
  所以可以通过“程序”实现自动截图。Windows批处理如下:\par\bigskip
  \inputminted[linenos,tabsize=4,bgcolor=srcbg]{bash}{srcdir/capture.bat}
  

  \part[Android Notes List]{Android笔记列表}
  \section[关于package的UID]{关于package的UID}
  在安装apk时,android为application分配一个uid,每个app一个uid,除非使用sharedUid。
	uid不会随着app的启动、关闭而改变,而且重新安装似乎不会改变uid. 但是文档声明只保证
	app在device上时uid不变,如果没有package使用某个uid,该uid将会被删掉。
	查看app的uid有两个方法:
  \begin{coloredenumerate}
    \item 查看系统的packages.xml文件(/data/system/packages.xml),从中找到app对应的package name,
		      其中便有app的uid
    \item 运行app,使用top/ps命令找到对应的pid,然后到/proc目录下查看该进程的信息:\\
      		cat /proc/\lt pid\gt/status
  \end{coloredenumerate}


  \section[Manifest文件中的属性]{Manifest文件中的属性}
  \subsection[Activity的noHistory属性]{Activity的noHistory属性}
  android:noHistory="true"

  该属性置为true时,当Activity A被别的Activity B覆盖时,该Activity A会退出,
  而无法从B返回A,也就是说A不会成为Activity Stack上的“历史”Activity。
  如A正在运行,此时来电,phone结束后,A finish(),但是如果A正在运行,用户按HOME键,
  A不会finish(),因为此时A仍在栈顶。

  \subsection[process属性]{process属性}
  process属性可以用在所有的组件(component)中,用来重新命名组件运行的进程名称(进程名称默认为package name),
  如果process属性值以冒号(:)开头,则组件将在新的进程中运行,该进程为application的私有进程。\par
  \begin{bashcode}
u0_a82    9981  1821  540268 71520 ffffffff 4006e25f S com.example.aidldemo
u0_a82    9995  1821  512052 45268 ffffffff 4006e25f S com.example.aidldemo:Service
  \end{bashcode}

  此时一个package中的组件会被安排到多个进程中执行,而如果不同进程中的组件都要用到preference,
  则要在获取preference对象的时候使用Context.MULTI\_PROCESS标志,而非Context.PRIVATE标志,
  否则通过process属性设置的组件无法获取到正确的preference对象。

  \subsection[exported属性]{exported属性}
  表明组件是否为私有组件,及外部组件是否可以调用该组件。
  \begin{description}
    \item[true:] 该组件为公开的,外部组件可以调用该组件。
    \item[false:] 该组件为私有的,外部无法看到该组件,即使该组件声明了intent-filter也无能为力。
  \end{description}

  \section[onUserLeaveHint()]{onUserLeaveHint()}
  onUserLeaveHint()函数用于提示系统当用户离开之后该怎么做。
  举例来说:当app正在运行,此时来电,app的onPause()被执行;但是,当HOME键被按下时,
  系统会先调用onUserLeaveHint(),然后再调用onPause(),可以在该中调用finish(),
  达到HOME键退出的目的。

  \section[通过Intent监视SCREEN\_ON、SCREEN\_OFF动作]{通过Intent监视SCREEN\_ON、SCREEN\_OFF动作}
  这里要做的事情是:
  \begin{coloredenumerate}
    \item 接收SCREEN\_OFF系统消息
    \item 点亮屏幕
    \item 接收SCREEN\_ON系统消息
    \item 播放视频文件(keep\_screen\_on)
  \end{coloredenumerate}
  \subsection[定义Broadcast receiver接收Intent]{定义Broadcast receiver接收Intent}
  \inputminted[linenos,tabsize=4,bgcolor=srcbg,fontsize=\small]{java}{srcdir/ScreenBroadcastReceiver.java}

  \subsection[在Manifest文件中声明receiver]{在Manifest文件中声明receiver}
  \inputminted[linenos,tabsize=4,bgcolor=srcbg]{xml}{srcdir/AndroidManifest.xml}

  \subsection[使用service注册receiver]{使用service注册receiver}
  \inputminted[linenos,tabsize=4,bgcolor=srcbg]{java}{srcdir/ReceiverService.java}

  \subsection[启动Service的桌面应用]{启动Service的桌面应用}
  \mint[bgcolor=srcbg]{java}|startService(new Intent(this, ReceiverService.class));|

  \section[通过Intent监听系统启动消息]{通过Intent监听系统启动消息}
  知道系统何时启动有时是很重要的,比如系统启动之后立刻启动后台Service监听事件。
  当Android系统启动完毕之后,会发送BOOT\_COMPLETE intent,通过在Receiver中监听该
  intent即可:\par
  \mint[bgcolor=srcbg]{xml}|<action android:name="android.intent.action.BOOT_COMPLETED"/>|

  不过这需求特殊的权限:\par
  \mint[bgcolor=srcbg]{xml}|<uses-permission android:name="android.permission.RECEIVE_BOOT_COMPLETED" />|


  \section[如何在代码中设置View/ViewGroup的尺寸]{如何在代码中设置View/ViewGroup的尺寸}
  在代码中动态设置View的尺寸是很常见的,比如横竖屏切换时,为了保持视频画面的尺寸,
  就需要重新设置SurfaceView/TextureView/VideoView的尺寸。设置的方法是使用LayoutParams类,
  LayoutParams类有多种,这里使用的必须是:该View的父类的LayoutParams类。

\begin{minted}[linenos,tabsize=4,bgcolor=srcbg]{java}
LayoutParams lp = mSurfaceView.getLayoutParams();
lp.width = frameWidth;
lp.height = frameHeight;
mSurfaceView.setLayoutParams(lp);
mSurfaceView.requestLayout();
\end{minted}

  有时getLayoutParams()返回null,这时就要根据该View在XML文件中定义时,其上层对象的类型,
  定义不同LayoutParams种类。不过,应该首先尝试getLayoutParams(),因为重新定义的
  LayoutParams对象需要重新设置其他信息,比如对齐方式等,而get出来的对象都已经设置好了
  (在XML文件中)。

  \section[捕获所有物理按键事件]{捕获所有物理按键事件}
  \begin{minted}[linenos,tabsize=4,bgcolor=srcbg]{java}
@Override
public boolean onKeyDown(int keyCode, KeyEvent event) {
    switch(keyCode) {
        default: {
            Log.d(TAG, "keyCode: " + keyCode + ", event: " + event.getAction());
            finish();
            return true;
        }
    }
}
  \end{minted}

  \section[keep\_screen\_on]{keep\_screen\_on}
  \begin{minted}[linenos,tabsize=4,bgcolor=srcbg]{java}
private void keepScreenOn(boolean on) {
    if(on) {
        getWindow().addFlags(WindowManager.LayoutParams.FLAG_KEEP_SCREEN_ON);
    } else {
        getWindow().clearFlags(WindowManager.LayoutParams.FLAG_KEEP_SCREEN_ON);
    }
}
  \end{minted}

  \section[获取屏幕物理尺寸]{获取屏幕物理尺寸}
  \begin{minted}[linenos,tabsize=4,bgcolor=srcbg]{java}
private void getScreenSize() {
    DisplayMetrics mDisplayMetrics = new DisplayMetrics();
    getWindowManager().getDefaultDisplay().getMetrics(mDisplayMetrics);
    mScreenWidth = mDisplayMetrics.widthPixels;
    mScreenHeight = mDisplayMetrics.heightPixels;
}
  \end{minted}

  \section[使系统静音]{使系统静音}
  \begin{minted}[linenos,tabsize=4,bgcolor=srcbg]{java}
private void muteMusicStream(boolean mute) {
    AudioManager mAudioManager = (AudioManager) getSystemService(AUDIO_SERVICE);
    Log.d(TAG, "mute music stream: " + mute);
    if(mAudioManager != null) {
        mAudioManager.setStreamMute(AudioManager.STREAM_MUSIC, mute);
    }
}
  \end{minted}

  \section[使用渐变背景色模拟动画]{\underline{使用渐变背景色模拟动画}}
  \begin{minted}[linenos,tabsize=4,bgcolor=srcbg]{java}
mPreviewButton = (LinearLayout) findViewById(R.id.footer);
mPreviewButton.setOnClickListener(new OnClickListener() {
    @Override
    public void onClick(View v) {
        v.setBackgroundResource(R.drawable.footer_transition);
        ((TransitionDrawable)v.getBackground()).startTransition(300);
        preview();
    }
});
  \end{minted}

  R.drawable.footer\_transition的定义如下:

  \begin{minted}[linenos,tabsize=4,bgcolor=srcbg]{xml}
<!-- drawable/footer_transition.xml -->
<?xml version="1.0" encoding="utf-8"?>
<transition xmlns:android="http://schemas.android.com/apk/res/android" >
    <item android:drawable="@color/trans_start_color" />
    <item android:drawable="@color/footer_background_color" />
</transition>
  \end{minted}

%\meaning\part
%\the\headsep
%\the\headheight

  \section[Activity的启动模式]{Activity的启动模式}
  android:launchMode变量,它与Intent的FLAG\_ACTIVITY\_*变量共同作用,决定如何启动Activity来处理Intent。

  \begin{description}
    \item[``standard'':] 一个Intent调起一个Activity实例来响应它,一一对应;
    \item[``singleTop'':] 一个Intent调起一个Activity实例,除非它恰好位于Task Stack顶部;
    \item[``singleTask'':] Activity只存在于Task Stack的栈底,该栈中允许有其他Activity;
    \item[``singleInstance'':] Activity实例“独占”一个Task Stack。
  \end{description}

  四个值分为两组:

  \begin{description}
    \item[``standard''、``singleTop'':] Activity可以有多个实例,而且可以处于多个Task Stack中;
    \item[``singleTask''、``singleInstance'':] Activity只有一个实例,而且都处于Task Stack底部。
  \end{description}


  \section[通过Intent传递对象]{通过Intent传递对象}
  Intent发送端:\par
  \begin{javacode}
Intent intent = new Intent(this, ReceiveActivity.class);
Bundle bundle = new Bundle();
bundle.putParcelable("book_info", new Book("Android APP Dev", 42));
intent.putExtras(bundle);

startActivity(intent);
  \end{javacode}

  接收端:\par
  \begin{javacode}
Intent intent = getIntent();
Bundle bundle = intent.getExtras();
Book book = (Book) bundle.getParcelable("book_info");
book.print();
  \end{javacode}

  Book类的定义:\par
  \inputminted[linenos,tabsize=4,bgcolor=srcbg]{java}{srcdir/ParcableBook.java}


  \section[Service组件]{Service组件}
  Service --- Android四大组件之一。

  \subsection[Service的分类]{Service的分类}
  Service分为两类:\par
  \begin{description}
    \item[Started Service:] 通过调用startService()接口启动的Service。这种Service将“无限期”运行在后台,
                            有两种方法可以结束它:在执行过程中调用stopSelf()结束自己、别的组件调用stopService() 结束它。
    \item[Bound Service:] 通过调用bindService()接口启动的Service。在工作完成时,调用者应该调用unBindService()
                          解除绑定。没有任何绑定时,Service将自行被系统销毁。
  \end{description}

  \subsection[Service basic interface]{Service basic interface}
  \begin{coloredenumerate}
    \item onStartCommand(): 当调用startService()时,系统将会调用到该接口,它与Started Service对应。
    \item onBind(): 当调用bindService()时,系统将会调用到该接口,它是Bound Service的启动接口。
    \item onCreate(): Service生命周期的开始,无论哪种Service,启动时都会调用该接口。
    \item onDestroy(): Service生命周期的结束,无论哪种Service,结束时都会调用到该接口。
  \end{coloredenumerate}

  由于started Service有可能无限期运行,所以系统资源紧张时会被kill掉,此时就需要onStartCommand()以返回值
  的形式告知系统,当Service被kill的时候,系统应该如何处理它。返回值有以下几种:\par
  \begin{coloredenumerate}
    \item START\_NOT\_STICKY: Service被kill掉之后,不要重新启动它,除非有Intent提交请求,推荐!
    \item START\_STICKY: Service被kill之后重新启动它,但是不要传递最后一个Intent。
    \item START\_REDELIVER\_INTENT: Service被kill之后,重新启动它并传递最后一个Intent给它。
  \end{coloredenumerate}

  \subsection[Started Service分类]{Started Service分类}
  Started Service分类两类:\par
  \begin{coloredenumerate}
    \item IntentService: 该类Service启动一个worker线程“逐个”、“顺序”处理所有的start请求。
    \item 直接继承Service类,一般用于启动多线程并发处理start请求的情况。
  \end{coloredenumerate}

  \subsection[Bound Service分类]{Bound Service分类}
  Bound Service允许其他组件绑定到它上面并与之交互,该service必须实现onBind()函数,并返回
  一个IBinder对象,该对象是Client与Service交互的接口。多个Client可以同时绑定到Service上。
  实现Bound Service的关键是实现IBinder对象,根据创建IBinder对象的方式可以分为
  以下几类Bound Service:\par
  \begin{coloredenumerate}
    \item 继承IBinder类:Service是app的私有组件并且与Client运行在相同的进程中。
    \item 使用Messenger:Service与Client不在同一个进程中,使用Messenger实现进程间通信,
          Messenger会将多个线程的请求统一到一个消息队列中,供Service逐个处理,所以
          不用考虑Service执行代码的线程安全与否。
    \item 使用AIDL:如果希望Service能够“同时”处理多个请求,可以直接使用AIDL,此时
          Service执行代码必须是多线程的,而且要做到线程安全性。事实上,Messenger的
          底层正是AIDL。
  \end{coloredenumerate}
  Bound Service将放到单独的notes条目\ref{sec:boundservice}中讲解。

  \subsection[Service的生命周期]{Service的生命周期}
  \begin{figure}[ht]
    \centering
    \includegraphics[width=.7\textwidth]{picturedir/life_cycle_of_service.png}\\
    \caption{Life cycle of Service}\label{fig:service}
  \end{figure}
  详细内容见图 \ref{fig:service}

  \subsection[Service v.s. Thread]{Service v.s. Thread}
  如果想要在UI线程以外执行任务,并且该任务只在用户与应用程序交互的过程中执行,
  那么应该创建线程而非Service。

  \subsection[service默认运行在UI线程]{service默认运行在UI线程}
  这一点是如此重要,以至于用一个单独的subsection来记录它。

  \section[Bound Service]{Bound Service}\label{sec:boundservice}
  实现Bound Service的关键是:IBinder接口。
  \subsection[使用IBinder]{使用IBinder}
  \subsubsection[Service端代码]{Service端代码}
  \inputminted[linenos,tabsize=4,bgcolor=srcbg]{java}{srcdir/LocalService.java}

  \subsubsection[Client端代码]{Client端代码}
  \inputminted[linenos,tabsize=4,bgcolor=srcbg]{java}{srcdir/BindingActivity.java}

  \subsection[使用Messenger]{使用Messenger}
  略。

  \subsection[AIDL的使用方法]{AIDL的使用方法}
  所有的代码均在Eclipse环境下运行。\par\bigskip
  AndroidManifest文件:\par
  \inputminted[linenos,tabsize=4,bgcolor=srcbg]{xml}{srcdir/AIDLManifest.xml}

  \subsubsection[Server端代码]{Server端代码}
  AIDL service的aidl文件:\par
  \begin{javacode}
/* IAIDLServerService.aidl */
package com.example.aidldemo.server;

import com.example.aidldemo.server.Book;

interface IAIDLServerService {
    String sayHello();
    Book getBook();
}
  \end{javacode}

  Book类的aidl文件:\par
  \begin{javacode}
/* IBook.aidl */
parcelable Book;
  \end{javacode}

  Service代码:\par
  \inputminted[linenos,tabsize=4,bgcolor=srcbg]{java}{srcdir/AIDLServerService.java}

  Book类代码:\par
  \inputminted[linenos,tabsize=4,bgcolor=srcbg]{java}{srcdir/Book.java}

  \subsubsection[Client端代码]{Client端代码}
  Client端代码,结果将在Logcat中打印出来:\par
  \inputminted[linenos,tabsize=4,bgcolor=srcbg]{java}{srcdir/ClientActivity.java}

  \subsubsection[输出结果]{输出结果}
  \begin{bashcode}
D/ClientActivity(10184): book name: Android Notes, book price: 42
  \end{bashcode}

\end{document}
