
\section[数据库操作]{数据库操作}
Android上与数据库打交道时要用到Cursor类,一般是从ContentResolver中query()到一个
Cursor对象,然后操作Cursor对象。

\subsection[query]{query}
通过ContentResolver的query()方法取得Cursor对象:

\begin{javacode}
Cursor c = android.content.ContentResolver.query(
    Uri uri,
    String[] projection,
    String selection,
    String[] selectionArgs,
    String sortOrder);
\end{javacode}

参数说明:

\begin{itemize}
\item uri: 表示从哪个provider中取数据,\\如android.provider.ContactsContract.Data.CONTENT\_URI
\item projection: 表示要取出哪些列的数据,其他列被忽略,null表示取出所有列,注意效率
\item selection: 表示要取出哪些行的数据,\\如android.provider.ContactsContract.Contacts.DISPLAY\_NAME + "='张三'"
\item selectionArgs: 如果selection中有问号,则按照顺序用这里的字符串替换掉问号
\item sortOrder: 排列顺序,如android.provider.ContactsContract.Contacts.\_ID + " DESC"
\end{itemize}

\subsubsection[第一个参数示例]{第一个参数示例}
\begin{javacode}
Cursor cursor = contentResolver.query(
    android.provider.ContactsContract.Contacts.CONTENT_URI,
    null, // 取出所有列
    null, // 取出所有行
    null,
    null); // 按照默认排序方式或者没有排序
\end{javacode}

\subsubsection[第二个参数示例]{第二个参数示例}
\begin{javacode}
Cursor cursor = contentResolver.query(
    android.provider.ContactsContract.Contacts.CONTENT_URI,
    // 只取出“名字”这一列
    new String[]{ android.provider.ContactsContract.Contacts.DISPLAY_NAME },
    null, // 取出所有行
    null,
    null);
\end{javacode}

\subsubsection[第三个参数示例]{第三个参数示例}
\begin{javacode}
Cursor cursor = contentResolver.query(
    android.provider.ContactsContract.Contacts.CONTENT_URI,
    // 只取出“名字”这一列
    new String[]{ android.provider.ContactsContract.Contacts.DISPLAY_NAME },
    // 只取出“张三”这一行
    android.provider.ContactsContract.Contacts.DISPLAY_NAME + "='张三'",
    null,
    null);
\end{javacode}

\subsubsection[第四个参数示例]{第四个参数示例}
\begin{javacode}
Cursor cursor = contentResolver.query(
    android.provider.ContactsContract.Contacts.CONTENT_URI,
    // 只取出“名字”这一列
    new String[]{ android.provider.ContactsContract.Contacts.DISPLAY_NAME },
    // 取哪些行还不确定,由第四个参数决定
    android.provider.ContactsContract.Contacts.DISPLAY_NAME + "=?",
    // “张三”这一行,也可以是:“张三,李四”,取出两行
    new String[]{ "张三" },
    null);
\end{javacode}

\subsubsection[第五个参数示例]{第五个参数示例}
\begin{javacode}
Cursor cursor = contentResolver.query(
    android.provider.ContactsContract.Contacts.CONTENT_URI,
    // 只取出“名字”这一列
    new String[]{ android.provider.ContactsContract.Contacts.DISPLAY_NAME },
    // 取哪些行还不确定,由第四个参数决定
    android.provider.ContactsContract.Contacts.DISPLAY_NAME + "=?",
    // 取出“张三,李四”两行
    new String[]{ "张三,李四" },
    // 然后按照手机号码的降序排列
    android.provider.ContactsContract.Contacts._ID + " DESC");
\end{javacode}

\subsection[Cursor]{Cursor}
理解和使用Cursor必须知道的几件事:\\
\begin{itemize}
\item Cursor是一个“行”的集合
\item 必须知道列名称和列的数据类型
\end{itemize}

Cursor中的常用方法:

\begin{enumerate}
\item close() 关闭cursor,释放资源
\item copyStringToBuffer(int colIndex, CharArrayBuffer buf)
\item getColumnCount()
\item getColumnIndex(String colName) 取得colName对应的列的索引
\item getColumnIndexOrThrow(String colName)
\item getColumnName(int ColIndex)
\item getColumnNames()
\item getCount() Cursor中的总行数
\item moveToFirst() 移动光标到第一行
\item moveToLast() 移动光标到最后一行
\item moveToNext() 移动到下一行
\item moveToPosition(int position) 移动到某一行
\item moveToPrevious() 移动到上一行
\end{enumerate}

其中moveToFirst(), moveToLast(), moveToNext()用于遍历Cursor,
getColumnIndex(String colName)取得列索引,copyStringToBuffer()
取得某一列的值,另外还有一系列的getXXX()方法,如getInt(), getString()等。

\subsection[三种查询方式]{三种查询方式}
三种查询方式,前两种大同小异,而且都是同步查询,第三种是异步查询,
当数据库信息量比较大时要使用异步查询。

\subsubsection[普通查询方法]{普通查询方法}
最普通的查询方法就是直接调用ContentResolver.query(),注意在使用完
Cursor之后要close()才行。

\subsubsection[让Activity去管理Cursor]{让Activity去管理Cursor}
调用Activity.managedQuery()方法获取Cursor对象,该Cursor对象在Activity
结束时自动关闭,该函数的参数跟ContentResolver.query()方法的参数一样。

\subsubsection[异步查询]{异步查询}
使用AsyncQueryHandler框架进行异步查询。

\begin{javacode}
private final class QueryHandler extends AsyncQueryHandler {
    public QueryHandler(ContentResolver cr) {
        super(cr);
    }

    @Override
    protected void onQueryComplete(int token, Object cookie, Cursor cursor) {
        super.onQueryComplete(token, cookie, cursor);
        // 现在拿到Cursor对象了,调用我们自己的方法进行处理
        dealQueryResult();
    }
}

// 如何使用,startQuery()是AsyncQueryHandler类中的方法
QueryHandler qh = new QueryHandler(getContentResolver());
qh.startQuery(/* params */); // 参数跟query()方法一样
\end{javacode}