
\section[关于package的UID]{关于package的UID}
在安装apk时,android为application分配一个uid,每个app一个uid,除非使用sharedUid。
uid不会随着app的启动、关闭而改变,而且重新安装似乎不会改变uid. 但是文档声明只保证
app在device上时uid不变,如果没有package使用某个uid,该uid将会被删掉。
查看app的uid有两个方法:
\begin{coloredenumerate}
  \item 查看系统的packages.xml文件(/data/system/packages.xml),从中找到app对应的package name,
	      其中便有app的uid
  \item 运行app,使用top/ps命令找到对应的pid,然后到/proc目录下查看该进程的信息:\\
    		cat /proc/\lt pid\gt/status
\end{coloredenumerate}


\section[Manifest文件中的属性]{Manifest文件中的属性}
\subsection[Activity的noHistory属性]{Activity的noHistory属性}
android:noHistory="true"

该属性置为true时,当Activity A被别的Activity B覆盖时,该Activity A会退出,
而无法从B返回A,也就是说A不会成为Activity Stack上的“历史”Activity。
如A正在运行,此时来电,phone结束后,A finish(),但是如果A正在运行,用户按HOME键,
A不会finish(),因为此时A仍在栈顶。

\subsection[process属性]{process属性}
process属性可以用在所有的组件(component)中,用来重新命名组件运行的进程名称(进程名称默认为package name),
如果process属性值以冒号(:)开头,则组件将在新的进程中运行,该进程为application的私有进程。\par
\begin{bashcode}
u0_a82    9981  1821  540268 71520 ffffffff 4006e25f S com.example.aidldemo
u0_a82    9995  1821  512052 45268 ffffffff 4006e25f S com.example.aidldemo:Service
\end{bashcode}

此时一个package中的组件会被安排到多个进程中执行,而如果不同进程中的组件都要用到preference,
则要在获取preference对象的时候使用Context.MULTI\_PROCESS标志,而非Context.PRIVATE标志,
否则通过process属性设置的组件无法获取到正确的preference对象。

\subsection[exported属性]{exported属性}
表明组件是否为私有组件,及外部组件是否可以调用该组件。
\begin{description}
  \item[true:] 该组件为公开的,外部组件可以调用该组件。
  \item[false:] 该组件为私有的,外部无法看到该组件,即使该组件声明了intent-filter也无能为力。
\end{description}


\subsection[launchMode]{launchMode}
注意:Task是用来管理Activity的,它确定了多个Activity之间的关系,
至于这些Activity运行的进程、线程它从不过问,即两个Task中的Activity
显然是两个实例,但是它们可以运行在同一个线程中(如正常启动Camera和
从短信中调用Camera,此时就是同一个进程、同一个线程中存在两个Camera
Activity的实例),反过来同一个Task中的两个Activity也可以运行在不同
的进程中,这是很显然的,短信调用Camera,Camera不可能运行在短信应用
的进程中去,很显然嘛,因为Camera需要的资源都在Camera的Application中啊!
\vspace{1em}
android:launchMode变量,它与Intent的FLAG\_ACTIVITY\_*变量共同作用,决定如何启动Activity来处理Intent。
两者都是在标准启动方式的基础上增加一些额外的启动条件,所以两者共同决定的结果就是:

实际的启动模式必须既满足android:launchMode规定的条件又满足Intent里规定的条件。

例如,launchMode设置为Standard而Intent的flag设置为FLAG\_ACTIVITY\_SINGLE\_TOP,
则Activity就按照singleTop来启动。

\subsubsection[android:launchMode]{android:launchMode}
android:launchMode有以下四种,
\begin{description}
  \item[``standard'':] One Intent create one (this kind) activity.
  
  \item[``singleTop'':] This activity is single on top. \\
  即不允许出现该activity的“多个”实例同时位于stack top的情况。
  当栈顶activity调用它自身时就会出现这种情况,此时不会创建新的实例,
  而是调用activity的onNewIntent()方法。
  
  \item[``singleTask'':] This activity is single in task(and in Android System,但是该栈中可以有其他activity).
    \begin{itemize}
      \item 具有相同affinity的其他activity调用singleTask的activity时(如同一个application中的activity或者设置了
      android:taskAffinity属性的activities),不会创建新栈而是在当前栈中入栈。
      \item 具有不同affinity的其他activity调用singleTask的activity时,创建新栈并作为根元素将其置于栈底。
      
      \item singleTask的activity可以正常调用别的activity并将它们入栈(置于自己的上面)。
      
      \item 如果singleTask的activity不在栈顶,此时它被作为target activity调用时(它上面的其他activity调用它),
      则“其上”的所有activity被弹出销毁。
    \end{itemize}
  
  \item[``singleInstance'':] This activity is single in Android System,并且“独占”一个stack。
\end{description}

在系统中,singleTask、singleInstance的activity都只允许一个实例存在!

注意:singleInstance的activity,由于它一定是栈底的根元素,所以如果它没有在桌面注册图标,
按下HOME键后就再也回不到它的“即时界面”了,例如:D是singleInstance的,当前栈(stack1):A, B, C,C调用D,
则系统会创建新的栈(stack2)并将D置于栈底,此时HOME按下,因为D没有在Launcher注册,所以无法调起stack2,
可以调起stack1(如果A在Launcher有注册的话),但是此时显示的是C的界面。如果再次调用D就会再次进入stack2,
但是D会被重新初始化(存疑?),第一次调起D时的临时界面永远丢失了。

再注意:四个modes将activity分为四类,进一步分为两组:standard和singleTop一组,singleTask和singleInstance一组,
第一组适合绝大多数activity,后一组使用很罕见,尽量不要使用。

\subsubsection[Intent Flags]{Intent Flags}

\begin{description}
  \item[FLAG\_ACTIVITY\_CLEAR\_TOP] 如果target activity已经在当前栈中,则将“其上”的activities销毁。\\
  如当前task stack是这样的:A, B, C, D,D位于栈顶,此时D调用B,并且在Intent中设置该参数,则C、D被弹出销毁。
  当然,如果target activity压根儿不在当前栈中,该参数就不起任何作用了。
  \item[FLAG\_ACTIVITY\_SINGLE\_TOP] 如果target activity已经在当前栈中,并且它位于栈顶,则不创建新实例而是调用其
  onNewIntent()方法。\\
  该参数跟android:launchMode中的singleTop作用一样,需要注意的是,这种情况只有一种,那就是栈顶activity调用
  它自身。
  \item[FLAG\_ACTIVITY\_NEW\_TASK] 要求新的activity放到别的栈中,系统会为该Intent寻找新的stack,寻找的依据
  就是activity的android:taskAffinity属性,如果找到一个栈,它的root activity也具有一样的affinity就创建一个新实例
  (或者如果已经存在一个实例,就清空其上的activity),并入栈;如果不存在这样的栈就新创建一个,并入栈。
  \item[FLAG\_ACTIVITY\_NO\_HISTORY] target activity不入栈。\\
  如当前栈情况:A, B, C,C在栈顶,它调用D并且设置该参数,则D启动之后当前栈的情况仍然是:A, B, C,D并不入栈,
  如果此时D再“正常”调用E,则当前栈变为:A, B, C, E,E返回之后直接进入C,D就像没有存在过一样。
\end{description}


\section[onUserLeaveHint()]{onUserLeaveHint()}
onUserLeaveHint()函数用于提示系统当用户离开之后该怎么做。
举例来说:当app正在运行,此时来电,app的onPause()被执行;但是,当HOME键被按下时,
系统会先调用onUserLeaveHint(),然后再调用onPause(),可以在该中调用finish(),
达到HOME键退出的目的。

\section[通过Intent监视SCREEN\_ON、SCREEN\_OFF动作]{通过Intent监视SCREEN\_ON、SCREEN\_OFF动作}
这里要做的事情是:
\begin{coloredenumerate}
  \item 接收SCREEN\_OFF系统消息
  \item 点亮屏幕
  \item 接收SCREEN\_ON系统消息
  \item 播放视频文件(keep\_screen\_on)
\end{coloredenumerate}
\subsection[定义Broadcast receiver接收Intent]{定义Broadcast receiver接收Intent}
\inputminted[linenos,tabsize=4,bgcolor=srcbg,fontsize=\small]{java}{srcdir/ScreenBroadcastReceiver.java}

\subsection[在Manifest文件中声明receiver]{在Manifest文件中声明receiver}
\inputminted[linenos,tabsize=4,bgcolor=srcbg]{xml}{srcdir/AndroidManifest.xml}

\subsection[使用service注册receiver]{使用service注册receiver}
\inputminted[linenos,tabsize=4,bgcolor=srcbg]{java}{srcdir/ReceiverService.java}

\subsection[启动Service的桌面应用]{启动Service的桌面应用}
\mint[bgcolor=srcbg]{java}|startService(new Intent(this, ReceiverService.class));|

\section[通过Intent监听系统启动消息]{通过Intent监听系统启动消息}
知道系统何时启动有时是很重要的,比如系统启动之后立刻启动后台Service监听事件。
当Android系统启动完毕之后,会发送BOOT\_COMPLETE intent,通过在Receiver中监听该
intent即可:\par
\mint[bgcolor=srcbg]{xml}|<action android:name="android.intent.action.BOOT_COMPLETED"/>|

不过这需求特殊的权限:\par
\mint[bgcolor=srcbg]{xml}|<uses-permission android:name="android.permission.RECEIVE_BOOT_COMPLETED" />|


\section[如何在代码中设置View/ViewGroup的尺寸]{如何在代码中设置View/ViewGroup的尺寸}
在代码中动态设置View的尺寸是很常见的,比如横竖屏切换时,为了保持视频画面的尺寸,
就需要重新设置SurfaceView/TextureView/VideoView的尺寸。设置的方法是使用LayoutParams类,
LayoutParams类有多种,这里使用的必须是:该View的父类的LayoutParams类。

\begin{minted}[linenos,tabsize=4,bgcolor=srcbg]{java}
LayoutParams lp = mSurfaceView.getLayoutParams();
lp.width = frameWidth;
lp.height = frameHeight;
mSurfaceView.setLayoutParams(lp);
mSurfaceView.requestLayout();
\end{minted}

有时getLayoutParams()返回null,这时就要根据该View在XML文件中定义时,其上层对象的类型,
定义不同LayoutParams种类。不过,应该首先尝试getLayoutParams(),因为重新定义的
LayoutParams对象需要重新设置其他信息,比如对齐方式等,而get出来的对象都已经设置好了
(在XML文件中)。

\section[捕获所有物理按键事件]{捕获所有物理按键事件}
\begin{minted}[linenos,tabsize=4,bgcolor=srcbg]{java}
@Override
public boolean onKeyDown(int keyCode, KeyEvent event) {
  switch(keyCode) {
      default: {
          Log.d(TAG, "keyCode: " + keyCode + ", event: " + event.getAction());
          finish();
          return true;
      }
  }
}
\end{minted}

\section[keep\_screen\_on]{keep\_screen\_on}
\begin{minted}[linenos,tabsize=4,bgcolor=srcbg]{java}
private void keepScreenOn(boolean on) {
  if(on) {
      getWindow().addFlags(WindowManager.LayoutParams.FLAG_KEEP_SCREEN_ON);
  } else {
      getWindow().clearFlags(WindowManager.LayoutParams.FLAG_KEEP_SCREEN_ON);
  }
}
\end{minted}

\section[获取屏幕物理尺寸]{获取屏幕物理尺寸}
\begin{minted}[linenos,tabsize=4,bgcolor=srcbg]{java}
private void getScreenSize() {
  DisplayMetrics mDisplayMetrics = new DisplayMetrics();
  getWindowManager().getDefaultDisplay().getMetrics(mDisplayMetrics);
  mScreenWidth = mDisplayMetrics.widthPixels;
  mScreenHeight = mDisplayMetrics.heightPixels;
}
\end{minted}

\section[使系统静音]{使系统静音}
\begin{minted}[linenos,tabsize=4,bgcolor=srcbg]{java}
private void muteMusicStream(boolean mute) {
  AudioManager mAudioManager = (AudioManager) getSystemService(AUDIO_SERVICE);
  Log.d(TAG, "mute music stream: " + mute);
  if(mAudioManager != null) {
      mAudioManager.setStreamMute(AudioManager.STREAM_MUSIC, mute);
  }
}
\end{minted}

\section[使用渐变背景色模拟动画]{\underline{使用渐变背景色模拟动画}}
\begin{minted}[linenos,tabsize=4,bgcolor=srcbg]{java}
mPreviewButton = (LinearLayout) findViewById(R.id.footer);
mPreviewButton.setOnClickListener(new OnClickListener() {
  @Override
  public void onClick(View v) {
      v.setBackgroundResource(R.drawable.footer_transition);
      ((TransitionDrawable)v.getBackground()).startTransition(300);
      preview();
  }
});
\end{minted}

R.drawable.footer\_transition的定义如下:

\begin{minted}[linenos,tabsize=4,bgcolor=srcbg]{xml}
<!-- drawable/footer_transition.xml -->
<?xml version="1.0" encoding="utf-8"?>
<transition xmlns:android="http://schemas.android.com/apk/res/android" >
  <item android:drawable="@color/trans_start_color" />
  <item android:drawable="@color/footer_background_color" />
</transition>
\end{minted}

%\meaning\part
%\the\headsep
%\the\headheight

\section[通过Intent传递对象]{通过Intent传递对象}
Intent发送端:\par
\begin{javacode}
Intent intent = new Intent(this, ReceiveActivity.class);
Bundle bundle = new Bundle();
bundle.putParcelable("book_info", new Book("Android APP Dev", 42));
intent.putExtras(bundle);

startActivity(intent);
\end{javacode}

接收端:\par
\begin{javacode}
Intent intent = getIntent();
Bundle bundle = intent.getExtras();
Book book = (Book) bundle.getParcelable("book_info");
book.print();
\end{javacode}

Book类的定义:\par
\inputminted[linenos,tabsize=4,bgcolor=srcbg]{java}{srcdir/ParcableBook.java}
