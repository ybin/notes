
\section[logcat - Log dumper for Android] {logcat命令}
\subsection[filter tag name]{过滤tag}
使用方法:adb logcat -s \lt TAG-NAME1\gt \lt TAG-NAME2\gt

\subsection[filter by priority]{按照等级过滤}
使用方法:adb logcat "TAG:PRIORITY"\\
或者\\
adb logcat "*:PRIORITY"

\subsection[filter by tag and priority]{结合tag和priority}
使用方法:\\
adb logcat -s TAG-NAME1:PRIORITY TAG-NAME2:PRIORITY

例如:\\
adb logcat -s camera:E ybsolar:I

priority包括:\\
V: verbose\\
D: debug\\
I: info\\
W: warning\\
E: error\\
F: fatal\\
S: silent

\subsection[use grep]{搭配grep}
使用方法:\\
adb logcat | grep "key1\bs|key2"

例如:\\
adb logcat | grep "Exception\bs|Error"\\
同时搜索Exception或者Error的log。

\subsection[clear buffer]{清空缓存}
使用方法:\\
adb logcat -c\\
清空缓存,即清空旧的日志数据。

\section[screencap]{screencap命令}
使用方法:adb shell screencap -p | perl -pe 's/\bs x0D\bs x0A/\bs x0A/g' > screen.png

\section[aapt - Android Asset Packaging Tool]{aapt命令}
使用方法: aapt \textless subcommand\textgreater\ \lt options\gt
\subsection[print apk file badging information]{打印apk文件概要信息}
aapt dump badging filename.apk

apk文件的概括性的信息,主要用来查看versionCode、versionName信息。

\subsection[print apk file verbose information]{打印apk文件详细信息}
aapt list -a filename.apk

输出的信息中包括resource和manifest两方面的内容。manifest文件中的所有信息都被打印出来,
包括声明的Activity、Receiver、Service,各种permission、intent-filter以及
versionCode、versionName等等,一应俱全。

\section[pm - Package Manager]{pm命令}
使用方法:pm \lt subcommand\gt\ \lt options\gt

\subsection[list packages]{查看设备上的package信息}
pm list packages [-f] [-d] [-e] [-s] [-3] FILTER

\begin{description}
\item[-f:] 显示package name对应的apk file路径
\item[-d:] 只显示disabled packages
\item[-e:] 只显示enabled packages
\item[-s:] 只显示system packages,apk文件位于/system/app/目录
\item[-3:] 只显示third party packages,apk文件位于/data/app/目录
\item[FILTER:] 过滤器,如字符串"camera"将过滤出包名中包含camera的包
\end{description}
应用:如何删除不想用的预安装软件呢?
\begin{coloredenumerate}
\item 在任务管理器中禁用不想用的应用
\item adb shell pm list packages -f -d
\item 删除2中的apk文件
\end{coloredenumerate}
应用:使用awk批量删除不想用的预安装应用\\
adb shell pm list packages -d | gawk -F : '\{system("adb shell pm uninstall " \$2)\}'


\subsection[package file path]{打印apk文件路径}
pm path \lt package-name\gt

\subsection[enable or disable packages]{启用、禁用app}
pm clear \lt package-name\gt

清空package的数据信息

pm enable \lt package-name\gt

启用某个应用

pm disable \lt package-name\gt

禁用某个应用

\subsection[安装应用]{安装应用}
pm install [-s] [-f] [-d] [-r] \lt filename.apk\gt
\begin{description}
\item[-s:] 将应用安装到sdcard上
\item[-f:] 将应用安装到内部flash上
\item[-d:] 允许降级安装,即安装当前应用的低版本apk
\item[-r:] reinstall app,保持应用数据不被删除
\end{description}

\subsection[卸载应用]{卸载应用}
pm uninstall [-k] \lt package-name\gt

\begin{description}
\item[-k:] 保持应用的data和cache不被删除
\end{description}

\subsection[查看权限信息]{查看权限信息}
pm list permissions [-f] [-s]

\begin{description}
\item[-f:] 列出所有系统权限信息
\item[-s:] 只列出权限的摘要信息
\end{description}


\section[am - Application Manager]{am命令}
使用方法:am \lt subcommand\gt\ \lt options\gt

\subsection[启动Activity]{启动Activity}
am start [-D] [-W] [-R \lt count\gt] [-S] \lt intent\gt

\begin{description}
\item[-D:] 开启debug
\item[-W:] wait for launch to complete
\item[-R \lt count\gt:] 重复启动count次,每次启动前先结束之前的Activity
\item[-S] 强制关闭先前的Activity再启动新的Activity
\end{description}

\lt intent\gt 有多种形式:

\begin{coloredenumerate}
\item -a \lt ACTION\gt] [-d \lt DATA\_URI\gt] [-t \lt MIME\_TYPE\gt
\item -c \lt CATEGORY\gt\ [-c \lt CATEGORY\gt] ...
\item -e|--es \lt EXTRA\_KEY\gt\ \lt EXTRA\_STRING\_VALUE\gt\ ...
\item --esn \lt EXTRA\_KEY\gt\ ...
\item --ez \lt EXTRA\_KEY\gt\ \lt EXTRA\_BOOLEAN\_VALUE\gt\ ...
\item --ei \lt EXTRA\_KEY\gt\ \lt EXTRA\_INT\_VALUE\gt\ ...
\item --el \lt EXTRA\_KEY\gt\ \lt EXTRA\_LONG\_VALUE\gt\ ...
\item --ef \lt EXTRA\_KEY\gt\ \lt EXTRA\_FLOAT\_VALUE\gt\ ...
\item --eu \lt EXTRA\_KEY\gt\ \lt EXTRA\_URI\_VALUE\gt\ ...
\item --ecn \lt EXTRA\_KEY\gt\ \lt EXTRA\_COMPONENT\_NAME\_VALUE\gt
\item --eia \lt EXTRA\_KEY\gt\ \lt EXTRA\_INT\_VALUE\gt [,\lt EXTRA\_INT\_VALUE...]
\item --ela \lt EXTRA\_KEY\gt\ \lt EXTRA\_LONG\_VALUE\gt[,\lt EXTRA\_LONG\_VALUE...]
\item --efa \lt EXTRA\_KEY\gt\ \lt EXTRA\_FLOAT\_VALUE\gt[,\lt EXTRA\_FLOAT\_VALUE...]
\item -n \lt COMPONENT\gt] [-f \lt FLAGS\gt
\item --grant-read-uri-permission] [--grant-write-uri-permission]
\item --debug-log-resolution] [--exclude-stopped-packages]
\item --include-stopped-packages]
\item --activity-brought-to-front] [--activity-clear-top]
\item --activity-clear-when-task-reset] [--activity-exclude-from-recents]
\item --activity-launched-from-history] [--activity-multiple-task]
\item --activity-no-animation] [--activity-no-history]
\item --activity-no-user-action] [--activity-previous-is-top]
\item --activity-reorder-to-front] [--activity-reset-task-if-needed]
\item --activity-single-top] [--activity-clear-task]
\item --activity-task-on-home
\item --receiver-registered-only] [--receiver-replace-pending]
\item --selector
\item \lt URI\gt\ | \lt PACKAGE\gt\ | \lt COMPONENT\gt
\end{coloredenumerate}

\subsubsection[通过package name启动应用]{通过package name启动应用}
am start -n \lt package-name\gt/\lt package-name.activity-name\gt

如:am start -n zte.com.cn.camera/zte.com.cn.camera.Camera\\
可以简写为:\\
am start -n zte.com.cn.camera/.Camera

\subsubsection[通过Intent启动应用]{通过Intent启动应用}
am start -a android.intent.action.CALL -d tel:10086

\subsection[启动Service]{启动Service}
am startservice \lt intent\gt

\subsection[关闭应用]{\underline{关闭应用}}
am force-stop \lt package-name\gt

\subsection[发送广播]{发送广播}
am broadcast \lt intent\gt

\subsection[dump heap]{dump heap}
am dumpheap \lt process\gt\ \lt file\gt

\begin{description}
\item[process:] 可以是pid,也可以是进程名称(一般为package name)
\item[file:] 设备上的文件路径
\end{description}
如:am dumpheap zte.com.cn.camera /sdcard/camera.hprof

\subsection[监视系统运行]{监视系统运行}
am monitor

启动、恢复app时都会有相应的package name打印。此方法可以很快得知设备上某个应用的package name.

\subsection[设置设备的显示尺寸、显示密度]{设置设备的显示尺寸、显示密度}
am display-size [reset|WxH]

am display-density [reset|density]

其中,WxH、density均为数值。

\section[获取系统属性列表]{获取系统属性列表}
\hspace*{2ex}命令:getprop\par\vspace{2ex}
该命令打印出当前系统属性(SystemProperties)列表,其中包括dalvik heapsize等。\par
如:[dalvik.vm.heapsize]: [256m]

\section[Android上的top命令]{Android上的top命令}
相比于Linux,top命令在android上被大大简化了。
\cmd{top [-m max\_procs] [-n iterations] [-d delay] [-s sort\_column] [-t] [-h]}
\begin{coloredenumerate}
\item -m max\_procs: 最多显示max\_procs个进程信息(默认显示所有进程信息)
\item -n iterations: 刷新iterations次就退出(默认一直刷新)
\item -d delay: 两次刷新之间的时间间隔(单位是秒)
\item -s sort\_column: 按照哪一行进行排序,可选项为:cpu,vss,rss,thr
\item -t: 显示线程号而非进程号
\item -h: 打印帮助信息
\end{coloredenumerate}
输出信息解释:

\begin{center}
\begin{tabular}{llllllllll}
PID & PR & CPU\% & S & \#THR & VSS & RSS & PCY & UID & Name\\ \hline
4145 & 3 & 0\% & S & 14 & 549400K & 89172K & bg & u0\_a11 & zte.com.cn.camera
\end{tabular}
\end{center}
\begin{coloredenumerate}
\item PID: 进程号
\item PR: Priority,进程优先级(数字越小,优先级越高)
\item CPU\%: 进程占用CPU的百分比
\item S: Status,进程状态:R(run)、S(sleep)、Z(zombie)
\item \#THR: 进程当前使用的线程数量
\item VSS: Virtual Set Size 虚拟耗用内存(包含共享库占用的内存)
\item RSS: Resident Set Size 实际使用物理内存(包含共享库占用的内存)
\item PCY: 进程前台/后台运行(bg/fg),对应的单词不清楚
\item UID: user id
\item Name: 进程名称,在Linux上为启动进程的命令
\end{coloredenumerate}
注意:VSS、RSS中的"K"表示kilo bits而非kilo bytes。\par
示例:
\cmd{adb shell top -d 5 -s rss -m 10 -n 3}
每隔5s显示一次系统内存占用量最多的10个进程,总共显示3次。

\section[使用screencap命令进行屏幕截图]{使用screencap命令进行屏幕截图}
利用screencap在电脑上通过adb shell进行手机屏幕截图,
截取的图片保存到当前目录下,文件名以当前系统时间命名,
所以可以通过“程序”实现自动截图。Windows批处理如下:\par\bigskip
\inputminted[linenos,tabsize=4,bgcolor=srcbg]{bash}{srcdir/capture.bat}
