
\section[Context类的构成]{Context类的构成}
Context对象作为整个Application的运行环境,它的构成其实并不复杂。
Activity, Service, Broadcast, ContentProvider四大组件都是通过
代理模式(implement Context,然后将真正的实现转移到ContextImpl对象中)
连接到真正实现Context接口的对象中。

\subsection[application中的各种资源]{application中的各种资源}
一个application有很多资源,图片、字符串、甚至是音视频文件等,Context提供了
访问这些资源的接口,

\begin{javacode}
// 获取Context对象本身
public Context getApplicationContext();
// assets/目录
public AssetManager getAssets();
// res/目录
public Resources getResources();
// 获取ContentResolver,用于访问数据库
public ContentResolver getContentResolver();
// 获取package manager,每个app都是有pm管理的
public PackageManager getPackageManager();
// 别的线程是如何获取到主线程的looper的呢,here it is!
public Looper getMainLooper();
public boolean startInstrumentation(
             ComponentName className,
             String profileFile,
             Bundle arguments);
// !IMPORTANT 获取系统服务
public Object getSystemService(String name);
// 创建Context对象
public Context createPakcageContext(String name, int flags);
\end{javacode}

\subsection[application相关信息]{application相关信息}
application的相关信息

\begin{javacode}
public ClassLoader getClassLoader();
public String getPakcageName();
public ApplicationInfo getApplicationInfo();
public String getPackageResourcePath();
public String getPackageCodePath();
\end{javacode}

\subsection[theme相关操作]{theme相关操作}
theme相关的操作,这些已经不推荐使用,使用ThemeManager代替

\begin{javacode}
public void setTheme(int resId);
public int getThemeResId();
public Resources.Theme getTheme();
\end{javacode}

\subsection[wallpaper相关操作]{wallpaper相关操作}
wallpaper相关操作,不推荐使用,使用WallpaperManager代替

\begin{javacode}
public Drawable getWallpaper();
public Drawable peekWallpaper();
public getWallpaperDesiredMinimumWidth();
public getWallpaperDesiredMinimumHeight();
public void setWallpaper(Bitmap bitmap);
public void setWallpaper(InputStream data);
public void clearWallpaper();
\end{javacode}

\subsection[file相关的操作]{file相关的操作}
application的“私有”文件相关操作,只需文件名,不需要具体路径

\begin{javacode}
public SharedPreferences getSharedPreferences(String name, int mode);
public FileInputString openFileInput(String name);
public FileOutputString openFileOutput(String name, int mode);
public boolean deleteFile(String name);
public File getFileStreamPath(String name);
public String[] fileList();
public File getFilesDir();
public File getExternalFilesDir(String type);
public File getObbDir();
public File getCacheDir();
public File getExternalCacheDir();
public File getDir(String name, int mode);
\end{javacode}

\subsection[database相关操作]{database相关操作}
database相关操作

\begin{javacode}
public SQLiteDatabase openOrCreateDatabase(
    String name, int mode, CursorFactory factory);
public SQLiteDatabase openOrCreateDatabase(
    String name, int mode, CursorFactory factory, DatabaseErrorHandler errorHandler);
public boolean deleteDatabase(String name);
public File getDatabasePath(String name);
public String[] databaseList();
\end{javacode}

\subsection[activity相关操作]{activity相关操作}
!IMPORTANT activity相关操作

\begin{javacode}
public void startActivity(Intent intent);
public void startActivity(Intent intent, Bundle options);
public void startActivities(Intent[] intents);
public void startActivities(Intent[] intents, Bundle options);
\end{javacode}

\subsection[broadcast相关操作]{broadcast相关操作}
!IMPORTANT broadcast相关操作

\begin{javacode}
public void sendBroadcast(Intent intent);
public void sendBroadcast(Intent intent, int userId);
public void sendBroadcast(Intent intent, String receiverPermission);
public void sendOrderedBroadcast(Intent intent, String receiverPermission);
public void sendOrderedBroadcast(
            Intent intent,
            String receiverPermission,
            BroadcastReceiver resultReceiver,
            Hanler scheduler,
            int initialCode,
            String initialData,
            Bundle initialExtra);
public void sendStickyBroadcast(Intent intent);
public void sendStickyOrderedBroadcast(
            Intent intent,
            String receiverPermission,
            BroadcastReceiver resultReceiver,
            Hanler scheduler,
            int initialCode,
            String initialData,
            Bundle initialExtra);
public void removeStickyBroadcast(Intent intent);
public Intent registerReceiver(BroadcastReceiver receiver, IntentFilter filter);
public Intent registerReceiver(
            BroadcastReceiver receiver,
            IntentFilter filter,
            String broadcastPermission,
            Hanler scheduler);
public void unregisterReceiver(BroadcastReceiver receiver);
\end{javacode}

\subsection[service相关操作]{service相关操作}
!IMPORTANT service相关操作

\begin{javacode}
public ComponentName startService(Intent service);
public boolean stopService(Intent name);
public boolean bindService(Intent service, ServiceConnection conn, int flags);
public boolean bindService(Intent service, ServiceConnection conn, int flags, int userId);
public void unbindService(ServiceConnection conn);
\end{javacode}

\subsection[permission相关操作]{permission相关操作}
!IMPORTANT permission相关操作

\begin{javacode}
public int checkPermission(String permission, int pid, int uid);
public int checkCallingPermission(String permission);
public int checkCallingOrSelfPermission(String permission);
public void enforcePermission(String permission, int pid, int uid, String message);
public void enforceCallingPermission(String permission, String message);
public void enforceCallingOrSelfPermission(String permission, String message);
public void grantUriPermission(String toPackage, Uri uri, int modeFlags);
public void revokeUriPermission(Uri uri, int modeFlags);
public int checkUriPermission(Uri uri, int pid, int uid, int modeFlags);
public int checkCallingUriPermission(Uri uri, int modeFlags);
public int checkCallingOrSelfUriPermission(Uri uri, int modeFlags);
public int checkUriPermission(Uri uri, String readPermission,
            String writePermission, int pid, int uid, int modeFlags);
public void enforceUriPermission(Uri uri, int pid, int uid,
            int modeFlags, String message);
public void enforceCallingUriPermission(Uri uri, int modeFlags, String message);
public void enforceCallingOrSelfUriPermission(Uri uri, int modeFlags,
            String message);
public void enforceUriPermission(Uri uri, String readPermission,
            String writePermission, int pid, int uid, int modeFlags,String message);
\end{javacode}

\section[Component]{Component}
Component是各种组件的统称,有时需要制定某个Component,就可以创建一个Component对象,
如通过Intent启动一个service,该service在onStartCommand()中处理各种action,但是我们
不想将各种action暴露出来(即Manifest文件中没有注册各种action),那该如何通过Intent
调用该service呢?

此时可以通过Intent的setComponent(Component c)函数来指定目标service,然后通过
Intent.setAction(String action)来指定各种action。

\begin{javacode}
Intent i = new Intent("com.example.syb.service.play_action");
Component c = new Component("com.example.syb", "com.example.syb.service.ActionService");
i.setComponent(c);
startService(i);
\end{javacode}

注意new Component时的参数,pkg参数为app的package,也即app的进程名称,第一个参数
必须为'root' package name;一个app里完全可以有多个java package,此时第二个参数的
classname就要用全称。