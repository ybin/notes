\section[绘图那些事儿]{绘图那些事儿}
在Android里绘图有两种途径:Canvas和OpenGL ES,前者兼容性广,放之四海而皆准,
各种Android版本通吃,后者性能好,但是兼容性不好,尤其是在较低的版本中。
无论如何,Android关心的只是Bitmap,而两者只是生成Bitmap的手段而已,Bitmap
才是最终目的。

现在我们暂且放下OpenGL ES,把注意力关注到Canvas上面。

回到现实世界,绘画是需要一些基本设备的:画布和画笔,人拿着画笔在画布上作画。
但是在Android中没有“人”,于是我们抽象一下,赋予画布绘画的功能,如画布可以
画点,画线,画圆,画椭圆,甚至给它一个定义好的Path它也可以画出来。至于画笔,
可以设置它的颜色,样式(stroke or fill or stroke \& fill),甚至设置画笔让其
绘制阴影(shadow layer)。

\subsection[Canvas]{Canvas}
\subsubsection[如影随形的Bitmap]{如影随形的Bitmap}
Canvas只是负责绘制,最终绘制结果要保存到Bitmap中,并将该bitmap交给Android
去呈现,所以Canvas必定有一个bitmap跟随,否则画了半天也白画。

\subsubsection[各种绘制操作]{各种绘制操作}
Canvas可以绘制点(point)、线(line)、圆(circle)、椭圆(oval)、矩形(rectangle)、
圆角矩形(round rect)、颜色(color)、弧形(arc)、bitmap、文本(text)。

\subsubsection[图像变换]{图像变换}
Canvas上当然可以绘制多个图形,每个图形都有不同的位置,大小,倾斜角度等,
这需要Canvas具有

变换(translate, rotate, skew, scale)

\subsubsection[画布裁剪]{画布裁剪}
裁剪(clip)

\subsubsection[其他]{其他}
save() \& restore

\subsection[Paint]{Paint}

字体相关操作

样式(stroke, fill, stroke and fill)

颜色(着色器)

笔画的宽度(stroke width)

