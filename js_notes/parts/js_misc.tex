\section[JavaScript里的对象]{JavaScript里的对象}
在JS里,对象其实就是Key-Value集合(键-值对集合)!
每一对儿K-V叫做一个“属性”,没有函数,因为函数也是一个属性。

\emph{一切皆对象,除了undefined!}

\begin{javascriptcode}
// toString()是Object.prototype对象(Object是一个构造函数)的一个函数(属性),
// JS的原型链从Object.prototype对象开始。

// boolean值是对象
true.toString();
false.toString();
// number是对象
(42).toString();
42..toString(); // 第一个“点”是小数点
'abc'.toString();
\end{javascriptcode}

\section[null v.s. undefined]{null v.s. undefined}
\begin{description}
  \item[null: ] 对象存在但是值为空(nothing),typeof(null)==='object'
  \item[undefined: ] 对象不存在,typeof(undefined)==='undefinded'
  \item[PK: ] null==undefinded, null!==undefinded
\end{description}

为何存在undefinded?

undefined表示“不存在”(nothing),在Java中,如果对象的某个属性不存在,
那么编译的时候就会报错,导致编译无法通过。但是JS并没有编译这一步,
它该如何表示“不存在”这个概念呢?当然用null来表示也是可以的,毕竟
“不存在”和“存在但没有任何值”几乎没有差别,但是JS更加细化了这一点,
用一个单独的undefined表示“不存在”,而Java由于存在编译期也就用不到
undefined这样的表述了。


\section[如何判断一个对象是否存在]{如何判断一个对象是否存在}
判断对象是否存在,

\begin{javascriptcode}
if(typeof(myObj)==='undefinded') {
  var myObj = { };
} else {
  // myObj存在(但是值可能为空)
}
\end{javascriptcode}

判断对象是否存在,并且是否为空,

\begin{javascriptcode}
if(!myObj) {
  // myObj === undefinded 或者 myObj === null
  var myObj = { };
} else {
  // myObj 存在且不为null
}
\end{javascriptcode}