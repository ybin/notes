\section[Prototype]{Prototype}
JavaScript的继承使用的是原型链(prototype chain)方式,原型链方式看似简单,
实则不然,理解原型链需从两方面进行:constructor模式、prototype模式。

\subsection[constructor mode]{constructor mode}
JavaScript 中没有类的概念,我们知道类是OOP中的"对象模型",
那JS中的对象模型又是什么呢?答案是“构造函数”。
在JS中,new关键词后面跟的便是构造函数。

另外,JS中的任何对象都有constructor属性,但是这个属性默认却并不表示
该对象的构造函数(对象),而是\emph{"构造函数".prototype.constructor},
如此一来,整个原型链中的对象的constructor都等于原型链顶端对象的constructor
属性,这样constructor便可以用来表明某个对象是隶属于哪个类别的了,
有点儿类似于Java中instanceof的意思。如果对象的构造函数没有prototype
属性,则该对象的constructor属性等于其构造函数(对象)。

\begin{javascriptcode}
// constructor是不是完成了类似Java instanceof的功能?!
var myObj = new Array();

if(myObj.constructor === Array) {
  document.write('This is an Array.');
} else if(myObj.constructor === Boolean) {
  document.write('This is a Boolean.');
} else if(myObj.constructor === Date) {
  document.write('This is a Date.');
} else if(myObj.constructor === String) {
  document.write('This is a String.');
}
\end{javascriptcode}

然而,当我们创建一个新类型的对象时,就需要手动设置constructor属性,
这通过设置新类型构造函数的prototype.constructor属性来完成,

\begin{javascriptcode}
// 我们要创建一个新的类型Person,类似于Array, Date, String等
function Person(name) {
  this.name = name;
}
Person.prototype.sayHello = function() {
  alert("I'm " + this.name);
}
// 重新设置constructor属性
Person.prototype.constructor = Person;

var peter = new Person(42);
peter.constructor === Person; // true
\end{javascriptcode}

\subsection[prototype mode]{prototype mode}
可以给构造函数增加"prototype"属性(函数也是对象哦!),其值为一个对象,以此来
达到对象的“继承”,这就是prototype mode。

\emph{prototype属性只在函数中起作用!}

\emph{构造函数prototype属性表示的对象是给其“实例”使用的!}

下面是实例解析,这三种继承方式其prototype chain是完全不同的。

\begin{javascriptcode}
  // Foo 构造函数
  function Foo() { foo1:1 };
  foo.prototype = { foo2:2 };

  // Bar 构造函数
  function Bar() { bar1:11 };

  // 这里填写继承关系,见下面三种继承方式

  var foo = Foo();
  var bar = Bar();
\end{javascriptcode}

\subsubsection[父子关系]{父子关系}
第一种继承方式:父子关系

\begin{javascriptcode}
  Bar.prototype = new Foo();
  // 如果要创建新类型,下面这行是必要的
  Bar.prototype.constructor = Bar;
\end{javascriptcode}

继承链:\\
bar\\
=> Foo的实例(即new Foo())\\
=> Foo.prototype对象\\
=> Object.prototype对象(即{},空对象)\\
foo与bar是父子关系。这是最常见的继承方式。

\subsubsection[兄弟关系]{兄弟关系}
第二种继承方式:兄弟关系

\begin{javascriptcode}
  Bar.prototype = Foo.prototype;
\end{javascriptcode}

继承链:\\
bar\\
=> Foo.prototype对象\\
=> Object.prototype对象(即{},空对象)\\
foo\\
=> Foo.prototype对象\\
=> Object.prototype对象(即{},空对象)\\
foo与bar是兄弟关系。

\subsubsection[陌生关系]{陌生关系}
第三种继承方式:陌生关系

\begin{javascriptcode}
  Bar.prototype = Foo;
\end{javascriptcode}

继承链:\\
bar\\
=> Foo对象(函数也是对象)\\
=> Foo.constructor.prototype(即Function.prototype)\\
foo\\
=> Foo.prototype\\
=> Object.prototype对象(即{},空对象)\\
Foo的实例与Bar的实例没有关系。


\section[变量声明、函数声明提升]{变量声明、函数声明提升}
名词解释:\\
声明(Declaration):\\
\begin{javascriptcode}
  var joe;
\end{javascriptcode}

初始化(Initialization):\\
\begin{javascriptcode}
  joe = 'plumber';
\end{javascriptcode}

JS代码的解析分为两部分:预编译和解释执行。
\begin{description}
  \item[预编译阶段]:\\
    对var定义的变量(包括全局、局部变量)进行声明,初始值设置为undefined;\\
    对定义式的函数(即:function myFunc() \{\})进行定义(这样函数就可用了)。
  \item[解释执行阶段]:\\
    对var定义的变量执行赋值操作(如果有的话);
    所有语句顺序执行。
\end{description}

示例:\\
\begin{javascriptcode}
  function f() {
    var a = 'a';
    // other code
    var b = 'b';
    // other code
    c();
    function c() {
      var aa = 'aa';
      // another code
      var bb = 'bb';
      // another code
    }
    // other code
  }
\end{javascriptcode}

等价于,

\begin{javascriptcode}
  function f() {
    var a, b; // 变量声明
    function c() {
      var aa, bb; // 变量声明
      aa = 'aa';
      // another code
      bb = 'bb';
      // another code
    }
    a = 'a';
    // other code
    b = 'b';
    // other code
    c();
    // other code
  }
\end{javascriptcode}

测试:alert输出时什么呢?

\begin{javascriptcode}
  var myvar = 'my value';
  (function () {
    alert(myvar); // 这个myvar是局部变量
    var myvar = 'local value';
    alert(myvar); // 输出:local value
  })();
\end{javascriptcode}

答案:undefined

函数声明也是一样的,所以可以在定义函数之前就使用函数,
因为“使用”是在解释执行阶段,在此之前函数已经定义完成了!

这也就是以下两种定义函数的方式之差别,

\begin{javascriptcode}
func_1(); // 报错!因为此时func_1 === undefined
func_2(); // 正常!因为此时func_2的定义已经完成了

// func_1的声明被提升到预编译阶段,但是赋值却是在解释执行阶段
var func_1 = function() { };
// 函数定义在预编译阶段就完成了
function func_2() { }
\end{javascriptcode}

注意,预编译、解释执行两个阶段针对每个JS code block进行一次,
即第一个code block预编译然后解释执行,第二个code block预编译
然后解释执行,依次顺序遍历所有code block。JS code放到HTML文件
中和通过src属性引入外部js文件没有任何区别。
