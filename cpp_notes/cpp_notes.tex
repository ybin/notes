
\documentclass[a4paper,11pt]{article}


%%% fontenc
%\usepackage{fontspec,xunicode,xltxtra}
%\setmainfont{Times New Roman}
%\setsansfont{Source Sans Pro}
%\setmonofont{Source Sans Pro}

%%% xeCJK
\usepackage{xeCJK}
\setCJKmainfont[BoldFont=Adobe Heiti Std]{Adobe Song Std}
\setCJKsansfont[BoldFont=Adobe Heiti Std]{Adobe Song Std}
\setCJKmonofont[BoldFont=Adobe Heiti Std]{Adobe Song Std}
\XeTeXlinebreaklocale "zh"
\XeTeXlinebreakskip=0pt plus 1pt minus 0.1pt

\usepackage{xcolor}
\usepackage{graphicx}

%%% get total page number
\usepackage{lastpage}

%%% customized definition
\makeatletter
\def\sybtitle#1{\def\@sybtitle{#1}}
\def\sybauthor#1{\def\@sybauthor{#1}}
\def\sybdate#1{\def\@sybdate{#1}}
\sybtitle{}
\sybauthor{}
\sybdate{}
\def\sybmaketitle{
  \begin{center}
  \vspace*{.8in}
  {\huge\bfseries\@sybtitle}
  \par
  \vspace{.8in}
  {\Large\@sybauthor}
  \par
  \vspace{.2in}
  \@sybdate
  \vspace{.5in}
  \end{center}
}
\makeatother
\setlength{\parindent}{0pt}
\renewcommand{\today}{\number\month 月 \number\day 日, ~\number\year 年}
\def\lt{\textless}
\def\gt{\textgreater}
\renewcommand\contentsname{\bfseries 目~~录}
\newcommand\bs{\texttt{\symbol{'134}}} % input backslash sign
%\newcommand\bs{\string\} % same as above definition
\long\def\cmd#1{\par\vspace{.5em}\hspace*{2em}#1\vspace{.5em}\par}
\def\cstr#1{\texttt{\string#1}} % e.g. \cstr{\latex}
\long\def\runcode#1{\par\bigskip#1\bigskip\par}
% 我不想看到那么多的underful hbox,尤其是minted环境加上背景色之后
\hbadness=10000
% 适当放宽overful hbox的限制,运行2pt的溢出
\hfuzz=2pt
\parskip=3\lineskip


%%% change background color & add frame for enumerate enviroment
\usepackage{mdframed}
\newmdenv[backgroundcolor=blue!10,linewidth=0pt]{coloredframe}
\newenvironment{coloredenumerate}{
  \begin{coloredframe}
  \begin{enumerate}
}{
  \end{enumerate}
  \end{coloredframe}
}

%%% geometry
\usepackage[includehead,includefoot,hmargin=21mm,vmargin=10.5mm,
            headsep=12pt,headheight=25pt]{geometry}
%\usepackage[includehead,includefoot,hmargin=1.2in,vmargin=1in]{geometry}

%%% fancyhdr
\usepackage{fancyhdr}
\makeatletter
\fancypagestyle{main} {
  \fancyhf{} % clear header & footer
  \fancyhead[L]{\bfseries\@sybtitle}
  \fancyhead[R]{\thepage/\pageref*{LastPage}}
  \renewcommand{\headrulewidth}{0.4pt} % header line
  \renewcommand{\footrulewidth}{0pt} % footer line
}
\fancypagestyle{header} {
  \fancyhf{} % clear header & footer
  \fancyfoot[C]{\roman{page}}
  \renewcommand{\headrulewidth}{0pt} % header line
  \renewcommand{\footrulewidth}{0pt} % footer line
}
\makeatother

\usepackage{titlesec}
\titleformat{\part}{\centering\Large\bfseries}{第\,\thepart\,部分}{1em}{}
\titleformat{\section}{\large\bfseries}{\thesection}{1em}{}
\titleformat{\subsection}{\normalsize\bfseries}{\thesubsection}{1em}{}
%\titlespacing*{章节命令}{左边距}{上文距}{下文距}[右边距]
\titlespacing*{\section}{0pt}{2\baselineskip}{\parsep}


\usepackage{hyperref}

%%% perfect source code display
\usepackage{minted}
%\usemintedstyle{colorful}
\definecolor{srcbg}{rgb}{0.95,0.95,0.95}
\newminted{java}{linenos,tabsize=4,bgcolor=srcbg}
\newminted{xml}{linenos,tabsize=4,bgcolor=srcbg}
\newminted{cpp}{linenos,tabsize=4,bgcolor=srcbg}
\newminted{bash}{linenos,tabsize=4,bgcolor=srcbg}
\newminted{latex}{linenos,tabsize=4,bgcolor=srcbg}
\newminted{scheme}{linenos,tabsize=4,bgcolor=srcbg}

\usepackage{amsmath}


\input{../styles/tikz_preamble}

\sybtitle{CPP Notes}
\sybauthor{孙延宾}
\sybdate{\today}

\begin{document}
\tt % I love Typewriter font.
%%%%%%%% the title page and toc %%%%%%%%%%
\pagestyle{header}
\sybmaketitle
\tableofcontents
\newpage

%%%%%%% the main content %%%%%%%%%
\pagestyle{main}
\setcounter{page}{1}

\part[Basic Usage]{Basic Usage}
\section[When destroy object]{When object is destroied}
一句话:Stack上的object自动销毁,Heap上的object手动销毁。

实例:
首先创建一个用于演示的类,\\
\begin{cppcode}
class Test {
public:
    Test() {
      cout<<"Test::constructor"<<endl;
    }

    ~Test() {
      cout<<"Test::destructor"<<endl;
    }

private:
    int a, b; // 占空用的
};
\end{cppcode}

然后分别在stack和heap上创建该类的对象,

\begin{cppcode}
void test() {
    Test t; // object is in stack
    Test tt = new Test(); // object is in heap
}
\end{cppcode}

调用test()函数打印结果,

\begin{bashcode}
Test::constructor # constructor of t
Test::destructor # destructor of tt, tt do NOT ness constructor
Test::destructor  # destructor of t
\end{bashcode}

heap上的对象tt不会自动销毁,内存泄露发生。

\section[variable, reference and pointer]{variable, reference and pointer}
variable特指stack上的variable,heap上的variable都是指针。

一句话解释:\\
variable赋值时分配内存,不论是基本类型还是对象;\\
reference只是起了个“别名”,不分配任何内存;\\
pointer分配size(*void)的内存,一般为4字节,真正的内存需额外分配。

实例(沿用上一节中的Test类):

\begin{cppcode}
void print_memory_addr() {
    Test t;   // 在stack上创建一个Test对象
    Test t0;  // 在stack上创建另一个Test对象,使用对象t初始化
    Test& tt = t; // 在stack上创建对象t的引用tt
    Test* ttt = new Test(); // 在heap上创建对象,在stack上创建指针

    cout<<(long)&t<<endl;
    cout<<(long)&t0<<endl;
    cout<<(long)&ttt<<endl;
    cout<<(long)&tt<<endl;
}
\end{cppcode}

打印结果,

\begin{bashcode}
af865990
af865980
af865978
af865990
\end{bashcode}

90-80共16个字节存放object t(为什么是16个字节?这是gcc的优化);\\
80-78共8个字节放置object t0;\\
78开始的4个字节放置pointer ttt;\\
有意思的是reference tt,它直接被翻译成object t的内址,
即reference在内存中是“不占用内存”的,它只能被编译器识别,
类似于Clojure里的reader macro,不是吗!

\section[Concurrency]{Concurrency}
在Linux上多线程资源同步的方法大致有如下三个,它们也是C/C++世界中
的同步方法,现在通过与Java对比,分别进行介绍。

\subsection[互斥锁]{互斥锁}
Mutex跟Java中的synchronized类似,用法也一致。

\begin{cppcode}
  mutex _lock;
  int count = 0;
  mutex_lock(_lock);
  count++;
  mutex_unlock(_lock);
\end{cppcode}

注意:cpp中利用对象的构造函数、析构函数,配合mutex,又创建出
AutoMutex,它只能用于cpp中。

另外,跟synchronized类似,这需要程序员自觉,如果程序员不手动
使用mutex,而是直接count++,那么mutex就被绕过去了。

\subsection[条件变量]{条件变量}
条件变量需要配合mutex使用,它类似于Java中的wait(), notify()/notifyAll()
机制,使用方法也是类似的。

\begin{cppcode}
  // 100个线程同时工作,取出前十个完工的线程
  mutex_lock(_lock);
  // 有一个线程完成了,全局变量增加1
  num = num + 1;
  if(num <= 10) {
    // 如果完成的线程数量不到10个就等待
    cond_wait(_lock, cond);
    printf("Top 10 threads have finished...");
  } else if(num = 11) {
    // 完成线程数已经达到10个了,通知10个等待的线程
    cond_broadcast(_lock, cond);
  }
  mutex_unlock(_lock);
\end{cppcode}

跟Java里的用法简直如出一辙。

\subsection[读写锁]{读写锁}
读取锁有三种状态:未锁状态(unlock)、共享读取状态(shared-read lock)、互斥写入状态(exclusive-write lock)

\begin{itemize}
\item unlock状态的lock可以被线程锁定为共享读取状态或者互斥写入状态
\item 共享读取状态的lock可以被其他线程继续锁定为共享读取状态
\item 互斥写入状态的lock,其他线程既不能锁定为Read锁也不能锁定为Write锁
\end{itemize}

这样,多个线程就可以共享资源的读取,而当有写入操作时,各个线程就转换为互斥,
互斥写入也是使用条件变量完成的。

互斥锁=>条件变量=>读写锁,层层深入,一个比一个精妙。

\end{document}