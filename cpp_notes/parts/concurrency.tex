\section[Concurrency]{Concurrency}
在Linux上多线程资源同步的方法大致有如下三个,它们也是C/C++世界中
的同步方法,现在通过与Java对比,分别进行介绍。

\subsection[互斥锁]{互斥锁}
Mutex跟Java中的synchronized类似,用法也一致。

\begin{cppcode}
  mutex _lock;
  int count = 0;
  mutex_lock(_lock);
  count++;
  mutex_unlock(_lock);
\end{cppcode}

注意:cpp中利用对象的构造函数、析构函数,配合mutex,又创建出
AutoMutex,它只能用于cpp中。

另外,跟synchronized类似,这需要程序员自觉,如果程序员不手动
使用mutex,而是直接count++,那么mutex就被绕过去了。

\subsection[条件变量]{条件变量}
条件变量需要配合mutex使用,它类似于Java中的wait(), notify()/notifyAll()
机制,使用方法也是类似的。

\begin{cppcode}
  // 100个线程同时工作,取出前十个完工的线程
  mutex_lock(_lock);
  // 有一个线程完成了,全局变量增加1
  num = num + 1;
  if(num <= 10) {
    // 如果完成的线程数量不到10个就等待
    cond_wait(_lock, cond);
    printf("Top 10 threads have finished...");
  } else if(num = 11) {
    // 完成线程数已经达到10个了,通知10个等待的线程
    cond_broadcast(_lock, cond);
  }
  mutex_unlock(_lock);
\end{cppcode}

跟Java里的用法简直如出一辙。

\subsection[读写锁]{读写锁}
读取锁有三种状态:未锁状态(unlock)、共享读取状态(shared-read lock)、互斥写入状态(exclusive-write lock)

\begin{itemize}
\item unlock状态的lock可以被线程锁定为共享读取状态或者互斥写入状态
\item 共享读取状态的lock可以被其他线程继续锁定为共享读取状态
\item 互斥写入状态的lock,其他线程既不能锁定为Read锁也不能锁定为Write锁
\end{itemize}

这样,多个线程就可以共享资源的读取,而当有写入操作时,各个线程就转换为互斥,
互斥写入也是使用条件变量完成的。

互斥锁=>条件变量=>读写锁,层层深入,一个比一个精妙。