\section[Notes of "Expert C Programming"]{Notes of "Expert C Programming"}

\subsection[函数原型]{函数原型}
函数原型是ANSI对函数声明的扩展,对比函数声明和函数原型的差别,

\begin{cppcode}
// K&R 函数声明
char *strcpy();

// ANSI 函数原型
char *strcpy(char *dst, const char *src);
\end{cppcode}

但是一般习惯用函数声明指代函数原型,此外,ANSI C的函数原型用于编译器在
\emph{编译期}对函数调用中的\emph{实参}和
函数原型中的\emph{形参}进行参数一致性检查,而K\&R C中类似的检查(返回值类型检查?)
被推迟到了链接时,甚至根本不做检查。

编译期,函数声明、函数定义、函数调用,三者必须满足:
\begin{enumerate}
  \item 函数声明与函数定义要相容
  \item 函数声明与函数调用要相容
\end{enumerate}
“相容”指的是“不产生抵触”,如参数的类型和个数一致,但是函数定义和函数调用之间
并不一定相容,这一点编译器不作检查,比如下面这个特例,

\begin{cppcode}
// test.h
int add(); // add()接收任意参数
// test_impl.c
int add(int a, int b) { return a+b; }
// test_main.c
int main()
{
  printf(" %d\n", add(1, 2, 4)); // result: 1 + 2 = 3, 4 is ignored!
}
\end{cppcode}

作为函数参数使用时,空括号("()")表示接受任意参数,而void("(void)")才
表示不接受任何参数。函数声明和函数定义之间、函数声明和函数调用之间都满足
相容性,编译没有问题,但是实际运行确实大相径庭。