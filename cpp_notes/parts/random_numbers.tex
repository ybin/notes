\section[Generate random numbers in C/C++]{Generate random numbers in C/C++}
在C/C++中生成随机数要用到stdlib.h中的rand()函数和srand()函数,
rand()函数根据seed值产生一个0..RAND\_MAX(32767)之间的值(包括端点),
但是由于rand()的seed值是固定的,所以每次运行程序,rand()产生的随机数
是相同的。为此,我们每次运行程序时需要设置一个不同的seed值,比如
pid或者运行时间,此时就需要调用srand(unsigned in seed)函数,示例如下,

\begin{cppcode}
#include <stdio.h>
#include <stdlib.h> // rand(), srand()
#include <time.h>   // time()

void rand_test() {
  int i;
  for(i = 0; i < 10; i++)
    printf(" %d ", rand());
  putchar('\n');
}

void srand_test() {
  int i;
  srand((unsigned)time(NULL));
  for(i = 0; i < 10; i++)
    printf(" %d ", rand());
  putchar('\n');
}

int main()
{
    rand_test(); // 每次运行程序,结果都一样
    srand_test(); // 每次运行,结果均不一致
    return 0;
}
\end{cppcode}

示例输出,

\begin{bashcode}
# 第一次运行结果
 41  18467  6334  26500  19169  15724  11478  29358  26962  24464
 24670  1024  18283  3498  30494  32031  32564  27417  1451  15655
# 第二次运行结果
 41  18467  6334  26500  19169  15724  11478  29358  26962  24464
 24673  11772  3379  27561  8040  3189  2909  24490  17962  9506
\end{bashcode}