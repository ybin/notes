\section[Variable declaration(another way)]{另一种C语言变量声明语法的理解}
前面讲了right-left规则,非常棒的规则,下面再介绍一种理解
变量声明(尤其是含指针的变量声明)的方法,这个方法是Brian Bi大神
在Quora上的回答,\href{http://www.quora.com/C-programming-language/Why-doesnt-C-use-better-notation-for-pointers/answer/Brian-Bi}{参考网址}。

该方法的核心是“声明记录使用”(Declaration follows use)。
这句话的意思是说:把变量的“使用过程”记录下来就是“声明”。

如果说right-left法则是为了“读”声明语句,那么这里介绍的方法
则主要是为了“写”声明语句,该方法要求首先要明确声明的变量
要如何使用,然后把使用过程记录下来,自然而然就形成了变量声明。

举例来说,

\begin{cppcode}
  int *p;
\end{cppcode}

我们如何使用变量p呢?\\
“我们将p解引用,然后就得到一个int值”,这就是我们使用p的过程。

又如,

\begin{cppcode}
  int *p[5];
\end{cppcode}

变量p的使用过程如下,
\begin{enumerate}
  \item 使用[]操作得到数组的元素(p[i])
  \item 每个元素都可以解引用(*p[i])
  \item 每个元素解引用的结果都是int类型的值
\end{enumerate}
为何先使用[]操作(indexing)而不是先解引用呢?因为优先级不同。

再如,

\begin{cppcode}
  int (*p)[5];
\end{cppcode}

变量p的使用过程如下,
\begin{enumerate}
  \item 解引用p(*p)得到一个数组
  \item 使用[]操作得到数组的元素
  \item 结果每个元素都是int类型
\end{enumerate}
因为优先级的原因,要保证使用过程中步骤的先后顺序,有时需要添加括号。

方法介绍到这里,下面是使用过程。
举个例子,我们想定义一个变量,它是一个指向函数的指针,这个函数返回
另一个函数指针,被指向的函数返回一个整型指针。

我们一边使用该变量一边记录下使用过程,
\begin{enumerate}
  \item 变量命名为p,因为它是一个指针,所以使用时需要解引用 => *p
  \item 解引用得到一个函数,我们调用该函数,注意优先级,使用括号保证顺序 => (*p)()
  \item 函数执行后返回一个指针,我们解引用该指针 => *(*p)()
  \item 解引用得到一个函数,我们调用该函数,注意优先级,使用括号保证顺序 => (*(*p)())()
  \item 函数执行后返回一个指针,我们解引用该指针 => *(*(*p)())()
  \item 解引用的结果是得到一个整型值 => int *(*(*p)())()
\end{enumerate}
接下来我们可以使用right-left法则检验该声明是否合法,完全正确。