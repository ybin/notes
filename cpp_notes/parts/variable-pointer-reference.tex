\section[variable, reference and pointer]{variable, reference and pointer}
variable特指stack上的variable,heap上的variable都是指针。

一句话解释:\\
variable赋值时分配内存,不论是基本类型还是对象;\\
reference只是起了个“别名”,不分配任何内存;\\
pointer分配size(*void)的内存,一般为4字节,真正的内存需额外分配。

实例(沿用上一节中的Test类):

\begin{cppcode}
void print_memory_addr() {
    Test t;   // 在stack上创建一个Test对象
    Test t0;  // 在stack上创建另一个Test对象,使用对象t初始化
    Test& tt = t; // 在stack上创建对象t的引用tt
    Test* ttt = new Test(); // 在heap上创建对象,在stack上创建指针

    cout<<(long)&t<<endl;
    cout<<(long)&t0<<endl;
    cout<<(long)&ttt<<endl;
    cout<<(long)&tt<<endl;
}
\end{cppcode}

打印结果,

\begin{bashcode}
af865990
af865980
af865978
af865990
\end{bashcode}

90-80共16个字节存放object t(为什么是16个字节?这是gcc的优化);\\
80-78共8个字节放置object t0;\\
78开始的4个字节放置pointer ttt;\\
有意思的是reference tt,它直接被翻译成object t的内址,
即reference在内存中是“不占用内存”的,它只能被编译器识别,
类似于Clojure里的reader macro,不是吗!