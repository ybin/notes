\section[最短的崩溃C程序]{最短的崩溃C程序}
学习如何让一个C程序崩溃并揭示出崩溃原因,是一种学习C语言的
好方法,因为C语言实在有太多坑了,一旦你把大部分坑都踩过了,
以后走路就会聪明很多。

本节我们试图用C语言编写一个最短的崩溃程序,
\href{http://blog.jobbole.com/40286/}{参考网页}(\href{http://llbit.se/?p=1744}{英文网页})。

首先想到的当然是“除数为0”这一条,

\begin{cppcode}
int main()
{
  return 1/0;
}
\end{cppcode}

还不够短小,"return"可以省略,但是这样一来"1/0;"其实没有任何用处,
GCC会自动优化,把"1/0;"这个表达式去掉,我们要把这个表达式利用起来才行,

\begin{cppcode}
/* 变量声明都有隐形的int类型,返回值也是如此!但是局部变量不可! */
// int i;
i;
/*int */main()
{
  i=1/0;
}
\end{cppcode}

去掉"return"、"int"共8个字符,又增加了4个字符,总体减少了,有进步。
到此为止,我们利用了C的一个坑---变量声明、函数返回值默认为int类型。

接下来就需要深入一点儿了,编译器在编译时不关心类型,它不管某个符号
是变量还是函数,比如说main这个符号,它只管傻乎乎的进行编译;链接器
也不关注符号的类型,它只关心符号的地址,于是我们有了下面的程序,

\begin{cppcode}
int main=0;
\end{cppcode}

运行出现段错误,因为编译器把main放到了.data段,而这个段是不可执行的。
所以给main赋什么值都无所谓,不赋值当然也OK,进一步精简,

\begin{cppcode}
main;
\end{cppcode}

这里再次利用默认类型这一特性。GCC编译、链接都没有问题,不过-Wall参数
会给出一些警告,运行当然是报错啦。。。