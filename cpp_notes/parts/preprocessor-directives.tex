\section[Preprocessor directives]{Preprocessor directives}
本节参考\href{http://www.cplusplus.com/doc/tutorial/preprocessor/}{网址}。

\subsection[\#define,\#undef]{Macro definitions}
宏定义预处理指令包括\#define和\#undef,语法格式:

\begin{cppcode}
#define identifier replacement
// identifier在此期间有效
#undef identifier
\end{cppcode}

\#define支持两个操作符(operator):\#和\#\#。

\subsubsection[operator '\#']{operator '\#'}
在\#define的replacement部分,如果参数前面有\#,则参数会被替换为
字符串(string literal,相当于用双括号括起来)

\begin{cppcode}
#define str(x) #x
cout << str(test);
// 将会被翻译为
cout << "test";
\end{cppcode}

\subsubsection[operator '\#\#']{operator '\#\#'}
\#\#操作符将会组合两个参数,并且去掉它们之间的空格。

\begin{cppcode}
#define glue(a, b) a ## b
glue(c, out) << "test";
// 将会翻译为
cout << "test";
\end{cppcode}

\subsection[\#ifdef,\#ifndef,\#if,\#endif,\#else and \#elif]{Conditional inclusions}
条件判断预处理指令包括:\#ifdef,\#ifndef,\#if,\#endif,\#else 和 \#elif,望文生义即可。

\begin{cppcode}
// #if的操作数必须是编译之前就可以确定的值
#if TABLE_SIZE>200
#undef TABLE_SIZE
#define TABLE_SIZE 200

#elif TABLE_SIZE<50
#undef TABLE_SIZE
#define TABLE_SIZE 50

#else
#undef TABLE_SIZE
#define TABLE_SIZE 100
#endif

int table[TABLE_SIZE];
\end{cppcode}

\subsection[\#error]{Error directive}
\#error预处理指令用于在编译时抛出错误信息并终止编译,

\begin{cppcode}
#ifndef __cplusplus
#error A C++ compiler is required!
#endif
\end{cppcode}

\subsection[\#include]{Source file inclusion}

\subsection[\#pragma]{Pragma directive}
\#pragma指令用于给编译器提供选项,具体选项在编译器之间各不相同,在此不详述。

\subsection[Predefined macros]{Predefined macros}
预定义指令都以“双下划线”开始和结束,包括两部分,
\begin{enumerate}
  \item 各个编译器都支持的指令
  \begin{description}
    \item[\_\_LINE\_\_] 整数值,替换为当前行号
    \item[\_\_FILE\_\_] 字符串,替换为当前文件名
    \item[\_\_DATE\_\_] 字符串,替换为当前日期,"Mmm dd yyyy"格式
    \item[\_\_TIME\_\_] 字符串,替换为当前时间
    \item[\_\_cplusplus\_\_] 整数值,C++编译器会定义该宏,不同编译器定义的值可能不同
    \item[\_\_STD\_HOSTED\_\_] 1如果当前编译器是hosted implementation(包含标准头文件),否则为0
  \end{description}
  \item 编译器选择性支持的指令
  \begin{itemize}
    \item 略(详见\href{http://www.cplusplus.com/doc/tutorial/preprocessor/}{参考网址})。
  \end{itemize}
\end{enumerate}

\subsection[Why use 'do \{\} while(0)']{宏定义中为什么使用'do \{\} while(0)'}
使用\#define定义带参数的宏时,经常使用'do \{\} while(0)'这样的格式,

\begin{cppcode}
#define foo(x) \
  do { \
    printf("msg: ", #x); \
    /* 其他复杂的代码 */ \
  } while(0) // 注意这里没有分号

if(/*condition true*/)
  foo(OK); // 注意这里的分号
else
  // other operations
\end{cppcode}

这是因为do-while是一个“独立的语法单元”,它的作用类似于用括号把参数括起来,
但是为何不用\{\}把复杂的代码括起来呢?试试用大括号替换上面的do-while,
会报错,因为foo(OK)后面的分号(;)。

一般编译器都会优化do-while(0),所以不存在性能问题。

另外,注意宏定义中添加注释时使用多行注释而不是单行注释。

\vspace{0.5cm}

除此之外,do-while(0)还有另外一个用处,那就是避免使用goto语句,

\begin{cppcode}
// return value: if not 0 then success else fail
int manipulate() {
  // allocate resource
  int *p = (int*)malloc(1024, sizeof(int));
  int ret = 0;
  
  // any step failure will result to whole failure
  do {
    ret = do_step_1(p);
    if(!ret) break;
    
    ret = do_step_2(p);
    if(!ret) break;
    
    ret = do_step_3(p);
    if(!ret) break;
  } while(0);
  
  // free resource
  free(p);
  p = NULL;
  return ret;
}
\end{cppcode}
