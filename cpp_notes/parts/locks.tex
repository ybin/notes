\section[locks in kernel]{内核中的锁}
Kernel中主要有两种锁:
\begin{enumerate}
  \item 自旋锁(spinlock)
  \item 信号量(semaphore)
\end{enumerate}

\subsection[定义]{定义}
\begin{description}
  \item[spinlock: ] 自旋锁是专为防止“多处理器并发”而引入的一种锁。自旋锁最多只能被一个内核任务持有,如果一个内核任务视图请求
  一个已被持有的自旋锁,那么这个任务就会进入busy-wait loop,直到拿到锁为止。自旋锁忙等待会100\%占用CPU,所以它只适合于多处理器
  场景,显然它最好用于等待时间非常短的场景下。
  \item[semaphore: ] 信号量S是一个整数,S大于等于零表示可供使用的进程数目,S小于零表示正在等待使用共享资源的进程数目。
  \begin{itemize}
    \item P操作申请资源:
    \begin{description}
      \item[(1)] S减1
      \item[(2)] 若S减1后大于0,则进程继续执行
      \item[(3)] 如S减1后小于0,则进程被阻塞并放入该信号量对应的队列中,等待进程调度
    \end{description}
    \item V操作释放资源:
    \begin{description}
      \item[(1)] S加1
      \item[(2)] 若S加1后大于0,则进程继续执行
      \item[(3)] 若S加1后小于0,则从该信号量对应的队列中唤醒一个等待进程
    \end{description}
  \end{itemize}
  \item[mutex: ] 互斥锁可以视为S等于1的semaphore。
  \item[adaptive lock: ] 自适应锁是spinlock和semaphore的结合,它根据持有锁的进程的
  方面程度决定等待锁的进程是busy-wait还是进入sleep。
\end{description}
为何叫PV操作呢?\\
原因是信号量和PV操作都是荷兰计算机科学家Dijkstra发明的,
在荷兰语中,“通过”叫passeren,“释放”叫vrijgeven,PV操作因此得名。
这是在计算机术语中不是用英语表达的极少数的例子之一。

\subsection[对比]{对比}
\begin{itemize}
  \item spinlock是非睡眠的,它会一直循环等待,直到拿到锁,因此持有锁的进程禁止睡眠。
  \item semaphore是睡眠锁,如果进程申请锁失败就会被阻塞并放到该锁的等待队列中。
\end{itemize}
