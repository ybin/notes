\section[Pointer vs Array]{Pointer vs Array}
本节实例源自该\href{http://coolshell.cn/articles/5761.html}{网址}。

\begin{cppcode}
#include <stdio.h>
int main(void)
{
  int a[5];  // 假设a的内存地址是:0xBFE2E100
  printf("%x\n", a);  // output: 0xBFE2E100, 类似于int *a;
  printf("%x\n", a+1);  // output: 0xBFE2E100 + sizeof(int)
  printf("%x\n", &a);  // output: 0xBFE2E100, 类似于int (*a)[]; 
  printf("%x\n", &a+1);  // output: 0xBFE2E100 + 5*sizeof(int)
}
\end{cppcode}

一般来说,数组等同于指针,但是严格来说,他们有很多差别,
前两个printf中,a的行为跟整型指针很相似,可以说是一样的。
但是数组就是数组,它不是指针,\&操作下这种差别就很明显了,

\begin{cppcode}
int *p;
int a[5];
/*
 * a被翻译为int *a; 指向int的指针
 * &a被翻译为int (*a)[]; 指向数组的指针
 * &p指针的内存地址
 * &&a被翻译为int (*a)[][]; 指向二维数组的指针
 * &&p不存在,因为p是一级指针
 */
\end{cppcode}

需要注意的是\&a和\&\&a仍然是指针,只是其指向的“内容的粒度”不同而已,
这一点从a+1(跨过第二个维度的一个元素,即第一个维度的长度)就能反应出来,
不过只是粒度不同,内存地址仍然是一样的。

区分以下三中情形,

\begin{cppcode}
int *p;
// a POINTER to int
int (*p)[];
// a POINTER to an array whose elements are int
int* p[];
// an ARRAY whose elements are pointers to int
\end{cppcode}