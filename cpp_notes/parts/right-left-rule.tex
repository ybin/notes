\section[Right-Left rule]{Right-Left rule}
本节参考\href{http://ieng9.ucsd.edu/~cs30x/rt_lt.rule.html}{网址},
也可以阅读right-left-rule.txt文件,如果你能看到这篇文档的源码的话。

C语言中的“右-左法则”(不是“左-右法则“),用于解码各种声明(declarations),
包括变量声明、函数声明等等,这是一个通用法则,可以放心使用。

首先该法则需要一个解码表,用于把符号解码为英语,
\begin{description}
  \item[{* }:] 读作"pointer to" - 总是在左侧
  \item[{[~]}:] 读作"array of" - 总是在右侧
  \item[{(~)}:] 读作"function returning" - 总是在右侧
\end{description}

有了解码表,下面就是具体操作步骤,一步一步将C语言声明解码为可以
理解的英语语句。

注意顺序:先往右,再往左,并且走过的就不要再走了,
\begin{description}
  \item[Step 1] 找到你的identifier\\
  这是起点,然后翻译为"xxx is",xxx为identifier的名字
  \item[Step 2] 往右走,直到尽头或者遇到")"
  \item[Step 3] 往左走,直到尽头或者遇到"("
\end{description}
重复Step 2和Step 3,直到解析完成。

下面是具体实例,

\hspace{1cm}\textcolor{red}{int *p[];}

\begin{enumerate}
  \item int *\textcolor{red}{p}[];\\
  "p is"
  \item int *p\textcolor{red}{[]};\\
  "p is array of"
  \item int \textcolor{red}{*}p[];\\
  "p is array of pointer to"
  \item \textcolor{red}{int} *p[];\\
  "p is array of pointer to int"
\end{enumerate}
