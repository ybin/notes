
\documentclass[a4paper,11pt]{article}


%%% fontenc
%\usepackage{fontspec,xunicode,xltxtra}
%\setmainfont{Times New Roman}
%\setsansfont{Source Sans Pro}
%\setmonofont{Source Sans Pro}

%%% xeCJK
\usepackage{xeCJK}
\setCJKmainfont[BoldFont=Adobe Heiti Std]{Adobe Song Std}
\setCJKsansfont[BoldFont=Adobe Heiti Std]{Adobe Song Std}
\setCJKmonofont[BoldFont=Adobe Heiti Std]{Adobe Song Std}
\XeTeXlinebreaklocale "zh"
\XeTeXlinebreakskip=0pt plus 1pt minus 0.1pt

\usepackage{xcolor}
\usepackage{graphicx}

%%% get total page number
\usepackage{lastpage}

%%% customized definition
\makeatletter
\def\sybtitle#1{\def\@sybtitle{#1}}
\def\sybauthor#1{\def\@sybauthor{#1}}
\def\sybdate#1{\def\@sybdate{#1}}
\sybtitle{}
\sybauthor{}
\sybdate{}
\def\sybmaketitle{
  \begin{center}
  \vspace*{.8in}
  {\huge\bfseries\@sybtitle}
  \par
  \vspace{.8in}
  {\Large\@sybauthor}
  \par
  \vspace{.2in}
  \@sybdate
  \vspace{.5in}
  \end{center}
}
\makeatother
\setlength{\parindent}{0pt}
\renewcommand{\today}{\number\month 月 \number\day 日, ~\number\year 年}
\def\lt{\textless}
\def\gt{\textgreater}
\renewcommand\contentsname{\bfseries 目~~录}
\newcommand\bs{\texttt{\symbol{'134}}} % input backslash sign
%\newcommand\bs{\string\} % same as above definition
\long\def\cmd#1{\par\vspace{.5em}\hspace*{2em}#1\vspace{.5em}\par}
\def\cstr#1{\texttt{\string#1}} % e.g. \cstr{\latex}
\long\def\runcode#1{\par\bigskip#1\bigskip\par}
% 我不想看到那么多的underful hbox,尤其是minted环境加上背景色之后
\hbadness=10000
% 适当放宽overful hbox的限制,运行2pt的溢出
\hfuzz=2pt
\parskip=3\lineskip


%%% change background color & add frame for enumerate enviroment
\usepackage{mdframed}
\newmdenv[backgroundcolor=blue!10,linewidth=0pt]{coloredframe}
\newenvironment{coloredenumerate}{
  \begin{coloredframe}
  \begin{enumerate}
}{
  \end{enumerate}
  \end{coloredframe}
}

%%% geometry
\usepackage[includehead,includefoot,hmargin=21mm,vmargin=10.5mm,
            headsep=12pt,headheight=25pt]{geometry}
%\usepackage[includehead,includefoot,hmargin=1.2in,vmargin=1in]{geometry}

%%% fancyhdr
\usepackage{fancyhdr}
\makeatletter
\fancypagestyle{main} {
  \fancyhf{} % clear header & footer
  \fancyhead[L]{\bfseries\@sybtitle}
  \fancyhead[R]{\thepage/\pageref*{LastPage}}
  \renewcommand{\headrulewidth}{0.4pt} % header line
  \renewcommand{\footrulewidth}{0pt} % footer line
}
\fancypagestyle{header} {
  \fancyhf{} % clear header & footer
  \fancyfoot[C]{\roman{page}}
  \renewcommand{\headrulewidth}{0pt} % header line
  \renewcommand{\footrulewidth}{0pt} % footer line
}
\makeatother

\usepackage{titlesec}
\titleformat{\part}{\centering\Large\bfseries}{第\,\thepart\,部分}{1em}{}
\titleformat{\section}{\large\bfseries}{\thesection}{1em}{}
\titleformat{\subsection}{\normalsize\bfseries}{\thesubsection}{1em}{}
%\titlespacing*{章节命令}{左边距}{上文距}{下文距}[右边距]
\titlespacing*{\section}{0pt}{2\baselineskip}{\parsep}


\usepackage{hyperref}

%%% perfect source code display
\usepackage{minted}
%\usemintedstyle{colorful}
\definecolor{srcbg}{rgb}{0.95,0.95,0.95}
\newminted{java}{linenos,tabsize=4,bgcolor=srcbg}
\newminted{xml}{linenos,tabsize=4,bgcolor=srcbg}
\newminted{cpp}{linenos,tabsize=4,bgcolor=srcbg}
\newminted{bash}{linenos,tabsize=4,bgcolor=srcbg}
\newminted{latex}{linenos,tabsize=4,bgcolor=srcbg}
\newminted{scheme}{linenos,tabsize=4,bgcolor=srcbg}

\usepackage{amsmath}


\input{../styles/tikz_preamble}

\sybtitle{Spring framework Notes}
\sybauthor{孙延宾}
\sybdate{\today}

\begin{document}
\tt % I love Typewriter font.
%%%%%%%% the title page and toc %%%%%%%%%%
\pagestyle{header}
\sybmaketitle
\tableofcontents
\newpage

%%%%%%% the main content %%%%%%%%%
\pagestyle{main}
\setcounter{page}{1}

\part[Spring framework]{Spring framework}

\section[web.xml解析]{web.xml解析}
web.xml文件用来配置Web应用,主要包括context-param, listener, filter, servlet等,
它们的加载顺序为:\\
context-param => listener => filter => servlet

它们的加载顺序是固定的,与xml中的出现顺序无关,但是同一类型的元素之间是按照顺序加载的,
如多个filter是按照先后顺序加载的,多个servlet也是按照先后顺序加载的。


\subsection[schema]{schema}
\begin{xmlcode}
<?xml version="1.0" encoding="UTF-8"?>
<web-app version="2.4" 
    xmlns="http://java.sun.com/xml/ns/j2ee" 
    xmlns:xsi="http://www.w3.org/2001/XMLSchema-instance"
    xsi:schemaLocation="http://java.sun.com/xml/ns/j2ee 
        http://java.sun.com/xml/ns/j2ee/web-app_2_4.xsd">
</web-app>
\end{xmlcode}

\subsection[icon]{icon}
icon用来指定Web应用的大小图标。

\begin{xmlcode}
<icon>
    <small-icon>/images/app_small.gif</small-icon>
    <large-icon>/images/app_large.gif</large-icon>
</icon>
\end{xmlcode}

\subsection[display-name]{display-name}
设置Web应用的名称。

\begin{xmlcode}
<display-name>Tomcat Example</display-name>
\end{xmlcode}

\subsection[discreption]{discreption}
Web应用的描述。

\begin{xmlcode}
<disciption>Tomcat Example servlets and JSP pages.</disciption>
\end{xmlcode}

\subsection[context-param]{context-param}
设置参数,该参数在整个应用范围内均可见,它们将会保存到ServletContext对象中。

\begin{xmlcode}
<context-param>
    <param-name>ContextParameter</para-name>
    <param-value>test</param-value>
    <description>It is a test parameter.</description>
</context-param>
\end{xmlcode}

这些参数在listener, filter, servlet中均可见,可以这样使用:

getServletContext().getInitParameter("param-name");


\subsection[listener]{listener}
listener分为多种,如ServletContext监听器,Sesssion监听器,Log4j监听器等,
在具体实例创建、销毁时这些监听器的相关方法会被调用到,具体解释请参照
第\ref{sec:listener}节。

\begin{xmlcode}
<listener> 
    <listerner-class>com.listener.SessionListener</listener-class> 
</listener>
\end{xmlcode}

\subsection[filter]{filter}
filter用于对request和response进行拦截,在request到达servlet之前进行预处理,
在response发送给客户端之前进行预处理等,详见第\ref{sec:filter}节。

\begin{xmlcode}
<filter>
    <filter-name>setCharacterEncoding</filter-name>
    <filter-class>com.myTest.setCharacterEncodingFilter</filter-class>
    <init-param>
        <param-name>encoding</param-name>
        <param-value>UTF-8</param-value>
    </init-param>
</filter>
<filter-mapping>
    <filter-name>setCharacterEncoding</filter-name>
    <url-pattern>/*</url-pattern>
</filter-mapping>
\end{xmlcode}

filter用来定义过滤器,filter-mapping用于将filter映射到servlet或者url。

\subsection[servlet]{servlet}
servlet用于定义servlet,servlet-mapping用于将servlet映射到url。

\begin{xmlcode}
<servlet>
    <servlet-name>snoop</servlet-name>
    <servlet-class>SnoopServlet</servlet-class>
    <init-param>
        <param-name>foo</param-name>
        <param-value>bar</param-value>
    </init-param>
    <run-as>
        <description>Security role for anonymous access</description>
        <role-name>tomcat</role-name>
    </run-as>
</servlet>
<servlet-mapping>
    <servlet-name>snoop</servlet-name>
    <url-pattern>/snoop/*</url-pattern>
</servlet-mapping>
\end{xmlcode}

\subsection[session-config]{session-config}
配置session。

\begin{xmlcode}
<session-config>
    <!-- 单位为分钟 -->
    <session-timeout>120</session-timeout>
</session-config>
\end{xmlcode}

\subsection[mime-mapping]{mime-mapping}
将文件后缀映射为mime type。

\begin{xmlcode}
<mime-mapping>
    <extension>htm</extension>
    <mime-type>text/html</mime-type>
</mime-mapping>
\end{xmlcode}

\subsection[welcome-file-list]{welcome-file-list}
设置欢迎界面(首页)。

\begin{xmlcode}
<welcome-file-list>
    <!-- 依次查找,找到为止 -->
    <welcome-file>index.jsp</welcome-file>
    <welcome-file>index.html</welcome-file>
    <welcome-file>index.htm</welcome-file>
</welcome-file-list>
\end{xmlcode}

\subsection[error-page]{error-page}
两种方式设置错误页面。

\begin{xmlcode}
<!-- 1、通过错误码来配置error-page。 -->
<error-page>
    <error-code>404</error-code>
    <location>/NotFound.jsp</location>
</error-page>
<!-- 2、通过异常的类型配置error-page。 -->
<error-page>
    <exception-type>java.lang.NullException</exception-type>
    <location>/error.jsp</location>
</error-page>
\end{xmlcode}

\subsection[jsp-config]{jsp-config}
配置jsp页面。包括两部分:taglib和jsp-property-group,
其中,jsp-property-group有8个元素。

\begin{xmlcode}
<jsp-config>
    <taglib>
        <taglib-uri>Taglib</taglib-uri>
        <taglib-location>/WEB-INF/tlds/MyTaglib.tld</taglib-location>
    </taglib>
    <jsp-property-group>
        <description>Special property group for JSP Configuration JSP example.</description>
        <display-name>JSPConfiguration</display-name>
        <url-pattern>/jsp/* </url-pattern>
        <el-ignored>true</el-ignored>
        <page-encoding>UTF-8</page-encoding>
        <scripting-invalid>true</scripting-invalid>
        <include-prelude>/include/prelude.jspf</include-prelude>
        <include-coda>/include/coda.jspf</include-coda>
    </jsp-property-group>
</jsp-config>
\end{xmlcode}

\section[URL Mapping规则]{URL Mapping规则}
当一个请求发送到servlet容器时(一个servlet一个容器),servlet会首先将上下文路径去掉,
如请求的url为:http://localhost:8080/test/index.html,test为我们的servlet(即在webapp
路径存在下一个test.war),这是test这个servlet的容器首先将
http://localhost:8080/test去掉,然后拿/index.html来做mapping。

匹配规则大致如下:

\begin{itemize}
\item 路径越精确越优先匹配
\item 能匹配上的路径越多越优先,如/a/*比/*要优先
\item 扩展匹配,如*.html
\end{itemize}

\section[Listener in Spring]{Listener in Spring}
\label{sec:listener}

\section[Filter in Spring]{Filter in Spring}
\label{sec:filter}


\section[Spring Annotations]{Spring Annotations}
通过注解的方式进行“注入”,其侧重点是“注入”,也就是说只是为了方便“注入”,
所以bean的模板类、声明,该怎么写还得怎么写,下面通过对比进行说明。

首先定义bean的模板类,

\begin{javacode}
package annotation;
public class MyBean {
  private String name;

  public void setName(String name) {
    this.name = name;
  }

  public String getName() {
    return this.name;
  }
}
\end{javacode}

然后是对bean进行声明,

\begin{xmlcode}
  <!-- beans.xml -->
  <beans>
    <!-- 告诉spring,我们要使用注解进行注入 -->
    <context:annotation-config />

    <bean id="foo" class="annotation.MyBean">
      <property name="name" value="xxx" />
    </bean>

    <bean id="bar" class="annotation.MyBean">
      <property name="name" value="yyy" />
      <!-- what's this? -->
      <qualifier value="bar" />
    </bean>
  </beans>
\end{xmlcode}

\subsection[xml方式注入]{xml方式注入}
\begin{javacode}
  ApplicationContext ctx = new ClassPathXmlApplicationContext(
                           "beans.xml");
  MyBean foo = (MyBean) ctx.getBean("foo");
  MyBean bar = (MyBean) ctx.getBean("bar");
\end{javacode}

\subsection[annotation方式注入]{annotation方式注入}
\begin{javacode}
  @Autowired
  @Qualifier(value="bar")  // 跟xml中的设置对应起来了
  private MyBean bar;

  @Resource(name="foo")
  private MyBean foo;

  // 上面两个对象,spring会自动注入,下面直接用就可以了
  public void print() {
    System.out.println(foo.name + ", " + bar.name);
  }
\end{javacode}

说明:bean的模板类定义、xml声明,这些该怎么做怎么做,只是等到实际
注入时才有所区分。使用注释方式进行注入时,配置文件中必须增加对
annotion的配置,即<context:annotation-config />,然后spring就会自动
扫描Java代码,遇到@Autowired、@Resource这样的注释时就会自动注入了。

注意:@Autowired注入时是byType,而我们定义两个同样类型的bean,导致
注入时报错,所以使用@Qualifier对另个bean进行区分,@Resource注入时是
byName的,使用时必须添加name="foo"这样的设置,这无形中增加了代码和
xml之间的耦合。

另外,对于注释方式注入,MyBean还可以在模板类定义中直接通过注释声明
init method和destroy method,它们分别在对象构造完成和销毁之前被调用,如,

\begin{javacode}
package annotation;
public class MyBean {
  private String name;

  public void setName(String name) {
    this.name = name;
  }

  public String getName() {
    return this.name;
  }

  @PostConstruct
  public void init() {
    System.out.println("bean successfully initialized");
  }
  
  @PreDestroy
  public void cleanUp() {
    System.out.println("clean up called");
  }
}
\end{javacode}

对于xml方式注入,这两个函数要在bean声明时说明。

\section[Spring Resource读入]{Spring Resource读入}
读入resource,如文件,url等,这些内容会封装到
org.springframework.core.io.Resource对象中,然后使用该对象的
getInputStream获得input stream,

\begin{description}
\item [FileSystemResource: ] e.g. new FileSystemResource("src/sample.txt");
\item [UrlResource: ] e.g. new UrlResource("file:///C:/sample.txt");
\item [ClassPathResource: ] e.g. new ClassPathResource("sample.txt");
\end{description}

\section[PropertyPlaceholder]{PropertyPlaceholder}
引入property文件,然后在xml中读取变量。

\begin{xmlcode}
<bean class=
      "org.springframework.beans.factory.config.PropertyPlaceholderConfigurer">
  <property name="locations" value="classpath:com/foo/jdbc.properties"/>
</bean>

<bean id="dataSource" destroy-method="close"
    class="org.apache.commons.dbcp.BasicDataSource">
  <property name="driverClassName" value="${jdbc.driverClassName}"/>
  <property name="url" value="${jdbc.url}"/>
  <property name="username" value="${jdbc.username}"/>
  <property name="password" value="${jdbc.password}"/>
</bean>
\end{xmlcode}

jdbc.properties文件内容,

\begin{bashcode}
jdbc.driverClassName=org.hsqldb.jdbcDriver
jdbc.url=jdbc:hsqldb:hsql://production:9002
jdbc.username=sa
jdbc.password=root
\end{bashcode}


\section[Default DispatcherServlet Configuration]{Default DispatcherServlet Configuration}
在\emph{org.springframework.web.servlet}包中,有一个
\emph{DispatcherServlet.properties}文件,其中保存了DispatcherServlet
的默认设置。

\section[Two kind of bean]{Two kind of bean}
bean有两种:普通bean、factory bean。

\subsection[普通bean]{普通bean}
普通bean实例化时直接创建对应class的实例,如

\begin{xmlcode}
<bean id="commonBean" class="com.example.main.CommonBean" />
\end{xmlcode}

Spring在实例化该bean时,直接创建一个com.example.main.CommonBean的实例。

\subsection[工厂bean]{工厂bean}
与普通bean不同,factory bean实例化时并不返回factory bean本身的实例,而是
返回getObject()的返回值,factory bean均实现FactoryBean<T>这个interface,

\begin{javacode}
public interface FactoryBean<T> {
  T getObject() throws Exception;
  Class<?> getObjectType();
  boolean isSingleton();
}
\end{javacode}

\subsection[工厂bean实例]{工厂bean实例}
连接数据库时经常会用到Spring的ORM模块,事实上这也是推荐的做法,

\begin{xmlcode}
<bean id="dataSource" class=
      "org.springframework.jdbc.datasource.DriverManagerDataSource">
    <property name="driverClassName" value="${jdbc.driver}" />
    <property name="url" value="${jdbc.url}" />
    <property name="username" value="${jdbc.user}" />
    <property name="password" value="${jdbc.password}" />
</bean>

<!-- LocalSessionFactoryBean便是一个factory bean,
     它的getObject()返回一个Hibernate的SessionFactory对象 -->
<bean id="sessionFactory" class=
      "org.springframework.orm.hibernate4.LocalSessionFactoryBean">
    <property name="dataSource" ref="dataSource"/>
    <property name="mappingLocations">
        <list>
            <value>classpath*:hbm/User.hbm.xml</value>
        </list>
    </property>
    <property name="hibernateProperties">
        <props>
            <prop key="hibernate.dialect">
                org.hibernate.dialect.PostgreSQLDialect</prop>
            <prop key="hibernate.show_sql">true</prop>
        </props>

    </property>
</bean>
\end{xmlcode}

然后在java代码中进行注入,

\begin{javacode}
public exampleDAO {
  // @Autowired注入时是byType的,而LocalSessionFactoryBean这个bean的
  // getObjectType()恰好返回SessionFactory类型,匹配!
  @Autowired
  private SessionFactory sf;

  public Session getCurrentSession() {
    return sf.getCurrentSession();
  }
}
\end{javacode}

注意:这是Spring与Hibernate结合时的标准实现方式:\\
Spring -> ORM -> Hibernate

\section[Log4j]{Log4j}
在Spring中使用log4j进行log输出。

\subsection[启动log4j]{启动log4j}
在web.xml中增加listener,这样wervlet启动时自动启动log4j。

\begin{xmlcode}
<!-- log4j configuration -->
<context-param> 
  <param-name> log4jConfigLocation</param-name> 
  <param-value>/WEB-INF/properties/log4j.properties</param-value> 
</context-param> 
<context-param> 
  <param-name> log4jRefreshInterval</param-name> 
  <!-- 每隔60s重新加载一次property文件 -->
  <param-value>60000</param-value> 
</context-param>

<listener> 
  <listener-class> 
    org.springframework.web.util.Log4jConfigListener 
  </listener-class> 
</listener>
\end{xmlcode}

\subsection[配置log4j]{配置log4j}
配置log4j。

\begin{bashcode}
# /WEB-INF/properties/log4j.properties

#### 定义格式: log4j.rootLogger = [ level ] , appenderName, appenderName, ...
# level: 依次为 debug, info, warn, error, fatal
# appenderName: 自定义的输出方式
# 下面的 console, R为两个自定义输出方式
log4j.rootLogger = INFO, console, R

#### 定义 console 的信息
# console 的执行类
log4j.appender.console = org.apache.log4j.ConsoleAppender
# 设置layout,有四种可选格式:
#org.apache.log4j.HTMLLayout
#org.apache.log4j.PatternLayout
#org.apache.log4j.SimpleLayout
#org.apache.log4j.TTCCLayout
log4j.appender.console.layout = org.apache.log4j.PatternLayout
# 设置输出log的格式
log4j.appender.console.layout.ConversionPattern =
                              %-d{yyyy-MM-dd HH:mm:ss} [%c]-[%p] %m%n

#### 定义 R 的信息
# 设置 R 将log输出到文件
log4j.appender.R = org.apache.log4j.RollingFileAppender
# 设置log文件路径,相对于tomcat的根目录
log4j.appender.R.File = logs/log4j.txt
# 设置log文件大小
log4j.appender.R.MaxFileSize = 4MB
# 当log文件达到上限时,将其重命名为xxx.1,xxx.2,直到xxx.MaxBackupIndex为止
log4j.appender.R.MaxBackupIndex = 10
log4j.appender.R.layout = org.apache.log4j.PatternLayout
log4j.appender.R.layout.ConversionPattern =
                        %-d{yyyy-MM-dd HH:mm:ss} [%c]-[%p] - %m%n

#### conversion pattern
# %c : class,日志信息所属类的全称
# %d : date,可以通过"%-d{}"的形式自定义时间
# %f : file,日志所属的文件名称
# %l : line,日志所在的行数
# %m : message,也就是具体的log内容
# %n : newline,根据平台自定适应
# %p : priority,优先级字符串
# %r : runtime,应用启动至日志输出的时长,单位毫秒
# %t : thread,日志所在的线程名称

#### 代码中的使用方法
# import org.apache.log4j.Logger
# private static Logger logger = Logger.getLogger(<ClassName>.class);
# logger.info("This is the log message");
\end{bashcode}

\subsection[使用log4j]{使用log4j}
在Java代码中使用log4j。

\begin{javacode}
import org.apache.log4j.Logger
private static Logger logger = Logger.getLogger(<ClassName>.class);
logger.info("This is the log message");
\end{javacode}


\section[使用Maven创建all-in-one jar文件]{使用Maven创建all-in-one jar文件}
all-in-one jar文件就是将所有依赖的jar文件以及“配置文件”全都打包进一个jar文件。
创建Spring的all-in-one jar文件时使用maven-shade-plugin,具体做法是在pom.xml中
添加该插件,

\begin{xmlcode}
<plugin>
  <groupId>org.apache.maven.plugins</groupId>
  <artifactId>maven-shade-plugin</artifactId>
  <version>2.2</version>
  <executions>
    <execution>
      <phase>package</phase>
      <goals>
        <goal>shade</goal>
      </goals>
      <configuration>
        <transformers>
          <!-- 操作MANIFEST.MF文件 -->
          <transformer
            implementation="org.apache.maven.plugins.shade.resource
                                            .ManifestResourceTransformer">
            <mainClass>com.example.service.App</mainClass>
          </transformer>
          <!-- 将每个jar包下的META-INF/spring.schemas文件合并到一起
               并放置到all-in-one jar文件的META-INF/spring.schemas文件中-->
          <transformer
            implementation="org.apache.maven.plugins.shade.resource
                                            .AppendingTransformer">
            <resource>META-INF/spring.schemas</resource>
          </transformer>
          <!-- 同spring.schemas文件类似 -->
          <transformer
            implementation="org.apache.maven.plugins.shade.resource
                                            .AppendingTransformer">
            <resource>META-INF/spring.handlers</resource>
          </transformer>
          <!-- 同spring.schemas文件类似 -->
          <transformer
            implementation="org.apache.maven.plugins.shade.resource
                                            .AppendingTransformer">
            <resource>META-INF/spring.tooling</resource>
          </transformer>
        </transformers>
      </configuration>
    </execution>
  </executions>
</plugin>
\end{xmlcode}

如果没有对META-INF/spring.schemas、spring.handlers文件的处理,运行时就会
找不到namespace的handler。


\part[Aspect Oriented Programming]{Aspect Oriented Programming}
Aspect Oriented Programming(AOP),面向方面编程,是OOP的补充和完善。它并非要取代OOP,
AOP是一种寄生式编程方式,它的存在的目的是为了“增强”OOP。

AOP的本质是Proxy pattern。AOP的作用就是选出业务对象中的某些方法,然后在这些
方法执行前、后、返回、异常时添加“动作”(函数,可以理解为回调函数)。


\section[术语]{术语}
代理模式(proxy pattern)中的概念:

\begin{description}
\item [target: ] 业务对象,也就是被代理的对象,真正的业务逻辑在此实现
\item [proxy: ] 代理对象,实际使用的对象
\end{description}

AOP中的概念:

\begin{description}
\item [joinpoint: ] 抽象概念,可以理解为业务对象中被增强的方法,一般为advice方法接受的参数
\item [pointcut: ] 业务对象方法(joinpoint)的选择器,AOP就是要“增强”被pointcut选出来的方法
\item [advice: ] 对pointcut选出来的方法执行的动作,如这些方法执行前、后、返回、异常时执行的动作
\item [aspect: ] pointcut和advice组成的类
\end{description}

AOP的主要advice类型:

\begin{description}
\item [before advice: ] 在pointcut之前执行
\item [after returning advice: ] 在pointcut return之后执行
\item [after throwing advice: ] 在pointcut抛出异常时执行
\item [around advice: ] 上述advice的综合,可以灵活的自定义需要哪些advice
\end{description}

\section[Proxy机制]{Proxy机制}
由于AOP本质上是proxy pattern,所以我们首先介绍一下proxy pattern。

\subsection[proxy模式]{proxy模式}
我们通过一个ITalk接口、一个ITalk实现、一个ITalk代理来演示。

首先定义一个ITalk接口,

\begin{javacode}
public interface ITalk {
  public void talk(String message);
}
\end{javacode}

然后是一个具体的实现,

\begin{javacode}
public class TalkImpl implements ITalk {
  public String name;
  public int age;

  public TalkImpl(String name, int age) {
    this.name = name;
    this.age = age;
  }

  @Override
  public void talk(String message) {
    System.out.println("name: " + name + ", age: " + age + ", say: " + message);
  }
}
\end{javacode}

接下来是代理类,

\begin{javacode}
public class TalkProxy implements ITalk {
  private ITalk talker;

  public TalkProxy(ITalk talker) {
    this.talker = talker;
  }

  @Override
  public void talk(String message) {
    talk.talk(message);
  }

  // let sing a song.
  public void talk(String message, String song) {
    talker.talk(message);
    sing(song);
  }

  private void sing(String song) {
    System.out.println("sing a song: " + song);
  }
}
\end{javacode}

这就是AOP的原理:
\begin{itemize}
\item talk()方法视为pointcut
\item sing()方法视为advice
\item sing()等其他“增强”方法(如果有的话)的集合视为aspect
\end{itemize}

代理模式有两个缺点:
\begin{itemize}
\item 必须定义接口
\item 不同的业务类需要定义不同的代理类,注意是类而不是对象
\end{itemize}

\subsection[使用JDK动态代理]{使用JDK动态代理}
为了弥补“不同业务类需要不同代理类”的缺点,我们使用JDK动态代理机制,
它利用反射机制,代理类只需实现InvocationHandler接口即可。

动态代理类定义:

\begin{javacode}
public class DynamicProxy implements InvocationHandler {
  // 目标对象
  private Object target;

  // 绑定目标对象
  public Object bind(Object target) {
    this.target = target;
    return Proxy.newProxyInstance(
           target.getClass().getClassLoader(),
           target.getClass().getInterfaces(), // 这里使用到接口了
           this);
  }

  @Override
  pbulic Object invoke(Object proxy, Method method, Object[] args)
                throws Throwable {
    Object result = null;
    // pointcut之前执行
    System.out.println("before pointcut...");
    // pointcut执行
    result = method.invoke(target, args);
    // pointcut之后执行
    System.out.println("after pointcut...");

    return result;
  }
}
\end{javacode}

不同的业务类都可以使用这个代理对象,只需bind一下即可,当然,
实际使用中invoke方法会比较复杂,因为我们在这里“增强”业务方法,
即pointcut。

它的确定也很明显:仍然需要定义接口。

\subsection[使用cglib代理]{使用cglib代理}
接下来我们使用cglib进行代理,它采用的继承机制,其原理是对指定的目标类生成一个子类,
并覆盖其中的方法,实现“增强”。它既不需要定义接口,也不需要为不同业务类创建不同
代理类。

cglib代理类定义:

\begin{javacode}
public class CGlibProxy implements MethodInterceptor {
  private Object target;

  // 创建代理对象
  public Object getInstance(Object target) {
    this.target = target;
    
    Enhancer enhancer = new Enhancer();
    enhancer.setSuperclass(this.target.getClass());
    enhancer.setCallback(this);
    //返回代理对象
    return enhancer.create();
  }

  // 调用代理对象的方法时会回调到intercept()方法
  @Override
  public Object intercept(
         Object proxy, // 代理对象,也就是业务对象的子类对象
         Method method,
         Object[] args,
         MethodProxy methodProxy) throws Throws {
    Object result = null;
    System.out.println("before pointcut...");
    result = methodProxy.invokeSuper(proxy, args);
    System.out.println("after pointcut...");

    return result;
  }
}
\end{javacode}

Spring AOP就是利用JDK动态代理和CGlib代理实现的。

\section[在Spring中使用AOP]{在Spring中使用AOP}
接下来我们看看,如何在Spring中使用AOP,我们以
before advice和around advice为例。

\subsection[使用继承方式]{使用继承方式}
通过继承的方式虽然繁琐,但是流程更加清晰。

业务类:

\begin{javacode}
package com.example.aop;

public class BookService {
  public void borrowBook() {
    System.out.println("get book...");
  }

  public void returnBook() {
    System.out.println("return book...");
  }

  public void washBook() {
    throw new RuntimeException("the book can not be washed.");
  }
}
\end{javacode}

before advice:

\begin{javacode}
package com.example.aop;
import org.springframework.aop.MethodBeforeAdvice;

public class BeforeAdvice implements MethodBeforeAdvice {
  @Override
  public void before(Method method, Object[] args, Object target)
                     throws Throwable {
    System.out.println("before advice...");
  }
}
\end{javacode}

around advice:

\begin{javacode}
package com.example.aop;

import org.aopalliance.intercept.MethodInterceptor;
import org.aopalliance.intercept.MethodInvocation;

public class AroundAdvice implements MethodInterceptor {
  @Override
  public Object invoke(MethodInvocation invocation) throws Throwable {
    // before advice
    System.out.println("around advice, before");
    try {
      // pointcut
      Object result = invocation.processd();
      // after pointcut
      System.out.println("around advice, after");
      
      return result;
    } catch (RuntimeException e) {
      System.out.println("around advice, exception");
      throw e;
    }
  }
}
\end{javacode}

pointcut:

\begin{javacode}
package com.example.aop;

// pointcut对象并非指的是业务对象方法实体,它的作用只是
// 选出那些方法来。
public class Pointcut extends NameMatchMethodPointcut {
  public boolean matches(Method method, Class targetClass) {
    // 设置单个方法匹配
    this.setMappedName("borrowBook");

    // 使用正则表达式匹配
    // this.setMappedName("borrow*");

    // 设置多个方法匹配
    // String[] methodNames = { "borrowBook", "returnBook" };
    // this.setMappedNames(methodNames);
  }
}
\end{javacode}

xml配置文件:

\begin{xmlcode}
<beans>
  <!-- 创建业务对象 -->
  <bean id="bookService" class="com.example.aop.Bookservice" />

  <!-- 创建before advice对象 -->
  <bean id="beforeAdvice" class="com.example.aop.Beforeadvice" />
  <!-- 创建around advice对象 -->
  <bean id="aroundAdvice" class="com.example.aop.Aroundadvice" />

  <!-- 创建pointcut对象 -->
  <bean id="borrowBookPointcut" class="com.example.aop.Pointcut" />
  <!-- 创建pointcut advisor对象,
       advisor的作用就是“给某些pointcut配备某些advise” -->
  <bean id="aroundBorrowBookPointcutAdvisor"
        class="org.springframework.aop.support.DefaultPointcutAdvisor" >
    <property name="pointcut">
      <ref bean="borrowBookPointcut" />
    </property>
    <property name="advise">
      <ref bean="aroundAdvice" />
    </property>
  </bean>
    
  <!-- 创建代理对象 -->  
  <bean id="bookServiceProxy"
        class="org.springframework.aop.framework.ProxyFactoryBean">
    <property name="target" ref="bookService" />
    <property name="interceptorNames">
      <list>
        <!-- 业务对象的每个方法(pointcut)均配备before advise -->
        <value>beforeAdvice</value>
        <!-- <value>aroundAdvice</value> -->
        <!-- 这里不直接使用around advise,而是使用一个advisor -->
        <value>aroundBorrowBookPointcutAdvisor</value>
      </list>
    </property>
  </bean>
</beans>
\end{xmlcode}

使用的时候,直接使用bookServiceProxy对象即可!

\subsection[使用Aspect类]{使用Aspect类}
业务类:

\begin{javacode}
package com.example.aop;

public class BookService {
  public void borrowBook() {
    System.out.println("get book...");
  }

  public void returnBook() {
    System.out.println("return book...");
  }

  public void washBook() {
    throw new RuntimeException("the book can not be washed.");
  }
}
\end{javacode}

aspect类:

\begin{javacode}
package com.example.aop;

public class MyAspect {
  public void doBefore(JoinPoint jp) {
    System.out.println("该动作将在"
               + jp.getSignature().getName() 
               + "之前执行");

  public void doAfter(JoinPoint jp, String result) {}

  public void doAround(ProceedingJoinPoint pjp) throw Throwable {
    // around before
    System.out.println("around before...");
    // execute pointcut
    Object result = pjp.proceed();
    // around after
    System.out.println("around after...");
  }

  public void doThrow(JoinPoint jp, Throwable e) {}
\end{javacode}

xml配置文件:

\begin{xmlcode}
<beans>
  <bean id="bookService" class="com.example.aop.Bookservice" />

  <aop:config>
    <aop:aspect id="myAspect" ref="bookservice">
      <aop:pointcut id="bookServicePointcut"
                    expression="execution(* com.example.aop.BookService.*(..))" />
      <aop:before pointcut-ref="bookServicePointcut" method="doBefore" />
      <aop:after pointcut-ref="bookServicePointcut" method="doAfter" />
      <aop:around pointcut-ref="bookServicePointcut" method="doAround" />
      <aop:after-throwing pointcut-ref="bookServicePointcut" method="doThrow"
                          throwing="e" />
    </aop:aspect>
  </aop:config>
</beans>
\end{xmlcode}



\subsection[使用注解]{利用注解}
实际使用中都是利用AspectJ注解,简洁明了。

业务类:

\begin{javacode}
package com.example.aop;

// @Component创建bean对象,不用在xml中配置了
@Component
public class BookService {
  public void borrowBook() {
    System.out.println("get book...");
  }

  public void returnBook() {
    System.out.println("return book...");
  }

  public void washBook() {
    throw new RuntimeException("the book can not be washed.");
  }
}
\end{javacode}

aspect类:

\begin{javacode}
package com.example.aop;

// 直接生成bean对象,并且是aspect类型的对象
@Component
@Aspect
public class MyAspect {
  // value表示pointcut,它会选出需要增强的函数,这里表示在
  // BookService的任何方法上执行
  @Before(value="execution(* com.example.aop.BookService.*(..))")
  public void doBefore(JoinPoint jp) {
    System.out.println("该动作将在"
               + jp.getSignature().getName() 
               + "之前执行");

  @AfterReturning(value="execution(* com.example.aop.BookService.*(..))")
  public void doAfter(JoinPoint jp, String result) {}

  @Around(value="execution(* com.example.aop.BookService.*(..))")
  public void doAround(ProceedingJoinPoint pjp) throw Throwable {
    // around before
    System.out.println("around before...");
    // execute pointcut
    Object result = pjp.proceed();
    // around after
    System.out.println("around after...");
  }

  @AfterThrowing(value="execution(* com.example.aop.BookService.*(..))",
                        throwing="e")
  public void doThrow(JoinPoint jp, Throwable e) {}
\end{javacode}

xml配置文件:

\begin{xmlcode}
<beans>
  <context:component-scan base-package="com.example.aop" />
  <aop:aspectj-autoproxy />
</beans>
\end{xmlcode}



\section[使用AOP进行事务管理]{使用AOP进行事务管理}
事务管理是Spring AOP的重要应用之一。

\subsection[tx命名空间]{tx命名空间}
\subsection[实例]{实例}


\section[JDK dynamic proxy]{JDK dynamic proxy}

\section[CGlib]{CGlib}



\part[Spring Annotations]{Spring Annotations}
尽可能全面的总结Spring的annotations。

\section[Java Annotation]{Java Annotation}
如何编写Java Annotation。

Java Annotations位于JDK的rt.jar中的java.lang.annotation包中,包括,
\begin{enumerate}
\item Documented
\item Inherited
\item Retention
\item Target
\end{enumerate}

\section[AOP Annotations]{AOP Annotations}
Spring使用AspectJ的AOP annotations,这些annotations定义位于
aspectjweaver.jar中,所处的package为:org.aspectj.lang.annotation,
包括如下annotation,
\begin{enumerate}
\item AdviceName
\item After
\item AfterReturning
\item AfterThrowing
\item Around
\item Aspect
\item Before
\item DeclareAnnotation
\item DeclareError
\item DeclareMixin
\item DeclareParents
\item DeclarePrecedence
\item DeclareWarning
\item Pointcut
\item SuppressAjWarnings
\end{enumerate}

\section[Stereotype Annotations]{Stereotype Annotations}
stereotype annotations是Spring的标准annotations,它们用于创建beans,位于
spring-context模块的org.springframework.stereotype包中,包括,
\begin{enumerate}
\item Component
\item Controller
\item Service
\item Repository
\end{enumerate}
其中,@Component会创建通用性bean,而@Controller、@Service、@Repository
分别创建针对表现层、业务层、持久化层的bean。

\section[Dependency Injection Annotations]{Dependency Injection Annotations}
这部分annotation用于beans的自动插入,它们位于spring-beans模块的
org.springframework.beans.factory.annotation包中,包括,
\begin{enumerate}
\item Autowired
\item Configurable
\item Qualifier
\item Required
\item Value
\end{enumerate}

\section[MVC Annotations]{MVC Annotations}
Spring MVC方面的annotations位于spring-web模块的
org.springframework.web.bind.annotation包中,包括,
\begin{enumerate}
\item ControllerAdvice
\item CookieValue
\item ExceptionHandler
\item InitBinder
\item Mapping
\item MatrixVariable
\item ModelAttribute
\item PathVariable
\item RequestBody
\item RequestHeader
\item RequestMapping
\item RequestParam
\item RequestPart
\item ResponseBody
\item ResponseStatus
\item RestController
\item SessionAttributes
\end{enumerate}

\section[Spring Context Annotation]{Spring Context Annotation}
这部分注解的作用是什么呢?这个包里的内容很有意思!!!
它们位于spring-context模块的org.springframework.context.annotation包中,
包括,
\begin{enumerate}
\item Bean
\item ComponentScan
\item Conditional
\item Configuration
\item DependsOn
\item Description
\item EnableAspectJAutoProxy
\item EnableLoadTimeWeaving
\item EnableMBeanExport
\item Import
\item ImportResource
\item Lazy
\item Primary
\item Profile
\item PropertySource
\item PropertySources
\item Role
\item Scope
\end{enumerate}

\section[javax annotation]{javax annotation}
另外还有javax里的annotation,它们位于JDK的rt.jar的javax.annotation包中,
Spring似乎对部分注解做了不同的解析,所有这方面的annotations包括,
\begin{enumerate}
\item Generated
\item PostConstruct
\item PreDestroy
\item Resource
\item Resources
\end{enumerate}

\section[如何使用annotation]{如何使用annotation}
annotation只不过是一些“标识”,一些marker罢了,真正让这些标识起作用的
是隐藏在背后的“解析器”。

\subsection[使用注解创建bean]{使用注解声明bean}
四个用于声明bean的注解,它们用法类似,都表示声明一个bean。
\begin{description}
\item [@Component: ] 表示一个通用性的bean,可用于任何层次
\item [@Repository: ] 用于持久化层
\item [@Service: ] 用于业务层
\item [@Controller: ] 用于表现层
\end{description}
要想让这些注解起作用,必须要注册相应的扫描器来识别它们,

\begin{xmlcode}
<context:component-scan base-package="your.base.package" />
\end{xmlcode}

这样,在读取到这些注解时,Spring调用相关的扫描器创建BeanDefinition对象
并注册到ApplicationContext中。然后这些生成的bean叫什么名字呢?它们的名字
会根据其扫描器的BeanNameGenerator策略生成,默认情况下,如果这些注解提供
name属性时,生成的bean就是name属性的值,否则就是用小写开头的非限定类名
(非限定类名,即为不包括报名的类名),当然也可以自定义命名策略,只需实现
BeanNameGenerator接口即可,

\begin{xmlcode}
<context:component-scan
  base-package="com.example"
  name-generator="com.example.MyBeanNameGenerator" />
\end{xmlcode}

另外,上述注解并没有作用域的属性,他们的默认作用域为singleton,可以使用
@Scope注解修改其作用域,

\begin{javacode}
  @Scope("prototype")
  @Repository
  public class MyBean { }
\end{javacode}

\subsection[使用注解进行Bean的依赖检查]{使用注解进行Bean的依赖检查}
使用@Required判断给定bean的setter方法在实例化时是否被调用了,某人为true,
如果setter方法没有被调用,Spring在解析时会抛出异常,该注解需要激活
org.springframework.beans.factory.annotation.RequiredAnnotationBeanPostProcessor
后处理器,所以需要对annotation进行配置,

\begin{xmlcode}
  <context:annotation-config />
\end{xmlcode}

\subsection[使用注解注入bean]{使用注解注入bean}

\subsection[使用Configuration和Bean注解声明Bean]{使用Configuration和Bean注解声明Bean}


\part[Hibernate]{Hibernate}

\end{document}