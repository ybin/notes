% !TEX TS-program = xelatex
% !TEX encoding = UTF-8 Unicode
% !Mode:: "TeX:UTF-8"

\documentclass{resume}
\usepackage{zh_CN-Adobefonts_external} % Simplified Chinese Support using external fonts (./fonts/zh_CN-Adobe/)
%\usepackage{zh_CN-Adobefonts_internal} % Simplified Chinese Support using system fonts
\usepackage{linespacing_fix} % disable extra space before next section
\usepackage{cite}

\begin{document}
\pagenumbering{gobble} % suppress displaying page number

\name{孙延宾}

% {E-mail}{mobilephone}{homepage}
% be careful of _ in emaill address
\contactInfo{ybin.sun@163.com}{(+86) 187 0671 8156}{http://ybin.cc/}
% {E-mail}{mobilephone}
% keep the last empty braces!
%\contactInfo{xxx@yuanbin.me}{(+86) 131-221-87xxx}{}
 
\section{\faGraduationCap\  教育经历}
\datedsubsection{\textbf{中南大学}, 湖南, 长沙}{2008 -- 2011}
\textit{硕士}\ 基础数学
\datedsubsection{\textbf{山东大学(威海)}, 山东, 威海}{2004 -- 2008}
\textit{学士}\ 信息与计算科学

\section{\faUsers\ 工作经历}
\datedsubsection{\textbf{中兴通讯}(西安)}{2011年7月 -- 至今}
\role{业务软件开发工程师}{}
\begin{onehalfspacing}
\begin{itemize}
  \item 共同完成camera app新架构的设计和开发
  \item 独立完成全新的camera app模式切换组件
  \item 独立完成列表的动画显示组件
  \item Music player app的项目维护
  \item Gallery app的项目维护
  \item 参与ZTE星星系列、AXON系列的camera app开发及维护
\end{itemize}
\end{onehalfspacing}


% Reference Test
%\datedsubsection{\textbf{Paper Title\cite{zaharia2012resilient}}}{May. 2015}
%An xxx optimized for xxx\cite{verma2015large}
%\begin{itemize}
%  \item main contribution
%\end{itemize}

\section{\faCogs\ Android技能}
% increase linespacing [parsep=0.5ex]
\begin{description}[parsep=0.5ex]
  \item[Android系统] \ 
    \begin{description}
         \item[整体架构] \ \\ 了解Android的整体架构,包括Linux kernel,Android native service(如service manager, activity manager, window manager etc.) 等。
         \item[启动流程] \ \\ 了解从Linux kernel到native service到Zygote到system service的整个流程,并阅读过相关源码。
         \item[Binder机制] \ \\ 理解Binder的机制以及service manger,能编写简单的native service和system service,了解binder driver的实现。
         \item[handler消息处理机制] \ \\ 理解Android的handler消息处理机制、Android application与Linux process之间的关系。
         \item[View渲染流程] \ \\ 理解Android view的渲染流程(measure, layout, draw),能完成自定义UI组件。
         \item[应用开发] \ \\ 熟练掌握Android App开发流程、开发细节、架构设计,动画视效等,能完成大型App的维护工作,能独立完成简单App的开发工作。
    \end{description}
  \item[Camera软件技术栈] \ 
    \begin{itemize}
       \item 熟练掌握camera app的业务知识以及开发流程
       \item 理解camera2 API并能简单应用
       \item 理解camera的framework架构及实现
       \item 理解Android HAL机制,部分了解高通camera HAL层的实现
    \end{itemize}
  \item[开发工具] \ 
    \begin{itemize}
        \item 熟练使用ADT开发环境
        \item 熟练使用Android Studio开发环境、Gradle构建系统
    \end{itemize}
\end{description}

\newpage

\section{\faCogs\ 编程技能}
\begin{description}[parsep=0.5ex]
  \item[C和C++] \ \\ 熟练掌握C语言编程,理解编译、链接的过程; 了解C++,面向对象、多重继承、类模板等,阅读代码无压力。
  \item[JVM平台] \ \\ 熟悉JVM平台(HotSpot VM),了解内存布局、GC、多线程、类加载机制等,熟练掌握JNI编程及原理,为学习JVM,曾仔细阅读过JamVM(Dalvik VM的前身)源码。
  \item[RxJava] \ \\ 能清晰的理解其实现原理,自己曾完成一个及其简易的实现,以此来理解RxJava的原理。
  \item[函数式编程] \ \\ 能清晰的理解函数的一等公民特性、数据不可变性、纯函数特性、闭包、惰性求值等,学习过Clojure编程,了解Emacs lisp语言、Groovy语言。
  \item[JavaScript] \ \\ 掌握JavaScript的语法,能理解ECMAScript6的新特性,清晰理解基于原型链的继承机制(prototype),了解Nodejs平台。
  \item[Linux平台] \ 
    \begin{description}
      \item[Kernel] \ \\ 了解内存管理、虚拟文件系统、驱动程序编写,曾深入内核源码进行部分学习,完整阅读过xv6项目的源码,熟悉进程的内存布局。
      \item[Linux distros] \ \\ 使用过多个linux发行版本,RedHat、Fedora、Debian,最后稳定在Ubuntu。
      \item[Shell环境] \ \\ 了解shell编程,纯命令行下工作无压力。
    \end{description}
\end{description}

\section{\faInfo\ 其他}
% increase linespacing [parsep=0.5ex]
\begin{description}
  \item[英语] \ \\ 英语六级。
  \item[Git] \ \\ 熟练使用并且透彻理解git的内部原理,工作以及自己业余写的代码都会用git来进行版本控制。
  \item[Python, Bat] \ \\ 使用简单的脚本来完成日常工作中一些重复性的工作,简单的用批处理(Windows平台)来完成,复杂一些的用Python完成(如自动清理Android app项目的无用资源来给apk瘦身)。
  \item[Nginx] \ \\ 搭建内部使用的web服务器,如为app团队提供的android sdk镜像服务器,常用开发工具下载服务器,用于调试blog的本地web服务器,结合PHP/Nodejs实现动态markdown解析服务器。
  \item[\LaTeX{}排版系统] \ \\ 熟练使用,毕业论文、技术交流使用的文档、公司培训使用的幻灯片、甚至当前这份简历都是用\LaTeX{}完成的。
\end{description}

%% Reference
%\newpage
%\bibliographystyle{IEEETran}
%\bibliography{mycite}
\end{document}
