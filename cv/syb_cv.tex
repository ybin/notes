

% possible options include:
% font size ('10pt', '11pt' and '12pt'),
% paper size ('a4paper', 'letterpaper', 'a5paper', 'legalpaper', 'executivepaper' and 'landscape')
% font family ('sans' and 'roman')
\documentclass[12pt,a4paper,roman]{moderncv}

% moderncv 主题
\moderncvstyle{oldstyle} % 选项参数是 ‘casual’, ‘classic’, ‘oldstyle’ 和 ’banking’
\moderncvcolor{grey}    % 选项参数是 ‘blue’ (默认)、‘orange’、‘green’、‘red’、‘purple’ 和 ‘grey’
%\nopagenumbers{}       % 消除注释以取消自动页码生成功能

% 字符编码
\usepackage[utf8]{inputenc} % 替换你正在使用的编码
%\usepackage{CJKutf8}

% 调整页面出血
\usepackage[scale=0.75]{geometry}
%\setlength{\hintscolumnwidth}{3cm} % 如果你希望改变日期栏的宽度

%%% xeCJK
\usepackage{xeCJK}
\setCJKmainfont[BoldFont=Adobe Heiti Std]{Adobe Heiti Std}
\setCJKsansfont[BoldFont=Adobe Heiti Std]{Adobe Heiti Std}
\setCJKmonofont[BoldFont=Adobe Heiti Std]{Adobe Song Std}
\XeTeXlinebreaklocale "zh"
\XeTeXlinebreakskip=0pt plus 1pt minus 0.1pt

% 个人信息
\name{孙延宾}{}
%\title{简历题目}                              % 可选项、如不需要可删除本行
\phone[mobile]{187~0671~8156}              % 可选项、如不需要可删除本行
\email{ybin.sun@163.com}                    % 可选项、如不需要可删除本行
\extrainfo{\\[.2em]github: @ybin\\ website: http://ybin.cc/}                 % 可选项、如不需要可删除本行
%\photo[64pt][0.4pt]{picture}                  % ‘64pt’是图片必须压缩至的高度、‘0.4pt‘是图片边框的宽度 (如不需要可调节至0pt)、’picture‘ 是图片文件的名字;可选项、如

\begin{document}
\maketitle


\section{教育经历}
\cventry{2004 -- 2008}{学士}{~山东大学(威海)}{}{}{信息与计算科学专业(数学院)}
\vspace{.35em}
\cventry{2008 -- 2011}{硕士}{~中南大学}{}{}{基础数学专业}  % 第3到第6编码可留白


\section{工作经历}
\cventry{2011 -- 现在}{软件开发工程师}{中兴通讯}{西安}{}{Android应用软件开发\\[.5em]%
工作内容:%
\begin{itemize}%
\item 照相机(Camera)应用层开发
\item 熟悉android camera的软件架构
\item 音乐播放器(Music Player)开发以及项目维护
\item 其他Android平台应用开发
\item Android应用平台UI界面设计实现、动画视效、新组件开发、应用架构设计等
\item 熟悉android平台框架,对进程间通讯机制、app组件生存周期、UI绘制机制等能够熟练掌握
\item 参与的项目包括星星系列、Axon系列等
\end{itemize}}

\section{编程技能}
\cvitemwithcomment{Java}{}{工作的主要语言} % 语言、水平、评价
\cvitemwithcomment{C}{}{最喜欢的编程语言}
\cvitemwithcomment{Linux Kernel}{}{一片向往的乐园,孜孜以求}
\cvitemwithcomment{JavaScript}{}{继承机制的另一种实现}
\cvitemwithcomment{Clojure}{}{初探函数式编程}
\cvitemwithcomment{Python,Bash etc}{}{日常工作偶尔用到}

\section{其他}
\cvlistitem{JVM平台:对虚拟机的实现及JNI都能熟练掌握}
\cvlistitem{Git版本控制系统:日常使用以及底层原理非常了解}
\cvlistitem{\LaTeX{}排版系统:熟练掌握}
\cvlistitem{Linux环境:常见发行版本都有涉及(个人日常使用的桌面环境),对内核也比较了解}

\section{个人介绍}
喜欢研究新鲜事物并深入探究;\\
喜欢阅读,喜欢coding并乐在其中。

\clearpage
\end{document}
