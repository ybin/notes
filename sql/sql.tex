
\documentclass[a4paper,11pt]{article}


%%% fontenc
%\usepackage{fontspec,xunicode,xltxtra}
%\setmainfont{Times New Roman}
%\setsansfont{Source Sans Pro}
%\setmonofont{Source Sans Pro}

%%% xeCJK
\usepackage{xeCJK}
\setCJKmainfont[BoldFont=Adobe Heiti Std]{Adobe Song Std}
\setCJKsansfont[BoldFont=Adobe Heiti Std]{Adobe Song Std}
\setCJKmonofont[BoldFont=Adobe Heiti Std]{Adobe Song Std}
\XeTeXlinebreaklocale "zh"
\XeTeXlinebreakskip=0pt plus 1pt minus 0.1pt

\usepackage{xcolor}
\usepackage{graphicx}

%%% get total page number
\usepackage{lastpage}

%%% customized definition
\makeatletter
\def\sybtitle#1{\def\@sybtitle{#1}}
\def\sybauthor#1{\def\@sybauthor{#1}}
\def\sybdate#1{\def\@sybdate{#1}}
\sybtitle{}
\sybauthor{}
\sybdate{}
\def\sybmaketitle{
  \begin{center}
  \vspace*{.8in}
  {\huge\bfseries\@sybtitle}
  \par
  \vspace{.8in}
  {\Large\@sybauthor}
  \par
  \vspace{.2in}
  \@sybdate
  \vspace{.5in}
  \end{center}
}
\makeatother
\setlength{\parindent}{0pt}
\renewcommand{\today}{\number\month 月 \number\day 日, ~\number\year 年}
\def\lt{\textless}
\def\gt{\textgreater}
\renewcommand\contentsname{\bfseries 目~~录}
\newcommand\bs{\texttt{\symbol{'134}}} % input backslash sign
%\newcommand\bs{\string\} % same as above definition
\long\def\cmd#1{\par\vspace{.5em}\hspace*{2em}#1\vspace{.5em}\par}
\def\cstr#1{\texttt{\string#1}} % e.g. \cstr{\latex}
\long\def\runcode#1{\par\bigskip#1\bigskip\par}
% 我不想看到那么多的underful hbox,尤其是minted环境加上背景色之后
\hbadness=10000
% 适当放宽overful hbox的限制,运行2pt的溢出
\hfuzz=2pt
\parskip=3\lineskip


%%% change background color & add frame for enumerate enviroment
\usepackage{mdframed}
\newmdenv[backgroundcolor=blue!10,linewidth=0pt]{coloredframe}
\newenvironment{coloredenumerate}{
  \begin{coloredframe}
  \begin{enumerate}
}{
  \end{enumerate}
  \end{coloredframe}
}

%%% geometry
\usepackage[includehead,includefoot,hmargin=21mm,vmargin=10.5mm,
            headsep=12pt,headheight=25pt]{geometry}
%\usepackage[includehead,includefoot,hmargin=1.2in,vmargin=1in]{geometry}

%%% fancyhdr
\usepackage{fancyhdr}
\makeatletter
\fancypagestyle{main} {
  \fancyhf{} % clear header & footer
  \fancyhead[L]{\bfseries\@sybtitle}
  \fancyhead[R]{\thepage/\pageref*{LastPage}}
  \renewcommand{\headrulewidth}{0.4pt} % header line
  \renewcommand{\footrulewidth}{0pt} % footer line
}
\fancypagestyle{header} {
  \fancyhf{} % clear header & footer
  \fancyfoot[C]{\roman{page}}
  \renewcommand{\headrulewidth}{0pt} % header line
  \renewcommand{\footrulewidth}{0pt} % footer line
}
\makeatother

\usepackage{titlesec}
\titleformat{\part}{\centering\Large\bfseries}{第\,\thepart\,部分}{1em}{}
\titleformat{\section}{\large\bfseries}{\thesection}{1em}{}
\titleformat{\subsection}{\normalsize\bfseries}{\thesubsection}{1em}{}
%\titlespacing*{章节命令}{左边距}{上文距}{下文距}[右边距]
\titlespacing*{\section}{0pt}{2\baselineskip}{\parsep}


\usepackage{hyperref}

%%% perfect source code display
\usepackage{minted}
%\usemintedstyle{colorful}
\definecolor{srcbg}{rgb}{0.95,0.95,0.95}
\newminted{java}{linenos,tabsize=4,bgcolor=srcbg}
\newminted{xml}{linenos,tabsize=4,bgcolor=srcbg}
\newminted{cpp}{linenos,tabsize=4,bgcolor=srcbg}
\newminted{bash}{linenos,tabsize=4,bgcolor=srcbg}
\newminted{latex}{linenos,tabsize=4,bgcolor=srcbg}
\newminted{scheme}{linenos,tabsize=4,bgcolor=srcbg}

\usepackage{amsmath}


\input{../styles/tikz_preamble}

\sybtitle{SQL Notes}
\sybauthor{孙延宾}
\sybdate{\today}

\begin{document}
\tt % I love Typewriter font.
%%%%%%%% the title page and toc %%%%%%%%%%
\pagestyle{header}
\sybmaketitle
\tableofcontents
\newpage

%%%%%%% the main content %%%%%%%%%
\pagestyle{main}
\setcounter{page}{1}

\part[Geting Started]{Geting Started}
\section[Create database]{Create database}
创建database并没有统一的标准,不同的软件有不同的实现方式,
比如SQLite创建database是很简单的:

\begin{bashcode}
sqlite3.exe test.db
\end{bashcode}

如果不存在test.db就创建否则就打开(open)。

\section[Create table]{Create table}
从创建table开始一切都按照标准来执行,创建table使用
CREATE TABLE命令,语法如下:

\begin{sqlcode}
CREATE TABLE <table name>
(<field 1> <data type>,
 <field 2> <data type>,
 <field 3> <data type>,
 ...
 <field N> <data type>);
\end{sqlcode}

\section[Insert data]{Insert data}
数据插入用INSERT INTO命令,语法如下:

\begin{sqlcode}
INSERT INTO <table name>
VALUES ('value1', 'value2', ...);
\end{sqlcode}

注意,valueX必须跟创建table时的fieldX对应,而且顺序一致。另外还有
一种更常用的格式:

\begin{sqlcode}
INSERT INTO <table name>
(<field 1>,
 <field 2>,
 ...
 <field N>)
VALUES
(<value for field 1>,
 <value for field 2>,
 ...
 <value for field N>);
\end{sqlcode}

不仅顺序可以自己调整,而且部分field可以忽略,不进行赋值,
这些field允许为NULL才可以,见下文的“约束”部分。
\section[Basic query]{Basic query}
数据查询使用SELECT命令,语法格式:

\begin{sqlcode}
SELECT <selection> FROM <table name>;
-- 选中所有数据
SELECT * FROM <table name>;
\end{sqlcode}


\part[Constraits-约束条件]{Constraits-约束条件}
\section[NULL约束]{NULL约束}
数据表有些字段可以禁止为NULL,可以使用NULL约束来设置,

\begin{sqlcode}
CREATE TABLE programming_language
(id INTEGER NOT NULL,
 language VARCHAR(20) NOT NULL,
 author VARCHAR(25) NOT NULL,
 year INTEGER NOT NULL,
 stardard VARCHAR(10) NULL);
\end{sqlcode}

除了国际标准这个字段之外,其他字段均不允许出现空值。
允许为NULL的字段,在数据插入的时候可以忽略掉。

\section[Primary Key约束]{Primary Key约束}
Primary Key,也叫做“主键”,它用来make each record unique,标注为
主键的字段具有唯一性,不可重复,而且最多只有一个字段可以标准为主键。

\begin{sqlcode}
CREATE TABLE programming_language
(id INTEGER NOT NULL PRIMARY KEY,
 language VARCHAR(20) NOT NULL,
 author VARCHAR(25) NOT NULL,
 year INTEGER NOT NULL,
 stardard VARCHAR(10) NULL);
\end{sqlcode}

\section[Unique Key约束]{Unique Key约束}
Unique Key也是用来标记字段的唯一性,即两条记录的unique key字段不能是一样的,
但是与Primary Key不同,我们可以将多个字段都标记为Unique Key。

\begin{sqlcode}
CREATE TABLE programming_language
(id INTEGER NOT NULL PRIMARY KEY,
 language VARCHAR(20) NOT NULL UNIQUE,
 author VARCHAR(25) NOT NULL,
 year INTEGER NOT NULL,
 stardard VARCHAR(10) NULL);
\end{sqlcode}

language标记为Unique的,所以不可能有两条记录具有相同的language字段,
除此之外,别的字段也可以同时标志位Unique的。

\section[Primary Key v.s. Unique Key]{Primary Key v.s. Unique Key}
primary key和unique key都可以标记字段的唯一性,他们的区别如下,

\begin{enumerate}
\item 最多只有一个字段可以标记为primary,但是可以同时标记多个字段为unique
\item primary字段不允许为NULL,但是Unique字段可以是NULL
\end{enumerate}

\section[外键约束]{外键约束}
外键用来关联两个表格,那么什么时候需要分割表格呢?考虑这样一种情形:
某种language的发明人有两个,A和B,那么我们在填写author字段的值的时候
改如何写呢?“A, B”、“A,B”还是别的?进一步,在query的时候又该如何查询
呢?“WHERE author='???'”,难以填写,原因在于author字段的意义不唯一,
这是应该杜绝的,解决方法就是分成两个表格,将author字段单独拿出来放在
另一个表中,那要如何关联两个表呢,通过外键。

\begin{sqlcode}
-- 创建language表格
CREATE TABLE lang_tbl
(id INTEGER NOT NULL PRIMARY KEY,
 language VARCHAR(20) NOT NULL,
 year INTEGER NOT NULL,
 standard VARCHAR(10) NULL);
-- 创建author表格
CREATE TABLE author_tbl
(id INTEGER NOT NULL,
 author VARCHAR(25) NOT NULL,
 lang_id INTEGER REFERENCES lang_tbl(id));
\end{sqlcode}

关键就在于author\_tbl最后一个字段,lang\_id,它被称作author\_tbl
的外键,既然是约束,那它到底约束的是啥呢?被作为外键的字段要求:
\begin{enumerate}
\item 外键引用的字段必须是primary或者unique的
\item 外键引用的值如果不存在就会报错
\end{enumerate}
例如,假设此时lang\_tbl中只有id为1、2、3的三条记录,此时在
author\_tbl中插入一条记录:

\begin{sqlcode}
INSERT INTO author_tbl (id, author, lang_id) VALUES (4, "John", 4);
\end{sqlcode}

它试图引用lang\_tbl中id为4的记录,此时根本不存在,所以报错。
除此之外,外键约束的字段没有任何特殊之处。

\part[操作table]{操作table}
\section[drop table]{drop table}
使用DROP TABLE命令丢弃一个table,也就是删除table,语法格式如下:

\begin{sqlcode}
DROP TABLE <table name>;
\end{sqlcode}

\section[create new table from existing tables]{create new table from existing tables}
有时需要将一个表格的某些字段集合起来构成一个新的表,使用CREATE TABLE和SELECT命令
的组合,以及新的命令AS,格式如下,

\begin{sqlcode}
CREATE TABLE <new table name> AS SELECT <selection> FROM <old table>;
-- 如,完全复制一个表
CREATE TABLE new_table AS SELECT * FROM old_table;
\end{sqlcode}

\section[modify tables]{modify tables}
修改表格,可以修改数据类型,甚至可以增加、删除字段,
使用ALTER TABLE命令,格式如下,

\begin{sqlcode}
ALTER TABLE <table name> <operation> <field with clauses>;
-- 如,将author字段的长度从25修改为30
ALTER TABLE programming_language ALTER author VARCHAR(30);
\end{sqlcode}

\section[describe table]{describe table}
描述table,也就是查看table的定义,使用DESCRIBE命令,

\begin{sqlcode}
DESCRIBE <table name>;
\end{sqlcode}

注意,不同的DBMS(Database Managing System)有不同的命令来
查看table的定义,DESCRIBE是SQL标准,但是有些DBMS并不支持,
如SQLite使用“.schema <table name>”命令来查看表格的定义。

\part[Basic Query]{Basic Query}
\section[查询“部分”字段]{查询“部分”字段}
使用SELECT命令可以只选择感兴趣的字段,

\begin{sqlcode}
SELECT language,year FROM programming_language;
\end{sqlcode}

把感兴趣的字段依次罗列出来即可。

\section[查询结果排序]{查询结果排序}
使用ORDER BY命令对查询结果进行排序,

\begin{sqlcode}
SELECT language,year FROM programming_language ORDER BY year;
-- 默认是升序排列,也可以设置为降序排列
SELECT language,year FROM programming_language ORDER BY year DESC;
\end{sqlcode}

另外,还可以使用字段的缩略格式,用数字序号代替字段名称,序号从1开始,

\begin{sqlcode}
SELECT language,year FROM programming_language ORDER BY 2 DESC;
\end{sqlcode}

因为只查询了两个字段,所以language、year的序号分别为1、2,按照出现的
顺序排序。

\section[增加查询条件]{增加查询条件}
通过上面的操作,我们已经选择出部分字段,也就是部分“列”,SELECT命令
还可以进一步过滤出我们感兴趣的“行”,通过增加WHERE命令来完成,

\begin{sqlcode}
SELECT language,author FROM programming_language WHERE standard='ANSI';
\end{sqlcode}

除了“=”之外,还可以使用“<”、“>”等操作符,

\begin{sqlcode}
SELECT language,author,year FROM programming_language
    WHERE year>1970 ORDER BY author;
\end{sqlcode}

\section[增加多个查询条件]{增加多个查询条件}
更进一步,查询条件还可以增加多个,没有限制,多个查询条件使用AND、OR、NOT来
组合,

\begin{sqlcode}
SELECT language,author,year FROM programming_language
    WHERE year>1970 OR standard IS NULL;
\end{sqlcode}


\part[处理数据]{处理数据}
\section[Insert NULL]{Insert NULL}
插入NULL数据,直接用NULL就行了,

\begin{sqlcode}
INSERT INTO programming_language VALUES (4 'Tcl', 'Ousterhout', '1988', NULL);
\end{sqlcode}

\section[插入从别的表中查询到的数据]{插入从别的表中查询到的数据}
从别的表中查询的结果直接插入新的表,这只需把INSERT和SELECT组合起来即可,

\begin{sqlcode}
INSERT INTO <table 1> SELECT language,standard FROM programming_language
    WHERE standard IS NOT NULL;
\end{sqlcode}

\section[更新数据]{更新数据}
更新数据用UPDATE命令,格式如下,

\begin{sqlcode}
UPDATE <table name> SET
<field 1> = <value 1>,
<field 2> = <value 2>,
...
<field N> = <value N>
WHERE <condition>;
\end{sqlcode}

例如,

\begin{sqlcode}
UPDATE programming_language SET year=1972, standard='ANSI'
    WHERE language='Forth';
\end{sqlcode}

注意,需要更新的数据必须存在才行。

\section[删除数据]{删除数据}
删除数据用DELETE命令,按行进行删除,格式如下,

\begin{sqlcode}
DELETE FROM <table name> WHERE <condition>;
-- 如,删除Forth语言的那一行
DELETE FROM programming_language WHERE language='Forth';
\end{sqlcode}


\part[高级查询]{高级查询}

\section[Column aliases]{Column aliases}
修改查询结果中字段的名称,使其显示为不同于字段名称的其他名称,
不影响真是的字段值,

\begin{sqlcode}
SELECT <field1> <alias1>, <field2> <alias2> ... FROM <table name>;
-- 例如,输出结果中author显示为creator
SELECT id, language, author creator FROM programming_language;
\end{sqlcode}

\section[Using LIKE operator]{Using LIKE operator}
之前已经在查询条件中使用过“=”、“IS NULL”这样的操作,这里介绍一个
“LIKE”操作符,它允许在查询条件中使用通配符:

\begin{enumerate}
\item \%: 匹配0、1、2等任意多个字符
\item \_: 匹配“一个”字符
\end{enumerate}

例如,

\begin{sqlcode}
-- 查找language以'P'开头的记录,它可以匹配
-- Perl, Prolog等
SELECT * FROM programming_language WHERE language LIKE 'P%';

-- 查询language为三个字符,且以'l'结尾的记录,如Tcl
SELECT * FROM programming_language WHERE language LIKE '__l';
\end{sqlcode}


\part[字段上的计算操作]{字段上的计算操作}
\section[字段上的数学计算]{字段上的数学计算}
标准SQL支持“+”、“-”、“*”、“/”、“%”操作,但并不仅限于此,
此时可以将字段视为变量进行操作。

\begin{sqlcode}
-- 重新计算year的值,并在结果中重命名为'decade',所以1972的计算结果为1970
SELECT language, year - (year % 10) decade FROM programming_language;
\end{sqlcode}

\section[字段上的字符串操作]{字段上的字符串操作}
字符串操作最常用的就是字符串拼接,用“||”进行拼接,不过有些DBMS使用
“+”进行拼接,由于数字可以视为字符串,所以数字可以直接用于字符串拼接,

\begin{sqlcode}
SELECT language, 'The ' || ((year/10)*10) || 's' decade FROM programming_language;
\end{sqlcode}

\section[常量-Literal]{常量-Literal}
SQL操作table,操作结果是table,对象是table,一切都是table,
我们当然可以在SQL语句中调整作为结果的table,字面值常量就是
用于此目的,它会在作为结果的table中整体插入一列,字段即为
该常量,该字段所有行的值也是该常量,字面值常量可以是string
也可以是number,

\begin{sqlcode}
SELECT language, year, 'AD' FROM programming_language;
\end{sqlcode}

编程语言的发明时间当然是公元后的事情啦!

\begin{sqlcode}
SELECT 'Total', SUM(year) FROM programming_language;
\end{sqlcode}

哦,好吧,把所有年份加起来似乎没有什么意义。


\part[聚合与重组]{统计函数与分组}
\section[统计函数:count]{统计函数:count}
如何获取table中共有多少条记录呢?使用COUNT函数,它接收一个参数,

\begin{sqlcode}
-- 表格中共有多少条记录
SELECT COUNT(*) FROM programming_language;
-- 计算某一个字段值不为NULL的记录数目
SELECT COUNT(standard) FROM programming_language;
\end{sqlcode}

\section[统计函数:distinct]{统计函数:distinct}
DISTINCT用于去除重复记录,

\begin{sqlcode}
-- 计算某一个字段非重复的记录数目
SELECT COUNT(DISTINCT standard) FROM programming_language;
\end{sqlcode}

\section[统计函数:min、max]{统计函数:min、max}
MIN、MAX用于获取某一个字段的最小、最大值,

\begin{sqlcode}
SELECT language, MAX(year) year FROM programming_language;
\end{sqlcode}

\section[统计函数:sum]{统计函数:sum}
SUM函数用于统计字段的和,

\begin{sqlcode}
SELECT 'Total', MAX(year) year FROM programming_language;
\end{sqlcode}

\section[统计函数:avg]{统计函数:avg}
AVG函数用于去平均值,

\begin{sqlcode}
-- 好吧,对年份取平均值似乎也没有多大意义!
SELECT 'Average', AVG(year) FROM programming_language;
\end{sqlcode}

\section[数据分组]{数据分组}
GROUP BY用于数据分组,它一般配合统计函数来使用,统计函数
的参数是'字段',它处理所有行的该字段,比如COUNT(standard),它
会把查询结果中所有行的'standard'字段的值计算个数,只要不是NULL
就算作一个,现在我们想知道,每一个组织都指定了几个标准,如
ANSI可能指定了3个,而ISO可能指定了2个等,此时,我们需要做的就是
首先查询standard字段,然后就结果按照standard的值进行分组,
然后计算每一组的个数,然后将最终结果返回,

\begin{sqlcode}
SELECT standard, count(standard) FROM programming_language
    GROUP BY standard;
\end{sqlcode}

这里需要注意,standard字段的行数要大于或等于分组数目,只有
standard没有重复的时候他们才相等,如果他们不相等,那么最终结果
要如何显示呢?假设国际标准组织有ANSI, ISO,但是他们制定了8个
语言的标准,此时根据standard会查询出8行,而分组之后只有2组,
那最终结果是显示2行呢还是8行呢?一般的DBMS都显示2行!

由此引出另一个问题,下面的情形,结果会显示几行呢?

\begin{sqlcode}
SELECT language, count(language) FROM programming_language;
\end{sqlcode}

根据language可能查询出8行,但是count(language)只有一个结果,
最终结果该显示几行呢?这个问题上不同的DBMS分歧就比较大了,
SQLite只显示1行,此时language显示哪一个又成了问题,SQLite
似乎是显示的最后一个,其他的DBMS一般显示8行,每个language
显示一行,第一列当然是language的值,第二列为count(language)
的值,当然每一行的第二列都是一样的。

\vspace{5pt}

另外还有一个问题,SQL标准规定GROUP BY的值只能从SELET的字段
列表中选择,而且要包含该列表中所有的字段,顺序可以自己排列,

\begin{sqlcode}
-- GROUP BY的参数只能从SELECT list中选择,而且必须全部包含
-- list中的字段
SELECT language, author, standard FROM programming_language
    GROUP BY standard, author, language;
\end{sqlcode}

但是,实际上不同的DBMS做法又不同,SQLite也实现了SQL标准,
但是同时提供了不同的实现方式,GROUP BY的参数不一定在SELECT
list中选择,也不必完全包含list中的所有字段,这在使用起来
比较方便。

\section[HAVING从句]{HAVING从句}
类似于WHERE从句可以添加查询条件,GROUP BY也有一个从句可以过滤
不同的分组,是的,分组之后我们也可以对不同的分组进行过滤,从而
只关注感兴趣的分组,HAVING从句紧跟GROUP BY。

\begin{sqlcode}
-- 对年份进行分组,然后过滤出1985年以后的年份用于进一步统计(count)
SELECT year, count(year) FROM programming_language
    GROUP BY year HAVING year > 1985;
\end{sqlcode}


\part[联合-JOIN]{联合-JOIN}
\section[什么是联合]{什么是联合}
什么是JOIN?想象这样一个场景,我们的编程语言用一个表格来记录,但是
由于一个语言可能有多个作者,为了atomonicy,我们为作者单独构建了一个
表,现在共有两个表:lang\_tbl和author\_tbl,其中author\_tbl的lang\_id
字段关联到lang\_tbl的id字段(还记得REFERENCES吗),我们要怎么进行查询
呢?SELECT提供了同时从多个表格中进行查询的功能,这就是多个表格的
“联合”(JOIN),

\begin{sqlcode}
SELECT language, author FROM lang_tbl, author_tbl
    WHERE lang_id = id;
\end{sqlcode}

这样就可以同时对两个表进行查询了。

以上是简略格式,下面是使用JOIN,ON的格式,两者功能一样,

\begin{sqlcode}
SELECT language, author FROM lang_tbl JOIN author_tbl ON lang_id = id;
\end{sqlcode}

\section[消除字段歧义]{消除字段歧义}
上面的lang\_id = id虽然没有指名各个字段归属于哪个表,但是他们含义是明确的,
因为只有author\_tbl中有lang\_id字段,只有lang\_tbl中有id字段(author\_tbl
中的primary key用author\_id表示),但是如果两个表格关联的字段名称一样该
怎么办呢,这要用到SQL的“.”操作,

\begin{sqlcode}
SELECT language, author FROM lang_tbl JOIN author_tbl
    ON author_tbl.lang_id = lang_tbl.id;
\end{sqlcode}

OK,这要意义就很明确了,另外还可以使用“表格别名”(table alias,还记得
之前的column alias吗,用法一致啊),功能是一样的,

\begin{sqlcode}
SELECT language, author FROM lang_tbl a JOIN author_tbl b
    ON a.id = b.lang_id;
\end{sqlcode}

\section[交叉联合]{交叉联合}
如果没有查询条件,即没有WHERE从句也没有ON从句,查询结果会是怎样的呢?
这叫做交叉联合,就是把所有可能的结果都给出来,也就是两个表格做
笛卡尔乘积,

\begin{sqlcode}
SELECT language, author FROM lang_tbl, author_tbl;
\end{sqlcode}

结果将会是:

language1 author1\\
language1 author2\\
language1 author3\\
language2 author1\\
language2 author2\\
language2 author3\\
language3 author1\\
...

另一种语法格式,他们功能是一样的,

\begin{sqlcode}
SELECT language, author FROM lang_tbl CROSS JOIN author_tbl;
\end{sqlcode}

\section[自关联]{自关联}
self-join,自关联,用于同一个表格中不同字段之间有关联关系的情形,
如语言A受语言B启发,语言B受语言C启发,等等,

\begin{sqlcode}
SELECT lang1.language, lang2.language influenced_by
FROM lang_tbl lang1, lang_tbl lang2
WHERE lang2.influenced_by = lang1.id;
\end{sqlcode}

\end{document}