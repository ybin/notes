% -*- coding: utf-8 -*-

\documentclass[a4paper,11pt]{article}


%%% fontenc
%\usepackage{fontspec,xunicode,xltxtra}
%\setmainfont{Times New Roman}
%\setsansfont{Source Sans Pro}
%\setmonofont{Source Sans Pro}

%%% xeCJK
\usepackage{xeCJK}
\setCJKmainfont[BoldFont=Adobe Heiti Std]{Adobe Song Std}
\setCJKsansfont[BoldFont=Adobe Heiti Std]{Adobe Song Std}
\setCJKmonofont[BoldFont=Adobe Heiti Std]{Adobe Song Std}
\XeTeXlinebreaklocale "zh"
\XeTeXlinebreakskip=0pt plus 1pt minus 0.1pt

\usepackage{xcolor}
\usepackage{graphicx}

%%% get total page number
\usepackage{lastpage}

%%% customized definition
\makeatletter
\def\sybtitle#1{\def\@sybtitle{#1}}
\def\sybauthor#1{\def\@sybauthor{#1}}
\def\sybdate#1{\def\@sybdate{#1}}
\sybtitle{}
\sybauthor{}
\sybdate{}
\def\sybmaketitle{
  \begin{center}
  \vspace*{.8in}
  {\huge\bfseries\@sybtitle}
  \par
  \vspace{.8in}
  {\Large\@sybauthor}
  \par
  \vspace{.2in}
  \@sybdate
  \vspace{.5in}
  \end{center}
}
\makeatother
\setlength{\parindent}{0pt}
\renewcommand{\today}{\number\month 月 \number\day 日, ~\number\year 年}
\def\lt{\textless}
\def\gt{\textgreater}
\renewcommand\contentsname{\bfseries 目~~录}
\newcommand\bs{\texttt{\symbol{'134}}} % input backslash sign
%\newcommand\bs{\string\} % same as above definition
\long\def\cmd#1{\par\vspace{.5em}\hspace*{2em}#1\vspace{.5em}\par}
\def\cstr#1{\texttt{\string#1}} % e.g. \cstr{\latex}
\long\def\runcode#1{\par\bigskip#1\bigskip\par}
% 我不想看到那么多的underful hbox,尤其是minted环境加上背景色之后
\hbadness=10000
% 适当放宽overful hbox的限制,运行2pt的溢出
\hfuzz=2pt
\parskip=3\lineskip


%%% change background color & add frame for enumerate enviroment
\usepackage{mdframed}
\newmdenv[backgroundcolor=blue!10,linewidth=0pt]{coloredframe}
\newenvironment{coloredenumerate}{
  \begin{coloredframe}
  \begin{enumerate}
}{
  \end{enumerate}
  \end{coloredframe}
}

%%% geometry
\usepackage[includehead,includefoot,hmargin=21mm,vmargin=10.5mm,
            headsep=12pt,headheight=25pt]{geometry}
%\usepackage[includehead,includefoot,hmargin=1.2in,vmargin=1in]{geometry}

%%% fancyhdr
\usepackage{fancyhdr}
\makeatletter
\fancypagestyle{main} {
  \fancyhf{} % clear header & footer
  \fancyhead[L]{\bfseries\@sybtitle}
  \fancyhead[R]{\thepage/\pageref*{LastPage}}
  \renewcommand{\headrulewidth}{0.4pt} % header line
  \renewcommand{\footrulewidth}{0pt} % footer line
}
\fancypagestyle{header} {
  \fancyhf{} % clear header & footer
  \fancyfoot[C]{\roman{page}}
  \renewcommand{\headrulewidth}{0pt} % header line
  \renewcommand{\footrulewidth}{0pt} % footer line
}
\makeatother

\usepackage{titlesec}
\titleformat{\part}{\centering\Large\bfseries}{第\,\thepart\,部分}{1em}{}
\titleformat{\section}{\large\bfseries}{\thesection}{1em}{}
\titleformat{\subsection}{\normalsize\bfseries}{\thesubsection}{1em}{}
%\titlespacing*{章节命令}{左边距}{上文距}{下文距}[右边距]
\titlespacing*{\section}{0pt}{2\baselineskip}{\parsep}


\usepackage{hyperref}

%%% perfect source code display
\usepackage{minted}
%\usemintedstyle{colorful}
\definecolor{srcbg}{rgb}{0.95,0.95,0.95}
\newminted{java}{linenos,tabsize=4,bgcolor=srcbg}
\newminted{xml}{linenos,tabsize=4,bgcolor=srcbg}
\newminted{cpp}{linenos,tabsize=4,bgcolor=srcbg}
\newminted{bash}{linenos,tabsize=4,bgcolor=srcbg}
\newminted{latex}{linenos,tabsize=4,bgcolor=srcbg}
\newminted{scheme}{linenos,tabsize=4,bgcolor=srcbg}

\usepackage{amsmath}


\input{../styles/tikz_preamble}

\sybtitle{Maven Notes}
\sybauthor{孙延宾}
\sybdate{\today}

\begin{document}
\tt % I love Typewriter font.
%%%%%%%% the title page and toc %%%%%%%%%%
\pagestyle{header}
\sybmaketitle
\tableofcontents
\newpage

%%%%%%% the main content %%%%%%%%%
\pagestyle{main}
\setcounter{page}{1}

\part[Basic Usage]{Basic Usage}
\section[What is maven plugin]{What is maven plugin}
A \emph{plugin} is a collection of \emph{goals} with a general common purpose.\label{sec:plugin}

Maven的所有功能都是通过插件完成的,这跟Eclipse类似,其实Maven的每个phase(阶段)
都是由一些goal组成的,

\begin{bashcode}
  mvn compiler:compile
\end{bashcode}

这个命令你可能不是很熟悉,compiler表示“compiler plugin”,
compile表示“compile goal”,组合形式就是“plugin:goal”。

\begin{bashcode}
  mvn compile
\end{bashcode}

这个命令你一定很熟悉,其实compile这个phase就是有compiler:compile组成的,也就是说
这个phase就是要完成compiler:compile这个goal。

类似的,package phase的目标为“jar:jar”(对于packaging为jar的工程)。

\section[lifecycle]{lifecycle}
The default lifecycle:

\begin{itemize}
\item validate
\item compile
\item test
\item package
\item integration-test
\item verify
\item install
\item deploy
\end{itemize}

The clean lifecycle:

\begin{itemize}
\item pre-clean
\item clean
\item post-clean
\end{itemize}

The site lifecycle:
\begin{itemize}
\item pre-size
\item site: create project site documentation.
\item post-site
\item site-deploy
\end{itemize}


\section[plugin help]{plugin help}
根据第\ref{sec:plugin}节的说明,一个plugin就是一个goal的集合,那我们如何查看
某一个plugin的说明呢,又如何得知某一个plugin又有哪些goal呢?

这些信息可以通过maven-help-plugin这个plugin来获取到,这个plugin有很多个goal,
我们一个一个介绍。

\subsection[describe]{describe}
describe就是显示plugin的说明信息以及它的goal列表。

\begin{bashcode}
  mvn help:describe -DgroupId=<xx> -DartifactId=<yy> [-Dversion=<zz>]
\end{bashcode}

或者用更简单的形式,

\begin{bashcode}
  mvn help:describe -Dplugin=<name>
  # 如,
  mvn help:describe -Dplugin=help
\end{bashcode}

其实还有更加直接的方式,一般plugin都提供help这个goal,所以不妨直接执行
这个goal,如果失败了再使用上面的help:describe来查看,

\begin{bashcode}
  mvn <plugin>:<goal>
  # help:describe不正是执行help插件的describe目标吗!
  mvn <plugin>:help
  # 例如,
  mvn help:help
  mvn eclipse:help
  mvn compiler:help
\end{bashcode}

一般来说,plugin的help都提供了进一步查询该插件的某个goal的用法的方法,

\begin{bashcode}
  mvn help:help -Ddetail=true -Dgoal=<goal>
  mvn shade:help -Ddetail=true -Dgoal=<goal>
\end{bashcode}

所以,可以用这种方法来查看某一个插件的某一个goal到底有哪些参数可以设置。

\subsection[system]{system}



\end{document}