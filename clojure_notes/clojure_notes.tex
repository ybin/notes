\documentclass[a4paper,11pt]{article}


%%% fontenc
%\usepackage{fontspec,xunicode,xltxtra}
%\setmainfont{Times New Roman}
%\setsansfont{Source Sans Pro}
%\setmonofont{Source Sans Pro}

%%% xeCJK
\usepackage{xeCJK}
\setCJKmainfont[BoldFont=Adobe Heiti Std]{Adobe Song Std}
\setCJKsansfont[BoldFont=Adobe Heiti Std]{Adobe Song Std}
\setCJKmonofont[BoldFont=Adobe Heiti Std]{Adobe Song Std}
\XeTeXlinebreaklocale "zh"
\XeTeXlinebreakskip=0pt plus 1pt minus 0.1pt

\usepackage{xcolor}
\usepackage{graphicx}

%%% get total page number
\usepackage{lastpage}

%%% customized definition
\makeatletter
\def\sybtitle#1{\def\@sybtitle{#1}}
\def\sybauthor#1{\def\@sybauthor{#1}}
\def\sybdate#1{\def\@sybdate{#1}}
\sybtitle{}
\sybauthor{}
\sybdate{}
\def\sybmaketitle{
  \begin{center}
  \vspace*{.8in}
  {\huge\bfseries\@sybtitle}
  \par
  \vspace{.8in}
  {\Large\@sybauthor}
  \par
  \vspace{.2in}
  \@sybdate
  \vspace{.5in}
  \end{center}
}
\makeatother
\setlength{\parindent}{0pt}
\renewcommand{\today}{\number\month 月 \number\day 日, ~\number\year 年}
\def\lt{\textless}
\def\gt{\textgreater}
\renewcommand\contentsname{\bfseries 目~~录}
\newcommand\bs{\texttt{\symbol{'134}}} % input backslash sign
%\newcommand\bs{\string\} % same as above definition
\long\def\cmd#1{\par\vspace{.5em}\hspace*{2em}#1\vspace{.5em}\par}
\def\cstr#1{\texttt{\string#1}} % e.g. \cstr{\latex}
\long\def\runcode#1{\par\bigskip#1\bigskip\par}
% 我不想看到那么多的underful hbox,尤其是minted环境加上背景色之后
\hbadness=10000
% 适当放宽overful hbox的限制,运行2pt的溢出
\hfuzz=2pt
\parskip=3\lineskip


%%% change background color & add frame for enumerate enviroment
\usepackage{mdframed}
\newmdenv[backgroundcolor=blue!10,linewidth=0pt]{coloredframe}
\newenvironment{coloredenumerate}{
  \begin{coloredframe}
  \begin{enumerate}
}{
  \end{enumerate}
  \end{coloredframe}
}

%%% geometry
\usepackage[includehead,includefoot,hmargin=21mm,vmargin=10.5mm,
            headsep=12pt,headheight=25pt]{geometry}
%\usepackage[includehead,includefoot,hmargin=1.2in,vmargin=1in]{geometry}

%%% fancyhdr
\usepackage{fancyhdr}
\makeatletter
\fancypagestyle{main} {
  \fancyhf{} % clear header & footer
  \fancyhead[L]{\bfseries\@sybtitle}
  \fancyhead[R]{\thepage/\pageref*{LastPage}}
  \renewcommand{\headrulewidth}{0.4pt} % header line
  \renewcommand{\footrulewidth}{0pt} % footer line
}
\fancypagestyle{header} {
  \fancyhf{} % clear header & footer
  \fancyfoot[C]{\roman{page}}
  \renewcommand{\headrulewidth}{0pt} % header line
  \renewcommand{\footrulewidth}{0pt} % footer line
}
\makeatother

\usepackage{titlesec}
\titleformat{\part}{\centering\Large\bfseries}{第\,\thepart\,部分}{1em}{}
\titleformat{\section}{\large\bfseries}{\thesection}{1em}{}
\titleformat{\subsection}{\normalsize\bfseries}{\thesubsection}{1em}{}
%\titlespacing*{章节命令}{左边距}{上文距}{下文距}[右边距]
\titlespacing*{\section}{0pt}{2\baselineskip}{\parsep}


\usepackage{hyperref}

%%% perfect source code display
\usepackage{minted}
%\usemintedstyle{colorful}
\definecolor{srcbg}{rgb}{0.95,0.95,0.95}
\newminted{java}{linenos,tabsize=4,bgcolor=srcbg}
\newminted{xml}{linenos,tabsize=4,bgcolor=srcbg}
\newminted{cpp}{linenos,tabsize=4,bgcolor=srcbg}
\newminted{bash}{linenos,tabsize=4,bgcolor=srcbg}
\newminted{latex}{linenos,tabsize=4,bgcolor=srcbg}
\newminted{scheme}{linenos,tabsize=4,bgcolor=srcbg}

\usepackage{amsmath}


\input{../styles/tikz_preamble}

\sybtitle{Clojure Notes}
\sybauthor{孙延宾}
\sybdate{\today}

\begin{document}
  \tt % I love Typewriter font.
%%%%%%%% the title page and toc %%%%%%%%%%
  \pagestyle{header}
  \sybmaketitle
  \tableofcontents
  \newpage

%%%%%%% the main content %%%%%%%%%
  \pagestyle{main}
  \setcounter{page}{1}

  \section[make symbolic list]{make symbolic list}
  定义一个macro,该macro生成一个symbol列表。listq之于list,
  恰好是Emacs lisp中的setq之于set。

  \begin{schemecode}
    (defmacro listq
    "make a symbol list.
    
    listq vs list, just like setq vs set in Emacs lisp."
    [& args]
    `'~args)    
  \end{schemecode}

  注意"`"(syntax-quote),该语法应用于clojure reader,是在read或者编译时使用的,
  不是执行期使用的。

  \section[高阶函数举例]{高阶函数举例}
  常用高阶函数:

  \subsection[function map]{function map}
  语法:\par
  (map fn c1 \& coll)

  map函数接收一个函数以及任意多个collection作为参数,返回一个列表,要求函数fn的
  参数个数与collections个数相等,map依次从各个colls中取出一个元素,
  然后把这些元素作为参数传递给fn,返回值最为最终结果(一个列表)的元素,
  如果colls长度不一致,map按照最短的coll计算。

  例如,向量加法(向量相加还是向量):\par
  (map + [1 2 3 4] [2 3 4 1] [3 4 1 2] [4 3 2 1])\par
  四个向量相加。

  \subsection[function apply]{function apply}
  语法:\par
  (apply fn coll)

  将coll的所有元素”一次性“传递给fn,然后返回fn的结果。

  例如,将一个coll中的所有元素相加:\par
  (apply + [1 2 3 4])
  
  \subsection[function reduce]{function reduce}
  语法:\par
  (reduce fn coll)\\
  (reduce fn seed coll)

  函数fn必须可以接收两个参数,如果seed没有提供,就取coll的前两个元素作为
  fn的参数,返回值和第三个参数再次传入fn,依次类推,直到遍历coll。如果提供seed,
  就取seed作为第一个参数,coll的第一个元素作为第二个参数传入fn,返回值和第二个元素
  再次传入fn,依次类推直到遍历coll。

  例如,求向量的模($\sqrt{x_1^2+x_2^2+x_3^3+\cdots+x_n^2}$):\par
  (reduce \#(+ \%1 (* \%2 \%2)) 0 [1 2 3 4])
  
  \subsection[function filter]{function filter}
  语法:\par
  (filter predicate-fn coll)

  filter函数依次取coll的元素,传递给predicate-fn,如果该函数返回true
  就留下这个元素,否则就丢掉,结果是一个所有元素都满足predicate-fn的列表。

  例如,取出所有的偶数:\par
  (filter enven? [1 2 3 4 5 6])

  \section[使用自定义的java库和clojure库]{使用自定义的java库和clojure库}
  clojure可以轻松使用自定义的Java库和自定义的Clojure库。
  \subsection[自定义Java库]{自定义Java库}
  一般java库都是以.jar文件的形式发布,从clojure调用自定义java库只需:

  将.jar文件放到classpath路径下,如启动repl时使用参数-cp加载,标准的java方式均可,
  然后从clojure代码直接调用即可,如:\\
  (. org.ybin.Util greetFromJava)\\
  调用org.ybin这个java package里的Util这个类的名为greetFromJava的静态方法。

  \subsection[自定义clojure库]{自定义clojure库}
  自定义clojure库同样很简单,只需:

  1. 将clj文件,如core.clj,放到classpath下的目录中\\
  2. 设置正确的路径,如:我把sybUtil/这个目录加到classpath中,
  然后在sybUtil/org/ybin/目录下放置core.clj文件,该文件中定义
  namespace为(ns org.ybin.core),一个文件即为一个namespace,
  使用时先require:(require 'org.ybin.core),然后调用:(org.ybin.core/greet-from-clj),调用该
  namespace中的greet-from-clj函数。

  \section[clojure访问java enum元素]{clojure访问java enum元素}
  java enum定义:\par
  \begin{javacode}
public class JCLASS {
  static MyEnum {
    FIRST,
    SECOND,
    THIRD
  }
}
  \end{javacode}

  clojure访问java enum:\par
(println JCLASS\$MyEnum/FIRST)


  \section[clojure.java.io]{clojure.java.io}
  clojure中如何处理java io。java中reader, writer与input stream, output stream的
  区别主要是encode与decode,前者面向“字符流”后者面向“字节流”,后者不进行codec。
  
  \subsection[reader]{reader}
  reader默认返回一个java.io.BufferedReader。
  它可以读取Reader, BufferedReader, InputStream, File,
  URI, URL, Socket, byte arrays, character arrays以及String.

  如果参数为一个字符串,reader首先尝试按照URI解析,再尝试按照
  local file name解析。

  reader应该在with-open中使用,以确保其被正确关闭。

  示例:\\
  \begin{schemecode}
    (with-open [rdr (clojure.java.io/reader "http://www.baidu.com")]
      (println (clojure.string/join "\n" (line-seq rdr))))
  \end{schemecode}

  另一个函数clojure.core/slurp跟reader接受同样的参数,但是它不会缓开的内容,
  而是一次性把所有内容读入内存,然后返回一个字符串(注意编码方式),之后关闭文件
  所以当内容不被考虑的情况下,可以使用slurp。
  
  \begin{schemecode}
    (println (str (slurp "http://www.baidu.com" :encoding "UTF-8")))
  \end{schemecode}

  于此对应的是clojure.core/spit,它与slurp正好相反,它使用writer将内容写入文件,
  然后关闭文件。

  \begin{schemecode}
    (spit "c:/log.txt" "this is log content.")
  \end{schemecode}

  spit接收optional参数,比如:append等,所有writer接收的参数均可。

  名词翻译:slurp - 出声的吃、喝; spit - 吐,吐出。


  \subsection[writer]{writer}

  \subsection[input-stream]{input-stream}

  \subsection[output-stream]{output-stream}

  \subsection[file]{file}

  \subsection[delete-file]{delete-file}

  \subsection[make-parents]{make-parents}


  \section[clojure source code]{clojure source code}
  clojure源码中的Axxx文件中的A表示Abstract,用于表示抽象类;
  Ixxx文件中的I表示Interface,用于表示接口类。


  \section[function into]{function into}
  函数into的语法:\par
  \begin{schemecode}
    (into to-coll from-coll)
  \end{schemecode}

  使用into可以在coll之间进行类型转换,比如将list转为vector:\par
  \begin{schemecode}
    (into [] '(1 2 3 4))
  \end{schemecode}

  但是,需要注意的是hash-map,into处理的是sequence,而hash-map是key-value pair组成的
  sequence,而这个key-value pair本质上是vector,所以不能直接将list或者vector into
  到hash-map,而要将key-value pair组成的vector或者list转换为hash-map,如:\par
  \begin{schemecode}
    ; convert hash-map to vector
    (into [] {:a 1 :b 2})
    
    ; convert vector to hash-map, ***WRONG***
    (into {} [:a 1 :b 2])

    ; convert vector to hash-map, ***RIGHT***
    (into {} [[:a 1] [:b 2]]
  \end{schemecode}

  \section[使用for遍历固定深度的tree]{使用for遍历固定深度的tree}
  假设树的深度为3,树用hash-map来表示:\par
  \begin{schemecode}
    (def tree
      {:root
        {:a
          {:A 1, :B 2, :C 3}
         :b
          {:D 4, :E 5, :F 6}
         :c
          {:G 7, :H 8, :I 9}}})
  \end{schemecode}

  然后遍历输出所有叶子节点的值:\par
  \begin{schemecode}
    (for [[k v] tree
          [k2 v2] v
          [k3 v3] v2]
      v3)    
  \end{schemecode}

  \section[任意精度数字运算]{任意精度数字运算}
  clojure的四则运算中,如果参数有至少一个超过java.lang.Integer的范围,则四则运算
  自动提升到BigInt进行运算;如果没有参数超过java.lang.Integer,但是运算结果却
  超出了Integer的范围,则抛出IntOverflow异常,四则运算不会自动进行类型提升,而相应的
  "+', -', *'"(除法没有该版本)则强制进行类型提升。

  所以,像quot, rem等运算是不用担心IntOverflow的,因为他们的运算结果只会比参数更小,
  不会存在结果比参数还要大的情况。

  clojure支持任意精度数字运算,即BitInt,其四则运算为:\par
  \begin{schemecode}
    +', -', *'
  \end{schemecode}

  例如:\par
  \begin{schemecode}
    (apply *' (range 1 101))
  \end{schemecode}

  \section[lazy sequence]{lazy sequence}
  如何使用lazy-seq函数创建lazy sequence,一般的用法如下:\par
  \begin{schemecode}
    (defn fib
      [a b]
      (cons a (lazy-seq (fib b (+ b a))))
  \end{schemecode}

  如上定义了一个斐波那契数列。

  可见,lazy-seq一般配合"递归"来定义lazy sequence。
  下面使用lazy-seq定义杨辉三角(Pascal's Triangle)。

  \begin{schemecode}
(fn pascal-triangle
  [coll]
  (cons coll (lazy-seq (pascal-triangle ((fn [c]
        (conj (into [(first c)]
          (map #(apply +' %) (partition 2 1 c))) (last c))) coll)))))
  \end{schemecode}

  一个更加简洁的实现方式:\par
  \begin{schemecode}
(defn yanghui-triangle
  [coll]
  (iterate
   (fn [[x & xs :as all]]
     (flatten
      [x (map (partial apply +)
              (partition 2 1 all)) (last all)]))
   coll))

(take 10 (yanghui-triangle [1]))
  \end{schemecode}

  结果:\par
  \begin{schemecode}
([1]
 (1 1)
 (1 2 1)
 (1 3 3 1)
 (1 4 6 4 1)
 (1 5 10 10 5 1)
 (1 6 15 20 15 6 1)
 (1 7 21 35 35 21 7 1)
 (1 8 28 56 70 56 28 8 1)
 (1 9 36 84 126 126 84 36 9 1))
  \end{schemecode}

  注意iterate和partial的用法。


  \section[从一个题目中体会算法]{从一个题目中体会算法}
  题目:\par
  写一个函数:f(N, p, q),求所有小于N,并能被p或者q整除的数之和,其中p、q互素。

  方案一:遍历、判断、求和。\par
  但是当N很大时,这是不可行的。

  方案二:利用等差数列已经互素的性质。\par
  “所有p的小于N的倍数之和”+“所有q的小于N的倍数之和”-“所有p*q的小于N的倍数之和”。

  三部分均为等差数列,所以可以利用等差数列公式,常数时间、常数空间内即可完成计算。


  \section[关于Macro]{关于Macro}
  defmacro的body部分展开解释。
  
  \begin{schemecode}
    (defmacro my-defn
      [name & body]
      `(defn ~name [] ~@body))
  \end{schemecode}

  然后开始使用该宏来定义函数:

  \begin{schemecode}
    (my-defn foo (println "hello"))
    ; 展开
    ; (def foo (clojure.core/fn ([] (println "hello"))))
  \end{schemecode}

  很好,一切正常!继续:

  \begin{schemecode}
    (def fbody '(println "hello"))
    (my-defn bar fbody)
    ; 展开
    ; (def foo (clojure.core/fn ([] fbody)))
  \end{schemecode}

  看到比较结果了吗?可能跟想要的结果不一样,但是想想defn的定义,符合逻辑,
  函数体就是fbody而不是(println "hello"),不是吗?!


  \section[体会递归、面向数据]{体会递归、面向数据}
  写一个简单的计算器,根据给定的变量表计算表达式的值。
  该计算器必须通过下列测试:\par
  \begin{schemecode}
;;; test
(= 2 ((__ '(/ a b))
      '{b 8 a 16}))
(= [6 0 -4]
     (map (__ '(* (+ 2 a)
                     (- 10 b)))
          '[{a 1 b 8}
            {b 5 a -2}
            {a 2 b 11}]))
  \end{schemecode}
  
  计算器实现如下:\par
  \begin{schemecode}
(defn parse
  [expr]
  (fn evaluate
    [vars]
    (cond
     (list? expr) (apply ({'+ +, '- -, '* *, '/ /} (first expr))
                         (map #((parse %) vars) (rest expr)))
     (vars expr) (vars expr)
     :else expr)))
  \end{schemecode}

  注意:如何不使用eval函数运行list变量,lisp对list的处理只会发生在“字面量”上,
  即解析代码的时候,如果一个list变量里面装的是“代码”,即数据即代码,你明知这个
  list可以被解析执行,但是lisp不会解析它,lisp只会将其视为普通的list变量,事实上,
  它就是一个list变量而已。如果要解析执行list,唯一的方式是使用eval函数,但是
  eval is bad! 那只好手动解析每个函数,不过好在lisp的函数调用规则及其简单:
  第一个是函数,剩下的都是参数。
  
  还有一种通过eval函数实现的方式,这种方式可能会让结构更加明了:

  \begin{schemecode}
(defn computer
  [expr]
  (let [parser (fn [v]
                 (map (fn mapper [form]
                        (cond
                         (list? form) (map mapper form)
                         (v form) (v form)
                         :else form))
                      expr))]
    (fn interpreter [vars]
      (eval (parser vars)))))
  \end{schemecode}

  上面的测试仍然可以通过,而且所有函数直接使用clojure的内置函数库,
  整个结构也比较清晰:\par
  给computer传递一个expr,它生成一个interpreter(通过内部的parser生成),
  然后给interpreter一个变量表,它便给出最终的计算结果,逻辑很清晰。
  唯一的缺点便是使用了eval函数,而eval is bad!


  \section[操作序列的index]{操作序列的index}
  有时序列的index是重要的,比如将0-1组成的字符串转换为整数,clojure提供了
  map和keep的indexed版本:map-indexed, keep-indexed\par
  map-indexed与map类似,区别在于操作函数,map的操作函数接受一个参数,
  而map-indexed接受两个参数:index, item,如将0-1字符串转换为整数。

  \begin{schemecode}
;;; map-indexed接受的函数为f(index, item)
(defn binary-to-int
  [string]
  (reduce + 0
          (map-indexed #(cond
                         (= %2 \0) 0
                         :else (int (Math/pow 2 %1)))
                       (reverse string))))
  \end{schemecode}

  keep-indexed于此类似,其接受的函数也是:f(index, item)


  \section[function apply]{apply函数}
  apply的语法:

  \begin{schemecode}
    (apply x y z arg-list)
  \end{schemecode}

  其中arg-list为collection/seq。

  apply的应用场景:

  当我们的函数需要多个参数,但是这些参数是通过一个collection/seq
  的形式给出的,形式上来看,我们需要把这个collection/seq解构为
  独立的参数,apply就是用来做这个事情的。

  实例:

  \begin{schemecode}
    ;;; args显示是一个list,但是+需要的是独立的任意多个参数,
    ;;; 于是我们需要使用apply取出list的内容,然后交给+来处理。
    (defn my-add
      [& args]
      (apply + args))

    ;;; 又比如,max, min的参数也是任意多个独立参数,当我们需要
    ;;; 找出一个collection/seq中的最大、最小值时:
    (apply max [0 1 2 3])
    (apply min [0 1 2 3])
  \end{schemecode}


  \section[使用complement函数]{使用complement函数}
  complement函数以函数为参数并且返回一个函数,两个函数返回的布尔真值互反。
  
  \begin{schemecode}
(defn print-coll
  [coll]
  (if (not (empty? coll))
    (apply println coll)
    (println "Empty collection!")))

;;; 使用complement函数
(defn print-coll-2
  [coll]
  (if ((complement empty?) coll)
    (apply println coll)
    (println "Empty collection!")))
  \end{schemecode}


  \section[-> and ->>]{-> and ->>}
  ->和->>很相似,他们可以方便的用于嵌套的函数调用。

  例如:当$x=3$时,求$(x+5-1)*2$

  \begin{schemecode}
(* (- (+ 3 5) 1) 2)
;;; 相比于-> macro:
(-> 3 (+ 5) (- 1) (* 2))
  \end{schemecode}

  ->与->>的区别为:->将参数插入表达式的“第二个”位置,
  而->>将参数插入表达式的“最后”。

  \begin{schemecode}
;;; 结果为:14
(-> 3 (+ 5) (- 1) (* 2))

;;; 结果为:-14
(->> 3 (+ 5) (- 1) (* 2))
  \end{schemecode}


  \section[分裂collection]{分裂collection}
  很多时候我们需要将一个coll分裂为多个部分,如将重复的内容分组:

  [1 1 1 2 3 3 4 4 4 4 5] => [[1 1 1] [2] [3 3] [4 4 4 4] [5]]

  根据各个小组的情况不同,clojure提供了不同的函数,如果每个小组大小相等,
  我们可以使用partition函数:

  \begin{schemecode}
;;; 结果为:((1 2) (3 4))
(partition 2 [1 2 3 4])
  \end{schemecode}

  如果分组大小不同,如上面的情况,可以使用partition-by函数:

  \begin{schemecode}
;;; 如果f返回的值变化了,那么partition-by就知道“该另起一组”了!
(partition-by f coll)
(partition-by identity [1 1 1 2 3 3 4 4 4 4 5]
  \end{schemecode}

  实例:\par
  当我们想“压缩”一个collection时可以使用这两个函数:

  \begin{schemecode}
;;; [1 1 1 2 3 3 4 4 4 4 5] => (1 2 3 4)
(defn compress-colletion
  [coll]
  (map first (partition-by identity coll)))
  \end{schemecode}


  \section[元素分类]{元素分类}
  将collection中的所有元素按照某个标准分类:group-by

  \begin{schemecode}
(group-by pred coll)
  \end{schemecode}

  group-by遍历coll,按照pred函数的返回值将所有元素分类,
  pred有几种返回值元素就会被分为几类,结果保存到一个map中,
  以pred的返回值作为key,分类好的元素构成vector并作为key的值。

  \begin{schemecode}
; e.g. 1
(group-by type [1 :a 2 :b "a" "b"])
; 结果:
;{java.lang.Long [1 2],
; clojure.lang.Keyword [:a :b],
; java.lang.String ["a" "b"]}

; e.g. 2
(group-by identity [1 1 2 3 2 1 1])
; 结果:
; {1 [1 1 1 1], 2 [2 2], 3 [3]}

; e.g. 3
; 计算元素出现的次数
(defn count-occurrences
  [coll]
  (into {}
    (for [[k v]] (group-by identity coll)
      [k (count v)])))

; e.g. 4
; 使用reduce的例子:去除重复元素
; [1 2 1 3 1 4 5] => [1 2 3 4 5]
(defn remove-duplacate
  [coll]
  (reduce #(if ((set %1) %2)
               %1
               (conj %1 %2))
          [] coll))
; reduce的作用就是“缩减”,当然包括将多个元素缩减为一个collection。

; e.g. 5
; 仍然是reduce的例子:反作用
; reduce一般是将函数作用到集合的每个元素上,但是反过来,集合的元素也可以是函数,
; 这样就可以来个“反作用”:例如重写comp函数。
(defn my-comp
  [& fs]
  (fn [& args]
    (let [ret (apply (last fs) args)
          fs (rest (reverse fs))]
      (reduce #(%2 %1) ret fs))))
  \end{schemecode}
  

  \section[函数类比]{函数类比}
  iterate函数对应到数学里面的类比:
  
  \begin{schemecode}
(iterate f init)
  \end{schemecode}
  $$f(...f(f(f(init)))...)$$

  reduce函数对应到数学里面的类比:
  
  \begin{schemecode}
(reduce f init [0 1 2 3 4 5])
  \end{schemecode}
  $$f(f(f(f(f(f(init, 0), 1), 2), 3), 4), 5)$$


  \section[lazy-seq]{lazy-seq}
  lazy-seq似乎很神奇,的确!

  \subsection[如何创建lazy-seq?]{如何创建lazy-seq?}
  首先考虑直接使用内置函数,如:\par
  repeat, repeatly, iterate, range, map, filter, remove, etc.
  
  \subsection[lazy-seq函数如何使用?]{lazy-seq函数如何使用?}
  如果内置函数无法完成工作,就直接使用lazy-seq函数吧:

  \begin{schemecode}
; body必须返回一个seq或者nil
(lazy-seq
  [& body])
  \end{schemecode}
  
  当lazy-seq第一次调用时,如(first my-lazy-seq),lazy-seq将body产生的序列返回给
  first,并且将该序列缓存来以供后续使用,下次再调用lazy-seq时就不会再执行body了。

  要创建一个lazy-seq必须使用递归!

  \begin{schemecode}
(defn my-reductions
  ([f [x & xs]] (reds f x xs))
  ([f init coll]
     (lazy-seq
      (cons init
            (when-let [[first & rest] (seq coll)]
              (my-reductions f (f init first) rest))))))
; 1. lazy-seq返回一个序列,当使用该序列时,如(ffirst (my-reductions + (range))),
;    body被执行生成一个序列(cons函数),该序列的rest部分仍然是一个lazy-seq,否则,
;    第一次使用时序列即被“全部展开”,那么该序列就不是lazy seq了。
; 2. ffirst首先取出一个元素,此时lazy-seq返回一个列表:
;      (init (lazy-seq))
;    该序列的rest部分仍然是一个lazy seq,它是由my-reductions递归调用产生的,
;    序列展开又被暂停了。
; 3. ffirst取出第二个元素,此时rest部分又被展开:
;      (init (init lazy-seq))
;    rest的rest部分又是一个lazy seq,它仍然由my-reductions递归调用产生,
;    序列展开又被暂停了。
; 循环往复,这个惰性序列用几次就被展开几次,也就是说,用一个它就计算一个。
; 当然,也可以用一个它计算多个,比如32个,这样每隔32个才计算一次,效率会有提升。
  \end{schemecode}
  
  \section[区分list和vector]{区分list和vector}
  对于任意coll,已知coll为list或者vector,

  \begin{schemecode}
(= (cons :type coll) (conj coll :type))
  \end{schemecode}

  如果返回true是否一定说明coll是list呢?答案是:否!

  反例:\par
  \begin{schemecode}
(def coll
  [:type :type])
; 或者
(def coll
  [])
  \end{schemecode}

  那么该如何区分list和vector呢?
  
  \begin{schemecode}
(let [c (conj coll :a)]
  (= (cons :type coll) (conj coll :type))
  \end{schemecode}

  对于vector, 两种返回值为:\par
  :type coll :a 和 coll :a :type

  注意结尾处,两者一定是不相等的!


  \section[Get var by name string]{Get var by name string}
  目的:name string => var/value of var

  \begin{schemecode}
;;; 获取name string对应的var
(resolve (symbol "first"))
; or
(-> "first" symbol resolve)
;;; 结果为name string对应的var

;;; 附带取var的值
(def my-var 42)
(-> "my-var" symbol resolve)
; 结果:#<Var@13e6f83: 42>
@(-> "my-var" symbol resolve)
; 结果:42

;;; 思考:function的value是什么呢?
  \end{schemecode}


  \section[Listing special forms]{Listing special forms}
  列出所有的special forms:

  \begin{schemecode}
(keys clojure.lang.Compiler/specials)
  \end{schemecode}


  \section[正则表达式]{正则表达式}
  clojure中的正则表达式使用方法总结。

  \subsection[re-pattern]{re-pattern}
  使用正则表达式首先要创建一个正则表达式对象(java.util.regex.Pattern),

  \begin{schemecode}
;;; \d 表示digit,+表示一个或者多个,\\用来转义"\",
;;; re-pattern创建pattern时必须转义。
(re-pattern "\\d+")

;;; 另外更常用的是reader macro的形式,这种方式不需要进行转义,如
#"\d+"
  \end{schemecode}

  \subsection[re-find]{re-find}
  有了pattern对象,我们就可以进行查找了:re-find

  re-find有两种格式,其结果是一样的,

  \begin{schemecode}
(re-find matcher)
;;; 什么是matcher呢,见下一节。
(re-find re s)
;;; 示例
(re-find #"\d+" "123A456B789C")
  \end{schemecode}
  
  \subsection[re-matcher]{re-matcher}
  matcher(java.util.regex.Matcher)就是Pattern对象和查找对象的组合:

  \begin{schemecode}
(def matcher (re-matcher #"\d+" "123A456B789C"))
;;; 使用matcher
(re-find matcher)
  \end{schemecode}

  问题:为何要使用matcher呢?

  答案:matcher有时是必须的,考虑从"123A456B789C"中找出所有数字的例子,

  \begin{schemecode}
(re-find #"\d+" "123A456B789C")
;;; 运行结果:"123"
(re-find #"\d+" "123A456B789C")
;;; 运行结果:"123"
(re-find #"\d+" "123A456B789C")
;;; 运行结果:"123"
  \end{schemecode}

  事与愿违不是吗?我们本想找到所有三处匹配,但是每次都只能找到第一处,
  此时就需要matcher了。matcher对象内部维护一个状态(bad!),它相当于一个
  迭代器,运行一次re-find,它就迭代一次,知道找到所有匹配。

  \begin{schemecode}
(def matcher (re-matcher #"\d+" "123A456B789C"))

(re-find matcher)
;;; 运行结果:"123"
(re-find matcher)
;;; 运行结果:"456"
(re-find matcher)
;;; 运行结果:"789"
  \end{schemecode}

  这就是matcher的作用!

  \subsection[re-groups]{re-groups}
  使用matcher时,re-find可不能乱用,因为它是迭代进行的,可是有时
  我们的确想多次使用当前的匹配结果(当然你可以增加临时变量)而不
  进行迭代,此时就要re-groups出场了,它只是返回"当前/最新"的匹配结果,
  而不进行迭代。

  \begin{schemecode}
;;; 定义matcher
(def matcher (re-matcher #"\d+" "123A456B789C"))
;;; 当前/最新的匹配结果是什么呢?
(re-groups matcher)
; 结果:IllegalStateException No match found  java.util.regex.Matcher.group (:-1)
; 为何如此呢?因为当前的确没有任何匹配,re-groups就报错了,需要首先运行re-find
; 找到一处匹配才行。
(re-find matcher)
; 结果:"123"
(re-groups matcher)
; 结果:"123"
(re-groups matcher)
; 结果:"123",仍然是"123"
(re-find matcher)
; 结果:"456"
(re-groups matcher)
; 结果:"456"
(re-find matcher)
; 结果:"789"
(re-find matcher)
; 结果:nil。匹配完了!
(re-groups matcher)
; 结果:IllegalStateException No match found  java.util.regex.Matcher.group (:-1)
; 没有匹配结果,更第一次运行re-groups一样的结果:异常!
  \end{schemecode}

  \subsection[re-seq]{re-seq}
  上面说的re-find的多次匹配情况是很"Java"的方式,并不是clojure的方式,真正的clojure
  的方式是使用re-seq,

  \begin{schemecode}
(re-seq #"\d" "clojure 1.1.5")
; 结果: ("1" "1" "5")
  \end{schemecode}

  re-seq返回一个惰性列表,它保存了所有的匹配结果。

  \subsection[分组匹配模式]{分组匹配模式}
  分组匹配模式是很重要的,如找到字母A两端的数字,

  \begin{schemecode}
(re-find #"(\d+)A(\d+)" "123A456B789C")
; 结果:["123A456" "123" "456"]
;;; 首先把整体的匹配结果列出来,然后紧接着是分组的结果。
  \end{schemecode}

  \subsection[re-matches]{re-matches}
  另外还有一个re-matches函数,它有点儿不同,re-find意在"寻找"匹配的模式,
  结果是找到"所有"的匹配,但是re-matches意在"判断"目标跟正则表达式是否匹配,
  如果匹配就返回目标,否则返回nil。

  \begin{schemecode}
(re-matches #"hello" "hello world")
; 结果:nil
(re-matches #"hello.*" "hello world")
; 结果:"hello world"
(re-matches #"hello(.*)" "hello world")
; 结果:["hello world" "world"]
  \end{schemecode}

  


  
  
  
   The seq function yields an implementation of ISeq appropriate to the collection. 
\end{document}