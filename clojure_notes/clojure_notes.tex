\documentclass[a4paper,11pt]{article}


%%% fontenc
%\usepackage{fontspec,xunicode,xltxtra}
%\setmainfont{Times New Roman}
%\setsansfont{Source Sans Pro}
%\setmonofont{Source Sans Pro}

%%% xeCJK
\usepackage{xeCJK}
\setCJKmainfont[BoldFont=Adobe Heiti Std]{Adobe Song Std}
\setCJKsansfont[BoldFont=Adobe Heiti Std]{Adobe Song Std}
\setCJKmonofont[BoldFont=Adobe Heiti Std]{Adobe Song Std}
\XeTeXlinebreaklocale "zh"
\XeTeXlinebreakskip=0pt plus 1pt minus 0.1pt

\usepackage{xcolor}
\usepackage{graphicx}

%%% get total page number
\usepackage{lastpage}

%%% customized definition
\makeatletter
\def\sybtitle#1{\def\@sybtitle{#1}}
\def\sybauthor#1{\def\@sybauthor{#1}}
\def\sybdate#1{\def\@sybdate{#1}}
\sybtitle{}
\sybauthor{}
\sybdate{}
\def\sybmaketitle{
  \begin{center}
  \vspace*{.8in}
  {\huge\bfseries\@sybtitle}
  \par
  \vspace{.8in}
  {\Large\@sybauthor}
  \par
  \vspace{.2in}
  \@sybdate
  \vspace{.5in}
  \end{center}
}
\makeatother
\setlength{\parindent}{0pt}
\renewcommand{\today}{\number\month 月 \number\day 日, ~\number\year 年}
\def\lt{\textless}
\def\gt{\textgreater}
\renewcommand\contentsname{\bfseries 目~~录}
\newcommand\bs{\texttt{\symbol{'134}}} % input backslash sign
%\newcommand\bs{\string\} % same as above definition
\long\def\cmd#1{\par\vspace{.5em}\hspace*{2em}#1\vspace{.5em}\par}
\def\cstr#1{\texttt{\string#1}} % e.g. \cstr{\latex}
\long\def\runcode#1{\par\bigskip#1\bigskip\par}
% 我不想看到那么多的underful hbox,尤其是minted环境加上背景色之后
\hbadness=10000
% 适当放宽overful hbox的限制,运行2pt的溢出
\hfuzz=2pt
\parskip=3\lineskip


%%% change background color & add frame for enumerate enviroment
\usepackage{mdframed}
\newmdenv[backgroundcolor=blue!10,linewidth=0pt]{coloredframe}
\newenvironment{coloredenumerate}{
  \begin{coloredframe}
  \begin{enumerate}
}{
  \end{enumerate}
  \end{coloredframe}
}

%%% geometry
\usepackage[includehead,includefoot,hmargin=21mm,vmargin=10.5mm,
            headsep=12pt,headheight=25pt]{geometry}
%\usepackage[includehead,includefoot,hmargin=1.2in,vmargin=1in]{geometry}

%%% fancyhdr
\usepackage{fancyhdr}
\makeatletter
\fancypagestyle{main} {
  \fancyhf{} % clear header & footer
  \fancyhead[L]{\bfseries\@sybtitle}
  \fancyhead[R]{\thepage/\pageref*{LastPage}}
  \renewcommand{\headrulewidth}{0.4pt} % header line
  \renewcommand{\footrulewidth}{0pt} % footer line
}
\fancypagestyle{header} {
  \fancyhf{} % clear header & footer
  \fancyfoot[C]{\roman{page}}
  \renewcommand{\headrulewidth}{0pt} % header line
  \renewcommand{\footrulewidth}{0pt} % footer line
}
\makeatother

\usepackage{titlesec}
\titleformat{\part}{\centering\Large\bfseries}{第\,\thepart\,部分}{1em}{}
\titleformat{\section}{\large\bfseries}{\thesection}{1em}{}
\titleformat{\subsection}{\normalsize\bfseries}{\thesubsection}{1em}{}
%\titlespacing*{章节命令}{左边距}{上文距}{下文距}[右边距]
\titlespacing*{\section}{0pt}{2\baselineskip}{\parsep}


\usepackage{hyperref}

%%% perfect source code display
\usepackage{minted}
%\usemintedstyle{colorful}
\definecolor{srcbg}{rgb}{0.95,0.95,0.95}
\newminted{java}{linenos,tabsize=4,bgcolor=srcbg}
\newminted{xml}{linenos,tabsize=4,bgcolor=srcbg}
\newminted{cpp}{linenos,tabsize=4,bgcolor=srcbg}
\newminted{bash}{linenos,tabsize=4,bgcolor=srcbg}
\newminted{latex}{linenos,tabsize=4,bgcolor=srcbg}
\newminted{scheme}{linenos,tabsize=4,bgcolor=srcbg}

\usepackage{amsmath}


\input{../styles/tikz_preamble}

\sybtitle{Clojure Notes}
\sybauthor{孙延宾}
\sybdate{\today}

\begin{document}
  \tt % I love Typewriter font.
%%%%%%%% the title page and toc %%%%%%%%%%
  \pagestyle{header}
  \sybmaketitle
  \tableofcontents
  \newpage

%%%%%%% the main content %%%%%%%%%
  \pagestyle{main}
  \setcounter{page}{1}

  \section[make symbolic list]{make symbolic list}
  定义一个macro,该macro生成一个symbol列表。listq之于list,
  恰好是Emacs lisp中的setq之于set。

  \begin{schemecode}
    (defmacro listq
    "make a symbol list.
    
    listq vs list, just like setq vs set in Emacs lisp."
    [& args]
    `'~args)    
  \end{schemecode}

  注意"`"(syntax-quote),该语法应用于clojure reader,是在read或者编译时使用的,
  不是执行期使用的。

  \section[高阶函数举例]{高阶函数举例}
  常用高阶函数:

  \subsection[function map]{function map}
  语法:\par
  (map fn c1 \& coll)

  map函数接收一个函数以及任意多个collection作为参数,返回一个列表,要求函数fn的
  参数个数与collections个数相等,map依次从各个colls中取出一个元素,
  然后把这些元素作为参数传递给fn,返回值最为最终结果(一个列表)的元素,
  如果colls长度不一致,map按照最短的coll计算。

  例如,向量加法(向量相加还是向量):\par
  (map + [1 2 3 4] [2 3 4 1] [3 4 1 2] [4 3 2 1])\par
  四个向量相加。

  \subsection[function apply]{function apply}
  语法:\par
  (apply fn coll)

  将coll的所有元素”一次性“传递给fn,然后返回fn的结果。

  例如,将一个coll中的所有元素相加:\par
  (apply + [1 2 3 4])
  
  \subsection[function reduce]{function reduce}
  语法:\par
  (reduce fn coll)\\
  (reduce fn seed coll)

  函数fn必须可以接收两个参数,如果seed没有提供,就取coll的前两个元素作为
  fn的参数,返回值和第三个参数再次传入fn,依次类推,直到遍历coll。如果提供seed,
  就取seed作为第一个参数,coll的第一个元素作为第二个参数传入fn,返回值和第二个元素
  再次传入fn,依次类推直到遍历coll。

  例如,求向量的模($\sqrt{x_1^2+x_2^2+x_3^3+\cdots+x_n^2}$):\par
  (reduce \#(+ \%1 (* \%2 \%2)) 0 [1 2 3 4])
  
  \subsection[function filter]{function filter}
  语法:\par
  (filter predicate-fn coll)

  filter函数依次取coll的元素,传递给predicate-fn,如果该函数返回true
  就留下这个元素,否则就丢掉,结果是一个所有元素都满足predicate-fn的列表。

  例如,取出所有的偶数:\par
  (filter enven? [1 2 3 4 5 6])

  \section[使用自定义的java库和clojure库]{使用自定义的java库和clojure库}
  clojure可以轻松使用自定义的Java库和自定义的Clojure库。
  \subsection[自定义Java库]{自定义Java库}
  一般java库都是以.jar文件的形式发布,从clojure调用自定义java库只需:

  将.jar文件放到classpath路径下,如启动repl时使用参数-cp加载,标准的java方式均可,
  然后从clojure代码直接调用即可,如:\\
  (. org.ybin.Util greetFromJava)\\
  调用org.ybin这个java package里的Util这个类的名为greetFromJava的静态方法。

  \subsection[自定义clojure库]{自定义clojure库}
  自定义clojure库同样很简单,只需:

  1. 将clj文件,如core.clj,放到classpath下的目录中\\
  2. 设置正确的路径,如:我把sybUtil/这个目录加到classpath中,
  然后在sybUtil/org/ybin/目录下放置core.clj文件,该文件中定义
  namespace为(ns org.ybin.core),一个文件即为一个namespace,
  使用时先require:(require 'org.ybin.core),然后调用:(org.ybin.core/greet-from-clj),调用该
  namespace中的greet-from-clj函数。

  \section[clojure访问java enum元素]{clojure访问java enum元素}
  java enum定义:\par
  \begin{javacode}
public class JCLASS {
  static MyEnum {
    FIRST,
    SECOND,
    THIRD
  }
}
  \end{javacode}

  clojure访问java enum:\par
  \begin{schemecode}
(println JCLASS$MyEnum/FIRST)
\end{schemecode}


  \section[clojure.java.io]{clojure.java.io}
  clojure中如何处理java io。
  \subsection[reader]{reader}
  reader默认返回一个java.io.BufferedReader。
  它可以读取Reader, BufferedReader, InputStream, File,
  URI, URL, Socket, byte arrays, character arrays以及String.

  如果参数为一个字符串,reader首先尝试按照URI解析,再尝试按照
  local file name解析。

  reader应该在with-open中使用,以确保其被正确关闭。

  示例:\\
  \begin{schemecode}
    (with-open [rdr (clojure.java.io/reader "http://www.baidu.com")]
      (println (clojure.string/join "\n" (line-seq rdr))))
  \end{schemecode}

  另一个函数clojure.core/slurp跟reader接受同样的参数,但是它不会缓开的内容,
  而是一次性把所有内容读入内存,然后返回一个字符串(注意编码方式),所以当内容不被考虑的情况下,
  可以使用slurp。
  
  \begin{schemecode}
    (println (str (slurp "http://www.baidu.com" :encoding "UTF-8")))
  \end{schemecode}

  
\end{document}