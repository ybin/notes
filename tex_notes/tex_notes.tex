\documentclass{article}


%%% fontenc
%\usepackage{fontspec,xunicode,xltxtra}
%\setmainfont{Times New Roman}
%\setsansfont{Source Sans Pro}
%\setmonofont{Source Sans Pro}

%%% xeCJK
\usepackage{xeCJK}
\setCJKmainfont[BoldFont=Adobe Heiti Std]{Adobe Song Std}
\setCJKsansfont[BoldFont=Adobe Heiti Std]{Adobe Song Std}
\setCJKmonofont[BoldFont=Adobe Heiti Std]{Adobe Song Std}
\XeTeXlinebreaklocale "zh"
\XeTeXlinebreakskip=0pt plus 1pt minus 0.1pt

\usepackage{xcolor}
\usepackage{graphicx}

%%% get total page number
\usepackage{lastpage}

%%% customized definition
\makeatletter
\def\sybtitle#1{\def\@sybtitle{#1}}
\def\sybauthor#1{\def\@sybauthor{#1}}
\def\sybdate#1{\def\@sybdate{#1}}
\sybtitle{}
\sybauthor{}
\sybdate{}
\def\sybmaketitle{
  \begin{center}
  \vspace*{.8in}
  {\huge\bfseries\@sybtitle}
  \par
  \vspace{.8in}
  {\Large\@sybauthor}
  \par
  \vspace{.2in}
  \@sybdate
  \vspace{.5in}
  \end{center}
}
\makeatother
\setlength{\parindent}{0pt}
\renewcommand{\today}{\number\month 月 \number\day 日, ~\number\year 年}
\def\lt{\textless}
\def\gt{\textgreater}
\renewcommand\contentsname{\bfseries 目~~录}
\newcommand\bs{\texttt{\symbol{'134}}} % input backslash sign
%\newcommand\bs{\string\} % same as above definition
\long\def\cmd#1{\par\vspace{.5em}\hspace*{2em}#1\vspace{.5em}\par}
\def\cstr#1{\texttt{\string#1}} % e.g. \cstr{\latex}
\long\def\runcode#1{\par\bigskip#1\bigskip\par}
% 我不想看到那么多的underful hbox,尤其是minted环境加上背景色之后
\hbadness=10000
% 适当放宽overful hbox的限制,运行2pt的溢出
\hfuzz=2pt
\parskip=3\lineskip


%%% change background color & add frame for enumerate enviroment
\usepackage{mdframed}
\newmdenv[backgroundcolor=blue!10,linewidth=0pt]{coloredframe}
\newenvironment{coloredenumerate}{
  \begin{coloredframe}
  \begin{enumerate}
}{
  \end{enumerate}
  \end{coloredframe}
}

%%% geometry
\usepackage[includehead,includefoot,hmargin=21mm,vmargin=10.5mm,
            headsep=12pt,headheight=25pt]{geometry}
%\usepackage[includehead,includefoot,hmargin=1.2in,vmargin=1in]{geometry}

%%% fancyhdr
\usepackage{fancyhdr}
\makeatletter
\fancypagestyle{main} {
  \fancyhf{} % clear header & footer
  \fancyhead[L]{\bfseries\@sybtitle}
  \fancyhead[R]{\thepage/\pageref*{LastPage}}
  \renewcommand{\headrulewidth}{0.4pt} % header line
  \renewcommand{\footrulewidth}{0pt} % footer line
}
\fancypagestyle{header} {
  \fancyhf{} % clear header & footer
  \fancyfoot[C]{\roman{page}}
  \renewcommand{\headrulewidth}{0pt} % header line
  \renewcommand{\footrulewidth}{0pt} % footer line
}
\makeatother

\usepackage{titlesec}
\titleformat{\part}{\centering\Large\bfseries}{第\,\thepart\,部分}{1em}{}
\titleformat{\section}{\large\bfseries}{\thesection}{1em}{}
\titleformat{\subsection}{\normalsize\bfseries}{\thesubsection}{1em}{}
%\titlespacing*{章节命令}{左边距}{上文距}{下文距}[右边距]
\titlespacing*{\section}{0pt}{2\baselineskip}{\parsep}


\usepackage{hyperref}

%%% perfect source code display
\usepackage{minted}
%\usemintedstyle{colorful}
\definecolor{srcbg}{rgb}{0.95,0.95,0.95}
\newminted{java}{linenos,tabsize=4,bgcolor=srcbg}
\newminted{xml}{linenos,tabsize=4,bgcolor=srcbg}
\newminted{cpp}{linenos,tabsize=4,bgcolor=srcbg}
\newminted{bash}{linenos,tabsize=4,bgcolor=srcbg}
\newminted{latex}{linenos,tabsize=4,bgcolor=srcbg}



\sybtitle{\TeX{} family Notes}
\sybauthor{孙延宾}
\sybdate{\today}


\begin{document}
  \tt % I love Typewriter font.

%%%%%%%% the title page and toc %%%%%%%%%%
  \pagestyle{header}
  \sybmaketitle
  \tableofcontents
  \newpage

%%%%%%% the main content %%%%%%%%%
  \pagestyle{main}
  \setcounter{page}{1}

  \part[tex和plain tex笔记]{tex和plain tex笔记}
  \section[newif macro]{\bs newif macro}
  \bs newif 宏语法:
  \cmd{\bs newif\bs if\lt csname\gt}
  %\\[.5em]
  %\hspace*{4ex}\bs newif\bs if\lt csname\gt \\[.5em]
  该定义一次性定义了三个宏:
  \begin{description}
    \item[\bs if\lt csname\gt:] 条件判断语句
    \item[\bs\lt csname\gt true:] 调用这个宏会使得\bs if\lt csname\gt 中的条件判断为"true"
    \item[\bs\lt csname\gt false] 调用这个宏会使得\bs if\lt csname\gt 中的条件判断为"false"
  \end{description}
  下面是一个示例:

  \begin{latexcode}
\newif\ifboy
\newif\ifgirl
\ifboytrue
\ifboy{I love you, son!}\fi
\ifgirl{I love you, daughter!}\fi
  \end{latexcode}

  运行结果:
  \newif\ifboy
  \newif\ifgirl
  \boytrue
  \ifboy{I love you, son!}\fi
  \ifgirl{I love you, daughter!}\fi

  \section[loop macro]{loop macro}
  \bs loop宏的语法:
  \cmd{\bs loop\lt control sequences\gt \lt conditional sequence\gt \lt control sequences\gt \bs repeat}
  首先执行\bs loop后面的控制序列,然后,如果条件判断为true,就接着执行后面的控制序列,并循环之;
  如果条件判断为false,就退出循环。

  下面是一个嵌套循环的示例:

  \begin{latexcode}
\vbox{
  \count100=9
  \loop
    \count101=65 % ASCII `A'
    \advance\count100 by-1
    \hbox{% \loop的参数中不允许有\par等换行命令
          % 这里我们利用Tex的"模式"进行排版:
          % TeX默认进入vertical mode,然后使用\hbox进入stricted horizontal mode
          % 从而避免了换行命令。
      \loop
      \char\count101 \the\count100
      \advance\count101 by1
      \ifnum\count101<73
      \space
      \repeat
    }
  \ifnum\count100>0
  \repeat
}
  \end{latexcode}

  替换计数寄存器之后,代码更易读:

  \begin{latexcode}
\vbox{
  \newcount\num % \loop的参数中不能使用\newcount宏
  \newcount\chr % 不能使用\par宏
  \num=9
  \loop
    \chr=65 % ASCII `A'
    \advance\num by-1
    \hbox{% \loop的参数中不允许有\par等换行命令
          % 这里我们利用Tex的"模式"进行排版:
          % TeX默认进入vertical mode,然后使用\hbox进入stricted horizontal mode
          % 从而避免了换行命令。
      \loop
      \char\chr \the\num
      \advance\chr by1
      \ifnum\chr<73
      \space
      \repeat
    }
  \ifnum\num>0
  \repeat
}
  \end{latexcode}

  运行结果:
  \runcode{
    \vbox{
      \count100=9
      \loop
        \count101=65 % ASCII `A'
        \advance\count100 by-1
        \hbox{% \loop的参数中不允许有\par等换行命令
              % 这里我们利用Tex的"模式"进行排版:
              % TeX默认进入vertical mode,然后使用\hbox进入stricted horizontal mode
              % 从而避免了换行命令。
          \loop
          \char\count101 \the\count100
          \advance\count101 by1
          \ifnum\count101<73
          \space
          \repeat
        }
      \ifnum\count100>0
      \repeat
    }
  }

  \section[文本的分散对齐]{文本的分散对齐}
  在\TeX{}中,一行文本是要充满行宽的,在hbox中,文本要充满box的宽度,否则
  \TeX{}的断段成行算法也就可以省略了。所以分散对齐其实是没有必要的,\TeX{}
  一直都是这么做的,但是如果一行中只有两个单词,为何没有在行端各放置一个呢?
  这是因为一段的最后一行时,\TeX{}会自动将该行的\bs hbox长度设置为文本的自然长度。
  这里我们绕过\TeX{}的断行机制,使用\bs hbox来做实验。\par
  \begin{latexcode}
\hbox{word1 word2 word3}\par
\hbox to \hsize{word1 word2 word3}\par
\hbox spread 1in{word1 word2 word3}
  \end{latexcode}

  结果如下:
  \runcode{
    \hbox{word1 word2 word3}\par
    \hbox to \hsize{word1 word2 word3}\par
    \hbox spread 1in{word1 word2 word3}
  }
  \bs hbox的to参数设置box的宽度;spread参数将box的宽度在文本自然宽度的基础上增加一段。

  另外,\bs break宏告诉\TeX{}“在此处断行”,之前的行宽将为\bs hsize,之后的文本视其
  是否为段尾行而定,所以\bs break宏有可能造成之前的行填充不满而出现“分散对齐”的现象。
  此时可以使用\bs filbreak,它的功能相对于\bs hfil\bs break,于是之前的行就会左对齐了。

  \cstr{\tolerance}, \cstr{\penalty}, \cstr{\hfuzz}, \cstr{\hbadness}的区别!


  \section[显示特殊字符]{显示特殊字符}
  像“\textvisiblespace”(空格)、“\bs”(backslash,反斜杠),甚至“任何字符”都可以这样来显示:\\
  在\TeX{}中:
  \begin{enumerate}
    \item \bs char\lt int\gt,如\bs char92(backslash),\{\bs tt\bs char32\}
    \item \bs char\lt'octal\gt,如\bs char'134
    \item \bs char\lt"hex\gt,如\bs char"5C
    \item \bs chardef\bs cs=\lt int\gt
  \end{enumerate}
  空格符在有些字体中没有,所以使用\bs tt字体。

  在\LaTeX{}中:
  \begin{enumerate}
    \item \bs symbol\{number\},如\bs symbol\{'134\}
  \end{enumerate}
  但是在\LaTeX{}中,空格符显示不出来,必须使用\bs textvisiblespace 才行。

  \section[rules]{rules}
  rule就是一个黑色盒子,表现出来就是一个黑色的格子。
  \begin{description}
    \item[\bs vrule] 水平方向增长的黑格子,默认宽度为0.4pt,默认高度为0pt
    \item[\bs hrule] 竖直方向增长的黑格子,默认高度为0.4pt,默认宽度为\bs hsize
  \end{description}
  示例:\\
  \bs vrule width 1em height 1ex : \vrule width 1em height 1ex\\
  \bs hrule width 1em height .4pt : \hrule width 1em height .4pt\par
  \bs hrule 默认是从行首开始的,所以黑格子跑到行首去了,而且
  它始终是在baseline下面的。


  \section[特殊变量]{特殊变量}
  下面是一些特殊变量的说明及使用方法。

  \subsection[hbadness]{\bs hbadness}
  \bs hbox的badness超过hbadness时,\TeX{}会报告警告信息,
  如\bs hbox overful或者underful了。默认值是200,
  10000被认为是无限大,所以,如果不想看到那些烦人的警告信息,
  可以这样设置:
  \cmd{\bs hbadness=10000}
  \bs vbadness与此类似。

  \subsection[hfuzz]{\bs hfuzz}
  \bs hfuzz告诉\TeX{}:\bs hbox超出\bs hfuzz长度并不算overful。
  所以如果\bs hbox超出长度不超过\bs hfuzz时,就不算是overful,
  \TeX{}就不会报告警告信息:
  \cmd{\bs hfuzz=2pt}
  \bs vfuzz与此类似。


  \section[寄存器]{寄存器}
  \TeX{}预先定义有count、dimen、skip、toks等几种寄存器,每种寄存器都有
  0-255共256个。

  \subsection[count寄存器]{count寄存器}
  可以直接使用,如:
  \cmd{\bs count42=1024}
  也可以使用\bs countdef为count寄存器取个名字:
  \cmd{\bs countdef\bs mycount=42}
  然后就可以用同样的方法使用该寄存器了:
  \cmd{\bs mycount=1024}
  然而,并不推荐直接使用\bs countdef指令,如果两个package
  都使用该指令将同一个count寄存器取了两个名字,就会造成混乱,
  可以使用plain\TeX{}的\bs newcount指令:
  \cmd{\bs newcount\bs mynewcount}
  使用者并不知道定义的寄存器跟256个中的哪个对应,
  但是\bs newcount可以保证不会重复分配。

  \section[行间距与段间距]{行间距与段间距}
  在\TeX{}中是通过变量来控制行间距和段间距的。
  \subsection[行间距]{行间距}
  行间距用下面几个变量来控制:
  \cmd{\bs baselineskip, \bs prevdepth, \bs lineskiplimit, \bs lineskip}
  \begin{coloredenumerate}
    \item \bs baselineskip: 两个行的baseline之间的距离,类型为glue
    \item \bs prevdepth: 上一行组成的box的深度(depth)
    \item \bs lineskiplimit: 一个距离值
    \item \bs lineskip: 一个距离值
  \end{coloredenumerate}
  它们之间的关系如下:\par
  \begin{latexcode}
    \def\gap{}
    \L = \baselineskip - \prevdepth - "current line height"
    \if \L < \lineskiplimit
      \gap = \L
    \else
      \gap = \lineskip
  \end{latexcode}
  然后,\TeX{}先将\bs gap高度的空白放置到当前位置,再将当前行放置在空白之下,
  这样就完成了在两行之间插入空白的任务。
  \subsection[段间距]{段间距}
  设置段间距为一个行高距离(两个baseline之间的距离):
  \cmd{\bs parskip = \bs baselineskip}
  设置段落缩进为两个字符宽度:
  \cmd{\bs parindent = 2em}

  


  \part[tikz笔记]{tikz笔记}

  \section[使用tikz画树状图]{使用tikz画树状图}
  \section[使用tikz画matrix]{使用tikz画matrix}

  \part[beamer笔记]{beamer笔记}

  %\expandafter\show\csname LaTeX \endcsname
\end{document}
