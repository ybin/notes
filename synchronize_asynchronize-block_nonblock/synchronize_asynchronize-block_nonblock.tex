
\documentclass[a4paper,11pt]{article}


%%% fontenc
%\usepackage{fontspec,xunicode,xltxtra}
%\setmainfont{Times New Roman}
%\setsansfont{Source Sans Pro}
%\setmonofont{Source Sans Pro}

%%% xeCJK
\usepackage{xeCJK}
\setCJKmainfont[BoldFont=Adobe Heiti Std]{Adobe Song Std}
\setCJKsansfont[BoldFont=Adobe Heiti Std]{Adobe Song Std}
\setCJKmonofont[BoldFont=Adobe Heiti Std]{Adobe Song Std}
\XeTeXlinebreaklocale "zh"
\XeTeXlinebreakskip=0pt plus 1pt minus 0.1pt

\usepackage{xcolor}
\usepackage{graphicx}

%%% get total page number
\usepackage{lastpage}

%%% customized definition
\makeatletter
\def\sybtitle#1{\def\@sybtitle{#1}}
\def\sybauthor#1{\def\@sybauthor{#1}}
\def\sybdate#1{\def\@sybdate{#1}}
\sybtitle{}
\sybauthor{}
\sybdate{}
\def\sybmaketitle{
  \begin{center}
  \vspace*{.8in}
  {\huge\bfseries\@sybtitle}
  \par
  \vspace{.8in}
  {\Large\@sybauthor}
  \par
  \vspace{.2in}
  \@sybdate
  \vspace{.5in}
  \end{center}
}
\makeatother
\setlength{\parindent}{0pt}
\renewcommand{\today}{\number\month 月 \number\day 日, ~\number\year 年}
\def\lt{\textless}
\def\gt{\textgreater}
\renewcommand\contentsname{\bfseries 目~~录}
\newcommand\bs{\texttt{\symbol{'134}}} % input backslash sign
%\newcommand\bs{\string\} % same as above definition
\long\def\cmd#1{\par\vspace{.5em}\hspace*{2em}#1\vspace{.5em}\par}
\def\cstr#1{\texttt{\string#1}} % e.g. \cstr{\latex}
\long\def\runcode#1{\par\bigskip#1\bigskip\par}
% 我不想看到那么多的underful hbox,尤其是minted环境加上背景色之后
\hbadness=10000
% 适当放宽overful hbox的限制,运行2pt的溢出
\hfuzz=2pt
\parskip=3\lineskip


%%% change background color & add frame for enumerate enviroment
\usepackage{mdframed}
\newmdenv[backgroundcolor=blue!10,linewidth=0pt]{coloredframe}
\newenvironment{coloredenumerate}{
  \begin{coloredframe}
  \begin{enumerate}
}{
  \end{enumerate}
  \end{coloredframe}
}

%%% geometry
\usepackage[includehead,includefoot,hmargin=21mm,vmargin=10.5mm,
            headsep=12pt,headheight=25pt]{geometry}
%\usepackage[includehead,includefoot,hmargin=1.2in,vmargin=1in]{geometry}

%%% fancyhdr
\usepackage{fancyhdr}
\makeatletter
\fancypagestyle{main} {
  \fancyhf{} % clear header & footer
  \fancyhead[L]{\bfseries\@sybtitle}
  \fancyhead[R]{\thepage/\pageref*{LastPage}}
  \renewcommand{\headrulewidth}{0.4pt} % header line
  \renewcommand{\footrulewidth}{0pt} % footer line
}
\fancypagestyle{header} {
  \fancyhf{} % clear header & footer
  \fancyfoot[C]{\roman{page}}
  \renewcommand{\headrulewidth}{0pt} % header line
  \renewcommand{\footrulewidth}{0pt} % footer line
}
\makeatother

\usepackage{titlesec}
\titleformat{\part}{\centering\Large\bfseries}{第\,\thepart\,部分}{1em}{}
\titleformat{\section}{\large\bfseries}{\thesection}{1em}{}
\titleformat{\subsection}{\normalsize\bfseries}{\thesubsection}{1em}{}
%\titlespacing*{章节命令}{左边距}{上文距}{下文距}[右边距]
\titlespacing*{\section}{0pt}{2\baselineskip}{\parsep}


\usepackage{hyperref}

%%% perfect source code display
\usepackage{minted}
%\usemintedstyle{colorful}
\definecolor{srcbg}{rgb}{0.95,0.95,0.95}
\newminted{java}{linenos,tabsize=4,bgcolor=srcbg}
\newminted{xml}{linenos,tabsize=4,bgcolor=srcbg}
\newminted{cpp}{linenos,tabsize=4,bgcolor=srcbg}
\newminted{bash}{linenos,tabsize=4,bgcolor=srcbg}
\newminted{latex}{linenos,tabsize=4,bgcolor=srcbg}
\newminted{scheme}{linenos,tabsize=4,bgcolor=srcbg}

\usepackage{amsmath}


\input{../styles/tikz_preamble}

\sybtitle{同步、异步,阻塞、非阻塞}
\sybauthor{孙延宾}
\sybdate{\today}

\begin{document}
\tt % I love Typewriter font.
%%%%%%%% the title page and toc %%%%%%%%%%
\pagestyle{header}
\sybmaketitle
%\tableofcontents
\newpage

%%%%%%% the main content %%%%%%%%%
\pagestyle{main}
\setcounter{page}{1}

\section[术语]{术语}
\begin{itemize}
  \item 同步:synchronize
  \item 异步:asynchronize
  \item 阻塞:block
  \item 非阻塞:nonblock
\end{itemize}

\section[一个故事]{一个故事}
故事来源于网络,作者未知。

老张爱喝茶,经常烧水沏茶。下面是他烧水时的情景,
\begin{description}
  \item[情形1: ] 烧上水,然后站在炉灶前什么都不干,盯着水壶等水烧开。
  \item[情形2: ] 烧上水,然后发呆(或瞌睡)一会儿、检查水壶,再发呆一会儿、检查水壶,直到水烧开。
\end{description}
不论哪种情形,老张都要等到水烧开了才会做其他事情。

可是这样烧水实在太浪费时间,老张决定换种方式。他想水烧上之后不必一定等水开了才做其他事情,
毕竟水烧开需要一段时间,这段时间他完全可以做点儿别的,于是老张作了一下调整,
\begin{description}
  \item[情形3: ] 烧上水,然后浇花、检查水壶,喂鸟、检查水壶,看会儿电视、检查水壶....
\end{description}
这样,在烧水的这段时间里老张又做了很多事情。可是老张仍然觉得不满意,一次次检查水壶,
让他感觉很累人,还有没有别的方法呢?于是老张换了一个带哨子的水壶,水烧开后自动鸣哨,
这样他就不用一次次检查水壶了,
\begin{description}
  \item[情形4: ] 烧上水,然后浇花、喂鸟、(安心的)看电视,水烧开后自动鸣哨通知老张。
\end{description}

\section[抽象理解]{抽象理解}
\emph{同步、异步}是描述“消息通知方式”的。调用者发起调用,被调用者执行并结果返回给调用者。
我们以函数调用为例,当然也可以是网络通讯、文件IO等等。
\begin{itemize}
  \item 调用者发起函数调用,直到收到被调用者的执行结果,这期间不进行其他函数调用,此为同步。如情形1、情形2
  \item 调用者发起函数调用,直到收到被调用者的执行结果,这期间进行其他函数调用(提供效率啊),此为异步。
        异步需要额外的机制来确保调用者收到被调用者的结果,可以是轮询(如情形3),可以是回调(如情形4),也可以是其他机制。
\end{itemize}
更通俗一点儿说,同步是顺序演进而异步是乱序演进。

这里的执行结果不是函数返回值,而是实际想要的结果(如果文件读取时的数据)。

但是,无论同步还是异步,被调用者执行函数而得到结果所需的时间是固定的,同步调用时,
这期间调用者没有其他函数调用,那它会有怎样的行为呢?
\begin{itemize}
  \item 在此期间,调用者什么都不做,死等,那么就说发起者是阻塞的。如情形1
  \item 在此期间,调用者一次次检查是否有结果到来,那么就说发起者是非阻塞的。如情形2
\end{itemize}
由此可见,

\emph{阻塞、非阻塞}是描述发起者“等待状态”的。同步通讯时,调用者在等待期间可以阻塞也可以非阻塞,
阻塞与否只关乎调用者而与被调用者无关。而同步、异步的区分似乎也只跟调用者是否开启新任务有关,实则
不然,因为通过回调方式实现异步通讯时也需要被调用者配合。轮询(情形3中的轮询而非情形2中的轮询)、
回调、Event Loop都是异步通讯的实现方式。

总结,同步、异步,阻塞、非阻塞是两组概念,他们的关注点不同而使得他们之间没有任何关系,
当然两组之间可以搭配使用,没有必要强行比对这两组概念。

注意:异步通讯时,调用者也可以阻塞自己等待结果,但是,太傻了是不是?!

\end{document}
