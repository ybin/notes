
\documentclass[a4paper,11pt]{article}


%%% fontenc
%\usepackage{fontspec,xunicode,xltxtra}
%\setmainfont{Times New Roman}
%\setsansfont{Source Sans Pro}
%\setmonofont{Source Sans Pro}

%%% xeCJK
\usepackage{xeCJK}
\setCJKmainfont[BoldFont=Adobe Heiti Std]{Adobe Song Std}
\setCJKsansfont[BoldFont=Adobe Heiti Std]{Adobe Song Std}
\setCJKmonofont[BoldFont=Adobe Heiti Std]{Adobe Song Std}
\XeTeXlinebreaklocale "zh"
\XeTeXlinebreakskip=0pt plus 1pt minus 0.1pt

\usepackage{xcolor}
\usepackage{graphicx}

%%% get total page number
\usepackage{lastpage}

%%% customized definition
\makeatletter
\def\sybtitle#1{\def\@sybtitle{#1}}
\def\sybauthor#1{\def\@sybauthor{#1}}
\def\sybdate#1{\def\@sybdate{#1}}
\sybtitle{}
\sybauthor{}
\sybdate{}
\def\sybmaketitle{
  \begin{center}
  \vspace*{.8in}
  {\huge\bfseries\@sybtitle}
  \par
  \vspace{.8in}
  {\Large\@sybauthor}
  \par
  \vspace{.2in}
  \@sybdate
  \vspace{.5in}
  \end{center}
}
\makeatother
\setlength{\parindent}{0pt}
\renewcommand{\today}{\number\month 月 \number\day 日, ~\number\year 年}
\def\lt{\textless}
\def\gt{\textgreater}
\renewcommand\contentsname{\bfseries 目~~录}
\newcommand\bs{\texttt{\symbol{'134}}} % input backslash sign
%\newcommand\bs{\string\} % same as above definition
\long\def\cmd#1{\par\vspace{.5em}\hspace*{2em}#1\vspace{.5em}\par}
\def\cstr#1{\texttt{\string#1}} % e.g. \cstr{\latex}
\long\def\runcode#1{\par\bigskip#1\bigskip\par}
% 我不想看到那么多的underful hbox,尤其是minted环境加上背景色之后
\hbadness=10000
% 适当放宽overful hbox的限制,运行2pt的溢出
\hfuzz=2pt
\parskip=3\lineskip


%%% change background color & add frame for enumerate enviroment
\usepackage{mdframed}
\newmdenv[backgroundcolor=blue!10,linewidth=0pt]{coloredframe}
\newenvironment{coloredenumerate}{
  \begin{coloredframe}
  \begin{enumerate}
}{
  \end{enumerate}
  \end{coloredframe}
}

%%% geometry
\usepackage[includehead,includefoot,hmargin=21mm,vmargin=10.5mm,
            headsep=12pt,headheight=25pt]{geometry}
%\usepackage[includehead,includefoot,hmargin=1.2in,vmargin=1in]{geometry}

%%% fancyhdr
\usepackage{fancyhdr}
\makeatletter
\fancypagestyle{main} {
  \fancyhf{} % clear header & footer
  \fancyhead[L]{\bfseries\@sybtitle}
  \fancyhead[R]{\thepage/\pageref*{LastPage}}
  \renewcommand{\headrulewidth}{0.4pt} % header line
  \renewcommand{\footrulewidth}{0pt} % footer line
}
\fancypagestyle{header} {
  \fancyhf{} % clear header & footer
  \fancyfoot[C]{\roman{page}}
  \renewcommand{\headrulewidth}{0pt} % header line
  \renewcommand{\footrulewidth}{0pt} % footer line
}
\makeatother

\usepackage{titlesec}
\titleformat{\part}{\centering\Large\bfseries}{第\,\thepart\,部分}{1em}{}
\titleformat{\section}{\large\bfseries}{\thesection}{1em}{}
\titleformat{\subsection}{\normalsize\bfseries}{\thesubsection}{1em}{}
%\titlespacing*{章节命令}{左边距}{上文距}{下文距}[右边距]
\titlespacing*{\section}{0pt}{2\baselineskip}{\parsep}


\usepackage{hyperref}

%%% perfect source code display
\usepackage{minted}
%\usemintedstyle{colorful}
\definecolor{srcbg}{rgb}{0.95,0.95,0.95}
\newminted{java}{linenos,tabsize=4,bgcolor=srcbg}
\newminted{xml}{linenos,tabsize=4,bgcolor=srcbg}
\newminted{cpp}{linenos,tabsize=4,bgcolor=srcbg}
\newminted{bash}{linenos,tabsize=4,bgcolor=srcbg}
\newminted{latex}{linenos,tabsize=4,bgcolor=srcbg}
\newminted{scheme}{linenos,tabsize=4,bgcolor=srcbg}

\usepackage{amsmath}


\input{../styles/tikz_preamble}

\sybtitle{同步、异步,阻塞、非阻塞}
\sybauthor{孙延宾}
\sybdate{\today}

\begin{document}
\tt % I love Typewriter font.
%%%%%%%% the title page and toc %%%%%%%%%%
\pagestyle{header}
\sybmaketitle
%\tableofcontents
\newpage

%%%%%%% the main content %%%%%%%%%
\pagestyle{main}
\setcounter{page}{1}

\section[术语]{术语}
\begin{itemize}
  \item 同步:synchronize
  \item 异步:asynchronize
  \item 阻塞:block
  \item 非阻塞:nonblock
\end{itemize}

\section[一个故事]{一个故事}
故事来源于网络,作者未知。

老张爱喝茶,经常烧水沏茶。下面是他烧水时的情景,
\begin{description}
  \item[情形1: ] 烧上水,然后站在炉灶前什么都不干,等水烧开。
  \item[情形2: ] 烧上水,然后去客厅看电视,看一会儿就去检查水是否开了。烧一壶水有可能检查多次。
\end{description}
老张觉得这样烧水太累了,在炉灶前一直等累得慌,一次次检查也不轻松,
于是他决定买个高级的水壶,这种水壶烧开水后自己鸣哨,听到哨声他就知道水开了。
于是老张烧水的情形变成了这样,
\begin{description}
  \item[情形3: ] 烧上水,然后去客厅看电视,水烧开后自动鸣哨通知老张。
  \item[情形4: ] 烧上水,老张依然站在炉灶前等水烧开。(老张说自己不傻,这种情形事实上从未发生过)
\end{description}
从老张的角度看,烧水时他什么都不干而是专心等水烧开,那么他就是把自己阻塞(block)了,
而如果他去做别的事情(不管是再次检查水是否烧开,还是一直看电视直到水壶鸣哨),他都没有把自己阻塞住,
此时老张是非阻塞的。

而从水壶的角度看,普通的水壶水开后并不鸣哨,整个烧水过程是同步的,不管老张一直等着还是一次次跑来检查,
这个过程都是同步的。对于高级水壶,水开后自动鸣哨通知,整个烧水过程是异步的,不管老张等还是不等还是
一次次检查,烧水过程都是异步的。

\section[抽象理解]{抽象理解}
将参与事件的双方抽象为\emph{调用者}(老张)和\emph{被调用者}(水壶),
那么就可以这样区分这四个术语,
\begin{itemize}
  \item 从\emph{被调用者}角度可分为:
    \begin{enumerate}
      \item 同步:\emph{被调用者}做完事情,并不主动通知\emph{调用者}。
      \item 异步:\emph{被调用者}做完事情,主动通知\emph{调用者}(通过某种机制,如回调、鸣哨)。
    \end{enumerate}
  \item 从\emph{调用者}角度可分为:
    \begin{enumerate}
      \item 阻塞:\emph{调用者}啥也不做,就等着调用结果,死等。
      \item 非阻塞:\emph{调用者}并不死等结果而是做别的事情,必要时(如跟同步机制结合时)隔一段时间再来主动检查调用结果。
    \end{enumerate}
\end{itemize}
同步、异步和阻塞、非阻塞是两组概念,这两组之间没有必然的联系,
但是一般情况下,同步通讯时进一步结合阻塞、非阻塞方式,因为同步机制
规定了\emph{被调用者}在完成事情之后并不主动通知\emph{调用者},
所以调用者要采取措施获取调用结果,要么把自己阻塞,死等结果,
要么抽空做点儿别的,过后再来检查调用结果。而异步通讯时,由于
\emph{被调用者}做完事情之后会主动通知\emph{调用者},所以
\emph{调用者}完全没有必要阻塞自己,放心去做别的事情好了,
即\emph{调用者}一定是非阻塞的。当然
此时\emph{调用者}也可以什么都不做而是专心等结果,但是这样就太傻了。

综合起来,通讯机制可以分为:同步阻塞、同步非阻塞、异步。

\end{document}
