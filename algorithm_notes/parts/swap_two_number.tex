\section[Swap two numbers]{两数互换}
互换两个数的值,常规做法是使用中间变量,但是利用位操作却可以省去中间变量,
该操作与语言无关,只用到位操作“xor”,

\begin{cppcode}
// C语言版实现
int i = 42;
int j = 24;
printf("i=%d, j=%d\n", i, j);
i = i ^ j;
j = i ^ j;
i = i ^ j;
// 更加简洁的实现如下
// i ^= j = ^= i ^= j;
printf("i=%d, j=%d\n", i, j);
\end{cppcode}

原理是什么?

"xor"操作与顺序无关,即x <xor> y == y <xor> x,不过,为了便于理解,我们
称其中一个操作位为“操作因子”,

\centerline{x <xor> factor = y}

我们说:x经操作因子factor,变为y。观察factor的bit,如果是1表示“改变”,
如果是0表示“不变”,于是factor就变成了一个映射的对应关系,x、y分别是
由bit位组成的集合,并且该映射是一一对应的。

进一步,考虑到二进制的特殊性,“改变”无非就是0变1或者1变0,于是任意一个bit
经过两次“改变”就变回了自身,于是,

\centerline{(x <xor> factor) <xor> factor == x}

于是我们得出这样的结论:

\begin{cppcode}
x <xor> y = z
x <xor> z = y // x <xor> (x <xor> y) == y 经两次操作变回自身
y <xor> z = x
\end{cppcode}